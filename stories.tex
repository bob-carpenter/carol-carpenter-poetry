% Options for packages loaded elsewhere
\PassOptionsToPackage{unicode}{hyperref}
\PassOptionsToPackage{hyphens}{url}
%
% DOCUMENT CLASS
% =============================================
\documentclass[twoside,10pt]{book}
\usepackage{lmodern}
\usepackage{amssymb,amsmath}
\usepackage{ifxetex,ifluatex}
\ifnum 0\ifxetex 1\fi\ifluatex 1\fi=0 % if pdftex
  \usepackage[T1]{fontenc}
  \usepackage[utf8]{inputenc}
  \usepackage{textcomp} % provide euro and other symbols
\else % if luatex or xetex
  \usepackage{unicode-math}
  \defaultfontfeatures{Scale=MatchLowercase}
  \defaultfontfeatures[\rmfamily]{Ligatures=TeX,Scale=1}
\fi
% Use upquote if available, for straight quotes in verbatim environments
\IfFileExists{upquote.sty}{\usepackage{upquote}}{}
\IfFileExists{microtype.sty}{% use microtype if available
  \usepackage[]{microtype}
  \UseMicrotypeSet[protrusion]{basicmath} % disable protrusion for tt fonts
}{}
\makeatletter
\@ifundefined{KOMAClassName}{% if non-KOMA class
  \IfFileExists{parskip.sty}{%
    \usepackage{parskip}
  }{% else
    \setlength{\parindent}{0pt}
    \setlength{\parskip}{6pt plus 2pt minus 1pt}}
}{% if KOMA class
  \KOMAoptions{parskip=half}}
\makeatother
\usepackage{xcolor}
\IfFileExists{xurl.sty}{\usepackage{xurl}}{} % add URL line breaks if available
\IfFileExists{bookmark.sty}{\usepackage{bookmark}}{\usepackage{hyperref}}
\hypersetup{
  hidelinks,
  pdfcreator={LaTeX via pandoc}}
\urlstyle{same} % disable monospaced font for URLs
\setlength{\emergencystretch}{3em} % prevent overfull lines
\providecommand{\tightlist}{%
  \setlength{\itemsep}{0pt}\setlength{\parskip}{0pt}}
\setcounter{secnumdepth}{-\maxdimen} % remove section numbering

\author{}
\date{}

\usepackage{fontspec}
  \setmainfont[Ligatures=TeX]{Charter}
  \setsansfont{Lucida Grande}
  \setmonofont{American Typewriter}
  \newfontfamily\chapfont{Charter Black}
  \newfontfamily\secfont{Charter Black}
  \newfontfamily\handfont{Bradley Hand}

\setlength{\paperwidth}{6.125in}
\setlength{\paperheight}{9.25in}
\pdfpagewidth=\paperwidth
\pdfpageheight=\paperheight

\setlength{\textwidth}{4in}
\setlength{\textheight}{7in}
\setlength{\oddsidemargin}{0in}
\setlength{\evensidemargin}{0.125in}
\setlength{\topmargin}{-0.25in}


\renewcommand{\baselinestretch}{1.1}
\renewcommand{\arraystretch}{1.2}
\setlength{\tabcolsep}{2pt}

\raggedbottom

\usepackage{fancyhdr}
\pagestyle{fancy}
\fancyhf{} % clear all header and footer fields
\renewcommand{\headrulewidth}{0pt} % remove the header rule
\fancyfoot[C]{\small\thepage} % small page number in the center of the footer
\setlength{\footskip}{36pt}

\usepackage{tikz}
\usepackage{xcolor}
\definecolor{avocadogreen}{rgb}{0.34, 0.51, 0.01} % An example of avocado green
\definecolor{vintageorange}{rgb}{0.95, 0.65, 0.0} % An example of vintage orange
\definecolor{darkavocadogreen}{rgb}{0.20, 0.30, 0.01} % A darker shade of avocado green
\definecolor{darkvintageorange}{rgb}{0.70, 0.40, 0.0} % A darker shade of vintage orange




% TITLE
% ======================================================================
\begin{document}
\pagestyle{empty}
\cleardoublepage
\mbox{ }
\cleardoublepage
\frontmatter
\begin{center}
  \vspace*{48pt}
  \hfill {\fontsize{86}{52}\sffamily\bfseries stories}
  \\[12pt]
  \hrule height 1pt
  \hfill {\fontsize{24}{52}\sffamily carol carpenter}
  \vfill
  \hfill {\sffamily colloquial publishing}
  \\
  \hfill {\sffamily new york}
\end{center}
\color{black} % Sets the text color to white


\clearpage
\pagecolor{white}
\mbox{}
\\[36pt]
Carol Carpenter. 1943--2016.  Stories.  Colloquial Books.  New
York. 
\\[12pt]
\begin{center}
Copyright \copyright\, Carol Carpenter
\\[4pt]
All rights reserved.
\end{center}
\vfill
\begin{quote}
\emph{This is a work of fiction. Any resemblance to actual events or
persons, living or dead, is entirely coincidental.}
\end{quote}
\vfill
\vfill
\null

\cleardoublepage
\pagestyle{fancy}
\tableofcontents


\mainmatter
\cleardoublepage
\part{Fiction}
\thispagestyle{empty}

\cleardoublepage
\chapter{Big in the Bars}

One by one, the shoppers are attracted by the flashing blue light above
the table. I slice the potatoes into thin wafers, fanning them like a
deck of cards. ``The magic little machine that slices, dices, and
shreds.''

I slip on the round gadget, put in a small head of cabbage, and push the
button. ``Instant coleslaw.'' My thumbs point toward my chest. ``Proof
positive. Even a man can whip up a dish.'' Women laugh.

To a young man wearing a shiny gold wedding band: ``Want more of the
bride's time? Get her out of the kitchen with this.'' Crowd snickers and
presses forward, trying to get a look at the man. Hold up the shiny
machine. Make my pitch and sell four.

At five o'clock the crowd fills the aisle, spills over into hardware. I
talk faster as brown paper bags rustle and ``Jingle Bells'' plays over
the loudspeaker. I pass sliced apples in a clear plastic bowl.

When I see Johnnie in the crowd, chewing on an apple slice, my fingers
move, my words grow. I play to him. He asks questions, pretends doubt.
He plays my shill, buys one. I sell eight more Veg-O-Matics and turn off
the blue light.

``You never lose the touch,'' Johnnie says, helping me pack away the
equipment.

``This crowd's easy,'' I say, remembering the times I'd called Johnnie
to help me with the tough ones. The cynics, lunatics, born-agains who
tried to wrench the crowd from my fingers.

``Remember that county fair in Billows? They'd leave that revival
meeting full of hellfire.''

Johnnie laughs. ``Once we had the good Reverend hooked, they followed
like black sheep.'' He looks past me, past the shelves of cookie tins,
cake pans, and Pyrex cookware. ``Twenty-six years. What did we do with
all the time?''

``A lot, Johnnie. We did a lot.''

What can I say to him? That we ate two pounds of feta cheese when we
heard it would make us men at thirteen? That we beat up Ronald
Smalderhort for laughing at his mother's accent? That I still have the
Greek cross he gave me when we were ten? That I hate him for having had
it all and lost it?

Johnnie lays a sheaf of papers on top of the carrots I'll use tomorrow.
``The papers from the bank. Even a second mortgage and borrowing to my
limit isn't enough to buy that bar in Westhill.''

I hand him back the papers without reading them. ``So? There are other
ways.''

``Name one,'' Johnnie says as he straightens the tilting Christmas tree
by the coffee pots.

``Ask Santa. Rob a liquor store. Get a rich mistress. That's three for
you to think about. Let's eat.'' I hand him his grease-stained parka.
``Doesn't your wife ever wash this thing?''

We walk past the animated elves past the men's underwear, out into the
mall. Johnnie whistles ``Silent Night'' along with the Salvation Army
singers. I drop a dollar into their black kettle.

``Remember Sary Mae that Christmas in Arkansas? The one whose daddy
owned those weight loss clinics,'' says Johnnie.

``Not the one who put the cold towels on our heads the morning after? Or
was she the one who turned out to be a guy? You always were big in the
bars.''

At the Coney Island, Johnnie orders his hot dog without chili and grunts
when I get the works on two, French fries, and a Coke.

``That stuff will rot your guts,'' Johnnie says. He stands up and
tightens his stomach muscles. ``Go on, punch me.'' I give him a light
tap. ``Harder.'' I hit him harder and the waitress rushes over. ``It's
okay,'' he says without looking at her and sits down.

He takes a bite of his hot dog, chews it seventy times (his mother used
to count), and says, ``You should start working out with me.'' After he
finishes eating, he washes down his all-purpose multi­ple vitamin and
tablets of B, C, and E. He offers the bottles to me, but I tell him
Greeks put too much faith in food and vitamins.

Johnnie says his mother isn't dying after all. It's only the gall
bladder that has to come out; then she's coming to live with him. Stella
and kids aren't happy about it, but he knows how to handle her, how to
handle all of them.

I tell him to borrow some money for the bar from his mother, make her a
partner. Says that's not the Greek way. When I ask him what the Greek
way is, he shrugs his shoulders. I suspect he doesn't know. Once he said
he couldn't remember anything about Greece and must have been adopted.
He plans to go back someday, but I know he's just talking.

Wants to know if I'll help him fix Little Lite Mike's car tomorrow after
he bails J.B. out of jail.

``Christ, Johnnie. No wonder you never sleep or have any money. Do these
bums ever pay you back?'' I can tell by his droopy eyelids that he's not
listening. I poke his shoulder and he looks at me.

Johnnie says, ``Well? Can I count on you tomorrow?''

``See you tonight,'' and I pay our bill, leaving him to drink his black
coffee alone.

When I walk in the Pour Devil, Johnnie's on stage finishing my favorite,
``\ldots and gave birth to the blues.'' The sound holds, dark and thick
like a blackberry brandy. He sees me and gives the thumbs-up sign.

I wave to the old lady in orange chiffon, wondering if her puckered
chest ever gets cold as she sits listening on the same bar stool night
after night. All the regulars are here: Gracie wonders who'll buy her
breakfast; Roy looks for his ex-wife; LuAnne dances to forget she's no
one's ex.

Tonight they make me tired. All of them. Even Johnnie. I lean against
the carpeted back wall and watch him check the mike, move a loudspeaker.
Like a depressed housewife, he adjusts, readjusts the furniture.

During Johnnie's second set, the man next to me says, ``I used to watch
that guy twenty years ago at The Cat's Eye.''

``Oh yeah?'' I stare at a broken light fixture.

``Yeah, that's a fact. Knew him before he ever made it on `Saturday
Dance Party,' before he had a hit record, before he fell off the
charts.''

I imagine the man is a fly and drown him in my beer.

``Say, whatever happened to him?''

``Nothing,'' I say as I revive the fly and swat it against the Merry
Xmas sign.

``Bet it was either dames or whiskey. What do you think?'' The man says.

``I think it was flies. They done him in,'' I whisper.

The man takes his Scotch and soda and moves to the bar. I push open the
Gents door, push it right into Nelson's back.

``Can't a guy even comb his hair in this joint without being knocked
around?'' Nelson stoops so he can see his hair in the lower half of the
cracked mirror. He sees me behind him. ``Hey, the man I've been wanting
to talk to. Do you think Johnnie'd lend me a hundred? Big poker game
tomorrow. I could take all they've got. Watch it,'' Nelson says as I
start to step into the stall. ``Toilet's leaking again.''

The jesters are present with their pom-pom hats for Johnnie. I feel them
tugging at my scalp.

``Sure, he'll give you the money. Didn't you see the sign outside?
Drive-in banking, twenty-four hours a day.'' Everyone has a pitch these
days.

``He gets most of my paycheck every week. Least he can do is give some
back. Wish I owned this place.'' Nelson shakes his wet hands. ``You'd
think he'd at least spring for some paper towels. What a cheap
bastard.''

``Yeah,'' I say thinking of Johnnie's overweight wife who works at Ford,
his rusted-out Mustang parked in back. One hundred and fifteen thousand
miles on it, a piece of cardboard stuck over the side window, a broken
defroster.

Yet, Johnnie has to play the big shot. Some of our biggest arguments
have been over that. I say he's afraid to tell people the truth. What
truth, he says. That you don't own this place, I say. What place, he
asks. Forget it, I say.

Johnnie is with the Scotch-and-soda man at the end of the bar. I try to
warn him but he's already talking, tucking his black shirt tight under
the black leather belt, hand-tooled in Greece, special ordered by his
mother who went back for a visit.

``I'm all right,'' Johnnie says automatically as he grabs my elbow and
pulls me into the kitchen. He doesn't remember that I stopped asking
that question years ago. ``You know any rich mistresses?'' He says.

``You've got to get a class place of your own. You're not getting any
younger, you know.'' I cut myself a piece of the spinach pie Johnnie's
wife made. ``That woman sure can cook.''

``I'll show you who's old,'' and we spar until he gets me in a headlock.
The spinach pie sits uneaten on the counter.

``Give, give. A drink for my life.''

Johnnie lets go and we fall to the floor. ``Thelma,'' Johnnie bellows,
``A beer and a shot.''

``You two are crazy,'' she says, setting the drinks on the floor beside
us. ``Must be a full moon out tonight,'' and she bares her teeth, bites
at Johnnie's neck. ``You better get out there and keep those animals
dancing.''

I sit on his stomach to hold him. ``Thought you were on the wagon.''

Nelson comes through the swinging doors, sees me, and turns to go.

``Wait,'' Johnnie says pushing me off. ``You start tomorrow,'' and he
hands Nelson a crumpled sheet of paper. ``The pay's good.'' He takes
money from his pocket and straightens it between two fin­gers before
handing it to Nelson. ``To hold you 'til payday. Don't mess up this
time,'' he tells him.

Nelson thanks him, says he owes him a lot, says he'll pay him back. Says
he can't pay him back for all he's done. He leaves by the back door.

``Christ, Johnnie. You dangle your feet in the water and the
bloodsuckers feed.\\
Bells jangle, peaked caps bend. Lions move on padded feet. As Johnnie
sings, the dancers' pounding moves along the edges of the oak dance
floor, under the carpet, through coated wires.

I watch Johnnie control the group; move them closer together, pull them
apart, urge them into twirls and dips.

It's not my Veg-O-Matic crowd, but I can sell them a dream they can
slice and dice. This morning was my dress rehearsal. Johnnie never did
know how to peddle himself.

When Johnnie steps off the stage, I step on. I turn on the Christmas
lights rising the backdrop, pick up the mike. People look at me and
point. Most of them know me but only off stage. No one uses Johnnie's
stage but Johnnie. Nelson leaves to get him.

The bells are quiet, jesters sit in chairs, the lion's cage\\
is open.

I pick up the tambourine and shake it above my head, against my hip.
``Tonight's the night,'' I say quietly, forcing them to lean forward in
their chairs. Louder. ``The night to pay old debts. The night of
truths.''

I shake the tambourine and point to Johnnie who's motioning me off
stage. ``Who doesn't laugh, dance, and sing with Johnnie. Who doesn't
tell him problems?'' I set the beat. ``Problems, we all got problems.''
The crowd's clapping with me. ``I got problems. You got problems. Even
Johnnie's got problems. Yeah, yeah,'' I chant.

I can feel the crowd with me, hear their echo. I make my pitch,
``Johnnie don't own this bar.'' The silence after revelation. Then
louder than ever. ``But we can buy stock in Johnnie's dream. Help
Johnnie buy his bar. Buy a dream.''

I pull silver and bills from my pocket and drop them into the
tambourine, shaking it. ``Dig deep for Johnnie,'' I say, passing it to
the silky girl beneath me. ``For Johnnie's bar. For Johnnie's dream,'' I
continue as the tambourine is passed from hand to hand, Thelma empties
it twice into her tip jar.

``For Johnnie,'' one final time as someone turns on the juke box and
everyone dances.

I find Johnnie in the back room stretched out on the mattress he sleeps
on when he's too tired to go home.

``I'm not a charity case,'' Johnnie says.

``No one says you are, you blue-eyed Greek. But you're big in the
bars.''

\cleardoublepage
\chapter{The Bottom Drawer}

A voice over the kitchen phone. His voice. I've been expecting it.
``Tonight,'' he whispers as if he's talking through a handkerchief.
``Just the two of us.''

``You shouldn't call me here,'' I pour milk in my coffee, slam the
refrigerator door shut with my foot.

``Why not?'' he says. ``No one's there.''

``True,'' I answer, looking around the empty kitchen, clean for a change
with the kids gone for the weekend on a Boy Scout campout, my husband
with them. But I have the feeling they could walk in any minute.

``How's you car?'' I ask, remembering he took the bus to work this
morning.

``Nothing \$300 won't fix,'' he laughs.

``I'll loan it to you,'' I offer.

``I don't want your money,'' he says.

Doesn't want his own either, gave it all to his ex-wife last month in
the divorce settlement.

``What do you want?''

``You,'' he says. ``Wrote you another poem, `Laughing Laura at Her
Desk.'\,''

No one ever wrote me poems before. I keep them locked in my bottom desk
drawer at work.

``You there?'' he asks.

``Yes, I'm here,'' I lie.

``You're not angry I called are you?'' he asks. When I don't answer, he
says quickly, ``Let's meet at the coffee shop down by the park. I need
your advice on something.''

Like a couple of high school kids meeting at the local hangout. Never
did that, though. Not in high school. Never with him.

``Advice on what?'' I ask, remembering his oldest son hadn't invited him
to his high school graduation last week.

``On us,'' he says.

``We're friends,'' I say, seeing him again in last night's dream,
leaning over me, stroking my forehead until the lines in my flesh
disappear and I'm sixteen again. Only this time I go to the dance.

``More than just friends,'' he says.

I wonder if he's dating the redhead who lives down the hall from him.

``Ever since that infamous lunch,'' I say, pulling the phone into the
living room. ``All because you forgot yours and ate half of my bologna
sandwich. Should have held out for the traditional apple.''

The pictures of my husband and children stare at me, smiling from their
frames on the piano.

``Aha,'' he laughs. You brought me an apple the next day.''

And so I had. But his laughter doesn't belong in this room. My husband's
eyes watch me, prac­tical as always. Serious. His mouth closed against
need, against silliness.

``Are you still there?'' he asks.

``No,'' I tell him, expecting to look up and see my appointment calendar
opened to the right place.

``I'll see you at lunch Monday,'' I tell him, picturing the tiny
storeroom where we eat and he reads his poems.

I hang up without saying goodbye. I know people at work are beginning to
talk. But that's at work. That's not here.


\cleardoublepage
\chapter{The Butterfly Dress}

Instead of my Saturday morning paper, I find Aunt Hattie's shoe, like a
warning right outside my door. A pointed-toe purple shoe with
rhinestones in the heel, the kind that women wore in the 1960s. It's so
like her to show up at my doorstep, actually on my apartment landing,
just when I've got my life on the right track, just when I'm about to
become the engine blowing steam instead of a caboose or box car carrying
someone else's load.

``Harry,'' I shout, spotting her at the end of the hall sitting on the
radiator reading the Saturday morning comics and chuckling to herself. I
call her Harry when I'm irritated, ever since five years ago when she
cashed my paycheck, boarded the wrong bus, ended up at the race track
and lost it all on some horse called Serendipity. If I didn't believe in
Serendipity, she said, I had a lot to learn.

I call again and remind myself to have maintenance fix the burned-out
light bulbs in the brass sconce across the hall. I roll the sounds
around on my tongue, ``Harry,'' knowing she doesn't like to be called by
a man's name, especially the same name as her fourth husband, the
astrologer. She told me once that she married him on the strength of his
astrological chart and left him a year later for the same reason. Still,
she said, there's something to love in a man who can read the stars.

She hobbles toward me on one shoe, dropping newspaper pages like mulch,
her Hawaiian print dress bursting into red and pink hibiscus as she gets
closer.

``This slush almost ruined my best shoes. Stepped out of the cab right
into a puddle,'' she says and bends down to pick up her abandoned shoe.
``But look at that, dried good as new. Can't say I miss these Detroit
winters.''

Aunt Hattie winks at me as I yank her into the apartment. ``Got any
bacon and eggs? That study they serve on the plane isn't good for your
digestive system.'' She peeks into the vestibule mirror, fluffs her
pink-tinted hair, all the time peering myopically at my reflection over
her shoulder.

When Aunt Hattie starts poking around in things, she stirs up a lot of
dust. ``You didn't say you were coming. Where's your luggage? Where's
your coat?'' I ask, figuring attack is better than standing by doing
nothing. If it gets really bad, I can put her back on a plane. They
leave for Florida day and night. ``Now I suppose we've got to go back to
the airport to collect your things.''

``Nope. I came unencumbered,'' she says, already in the kitchen cracking
eggs into a bowl. ``Got to thinking about that red and purple sequined
dress I bought years ago at downtown Hudson's. Your mother said it was
scandalous. Cut to here.'' She sucks in her stomach. ``Big butterfly
across here,'' and she draws her fingers across her bosom, leaving
traces of egg shell. ``Need something like that for the Misty Mellow
Seniors Ball.''

``What's that got to do with anything?'' Sometimes she rambles, while I
like to get to the point. ``Besides, downtown Hudson's is gone. Don't
tell me you don't have stores in Miami.''

``Not the kind that sells butterfly dresses. At least not around my
place,'' she says as she flips the bacon, splashing grease that spurts
into brief flame on the burner.

I tap my fingers against the oak table, a habit I've been trying to
break since my boss men­tioned it during my performance review last
year. Instead, I scrape off some raspberry jam that has dried into a
small jewel atop the wood. All the time, I try to think of where we can
find a butterfly dress. I haven't been shopping for clothes since I
found the black business suit on sale at Lord \& Taylor's at Fairlane
Center Mall a few months back. Who knows if they carry butterfly
dresses.

``Nice place you have here.'' Aunt Hattie keeps her back turned. ``Your
mother told me you got a promotion a while back to supervisor of
something or other and moved into this new place downtown.''

For now, I refuse to close the space between us. I stare at her back,
straight as always, and her shoulders that curve toward the pan, and the
top of her head that dips toward the bacon as if she's listening to its
sizzle and pop.

Never one to give up, Aunt Hattie adds, ``Your mother says you work too
hard, very long hours. But she says the money's good and you're moving
up.'' She glances over her shoulder at me. ``You know, all work and no
play makes for a dull day.''

``Easy for you to say.'' I grin. ``Times have changed.'' I glance at the
kitchen clock and wonder how I'm going to get the work done that I
planned on doing this weekend.

``Times always change,'' Aunt Hattie shoots back. ``That's no excuse.''
She pops two slices of bread into the toaster. ``Ah, well, this white
bread will have to do but the preservatives will clog up your system.
You really should take better care of yourself.'' She moves the toaster
setting from dark to light. ``Your mother said you can see the Detroit
River from your bedroom window. Can you really?''

``Not when it's dark and foggy like today.'' I know Aunt Hattie has
something up her sleeve. ``Come on, Aunt Hattie, pull out your
handkerchief,'' I tease her with the old joke between us that lets her
know that I'm on to her. ``Bet you have some good-looking man tucked
under there and you're waiting for the right time to whisk him out.
Right?''

``Could be,'' Aunt Hattie says. ``But forget the looks. They don't count
in the end. I would find you a man who could tickle your funny bone.''

``I laugh enough,'' I assure her. I search my mind and actually can't
remember the last time I was with anyone besides Aunt Hattie who had me
gasping for breath between laughs and wiping tears from my cheek, the
way Aunt Hattie and I used to laugh at \emph{The Three Stooges} reruns.

``This man might even live in the neighborhood.'' She pauses as if
searching some map in her mind. ``But he sure wouldn't laugh at what's
happening to this city. It's a crying shame so many of the buildings are
crumbling and vacant around here. Didn't used to be like that,'' Aunt
Hattie says. ``Somehow you expect brick and mortar to last longer.''

She turns to look at me, as she pats the bacon dry between two paper
towels. Then she divides it equally onto our plates, four slices each.
``Still, I'm glad to see someone's building places like yours, someone's
trying to breathe new life into this city.''

Over breakfast she wants to know how my night classes are going, why I
need to get a master's degree, do I ever stay after to talk to my
professors, if they're all married. I tell her it's none of her
business, but she won't let up. Aunt Hattie complains that I think too
much. ``Comes from reading all those books,'' she says. ``Sometimes you
just have to feel something and do it.''

Right now I feel like going back to bed. After studying half the night
for my computer program­ming class, I'm not very alert. It dawns on me
that I better call my mother before she arrives on my doorstep in search
of Aunt Hattie.

Sure enough, my mother answers the phone with, ``Hattie, where are
you?'' When she hears my voice, she starts right in telling me how
Hattie is missing and how she and Charles have searched everywhere.
``Even went to that martial arts center where she's been taking tae kwon
do. Did I tell you she's doing that? Says she wants to cultivate her
chi. Can you believe it? Says it's something to do with her internal
energy.'' My mother sighs and I picture her rolling her eyes like she
always does when she talks about Aunt Hattie. ``That woman will be the
death of me.''

When I explain that Aunt Hattie is sitting at my kitchen table eating
breakfast, my mother sighs again. ``I didn't want to tell you but the
doctor says her mind may be slipping.'' I hear her air condi­tioner
humming in the background. ``I'm sorry, Dear. Just put her on the plane
and I'll meet her at this end.''

I promise to keep an eye on my aunt and send her back Monday morning. My
mother warns me to make sure she gets on the plane. ``Can't trust her to
do what she says. Not if something better comes along. ``Never could,''
Mother says.

I don't notice any difference in Aunt Hattie. She always seems the same
to me. Still, I feel sad, the same as when the wrecking crew tore down
the old hotel behind me. Dust was heavy in the air for days afterward.

Even after my mother hangs up, I hold the phone pressed against my ear,
not quite ready to break the connection. I remember how when I was a
little kid, maybe ten or eleven, and stayed with Aunt Hattie, she
treated me like a grown-up. She took me to midnight movies and dealt me
in when she played canasta with her neighbors. She even gave me a roll
of nickels to bet with and sometimes I carried my winnings in my coat
pocket for a week, just to hear them clinking against each other.

Now Aunt Hattie sits at a kitchen table like she did back then, piling a
forkful of scrambled eggs on her toast before taking a bite. I ask her
about the ball she's going to, wondering if she's making it up. With
her, it's often hard to tell what's real and what she just wants to
happen.

``Your mother's going with Charles Bronsbottom,'' she says, watching to
see if I recognize the name. ``You really ought to talk to her. She
won't listen to me. First thing every morning he grabs the newspaper to
check his stocks. Tells everyone how he invested in IBM back before
computers.''

``Do I detect some jealousy here?'' I tease.

``Of course not,'' she says, buttering another piece of toast. ``But
he's not right for her. You can tell a lot about someone by how he reads
the newspaper.''

I suppose she's right since I usually turn to the advice columns,
reconfirming that men are more trouble than they're worth.

``You know he insists on being called Charles. Not Charlie or Chuck.
Charles,'' she says, draw­ing out the syllables. ``Nothing like your
father, rest his soul. At least your father knew how to have a bit of
fun.''

Aunt Hattie licks her fork, taps it on the table. ``Reminds me of that
James you were engaged to a few years back.'' She taps louder, blinking
her eyes as if she's changing channels by remote con­trol. ``Humph,
James. Sounds like somebody's butler. Now, Brad, Rick, Mike---those are
names you can sink your teeth into and hold on. Even Johnny, a name that
wraps `round you like a quilt. Know any guys with those names?''

``No,'' I say, scrubbing harder to get the scorch marks off the frying
pan. ``Harry,'' I warn her, ``life's too short to mess it up with men.
Of all people, you ought to know that.'' Aunt Hattie's been divorced
five times, engaged more times than the rings on a red cedar, and always
on the lookout for what she calls good marriage material.

``You remind me of your mother sometimes,'' Aunt Hattie says, shaking
her head. ``Never mind. Did I tell you my new beau's name? It's Les. I
tell him Les is more.'' She chuckles and cups her face in her hands.
``So we're working on a new name for him. We've tried out Major, Grant
and Rock but they don't fit right. A name should mark a man, tell you
what he is,'' she muses, almost as if she's forgotten I'm here.

I stare at the wall behind the sink, wishing now that it didn't remind
me of mashed potatoes. From tiny chips of paint samples arranged on a
card, I remember struggling for hours to finally pick that shade, desert
sand, the same color as my office. Neutral and clean. Makes it easy to
concentrate on the computer programs I usually spread across my desk at
work or my kitchen table in my apart­ment. I wonder if I ought to hang
some posters or baskets or maybe even wallpaper to make it look more
homey. Maybe I should repaint it in a color like cranberry or grape or
even lettuce green, some­thing nourishing.

``Yes, maybe that's it. We can call him Mark. He's my sign, the
difference in my days, some­one important. Mark my word,'' she says
laughing, ``when you meet him, he will sweep you off your feet. He knows
every kind of dance and even makes up his own. I always did like a man
who knows how to dance.''

``Aunt Hattie, you \emph{are} crazy.'' I laugh, a deep freeing laugh
that bounces off my abandoned plans to work on my computer program all
weekend. ``You make me forget things,'' I scold her. ``How will I ever
get ahead at work if I spend my time hunting down butterfly dresses with
you?''

``You're too young to forget things,'' she says. ``Before you can forget
them, you have to do them.''

``So this Mark of yours is someone new,'' I say. ``What happened to the
one I met last Christ­mas? The one who was going to take you to Paris?''

``Oh, that one fell and broke his hip. Even after it healed, he was
afraid he would fall and break it again. Sad,'' she says, ``the way some
people just give up.'' She clucks her tongue against the roof of her
mouth, a talent I tried to copy when I was a kid and never could master.

When I try to make Aunt Hattie's sound again after all these years, she
sticks out her tongue and waggles it at me before tucking it back behind
her front teeth. ``Like this,'' she says, slurring the words. She keeps
up the clucking as if she could go on this way forever.

Again I try and again I sound like I've got something stuck in my throat
that I'm choking on. I gasp for breath.

``Can you whistle yet?'' Aunt Hattie asks, sticking two fingers in her
mouth and letting out a high-pitched screech that probably can heard all
the way to the fifteenth floor.

``No,'' I admit. ``I can't whistle blowing out, just sucking in.'' I
show her the puny, cracking, shrillness I can make if I purse my lips
and inhale.

``You just need more practice,'' Aunt Hattie reassures me, brushing my
uncombed hair off my forehead with my fingertips. She reaches down in
her black leather purse, worn along the edges, and pulls out a brush.
She moves behind me and starts to brush my hair, like she used to do
when I was a little girl.

``Beautiful color,'' she says, ``like the cherry wood your grandfather
used to turn into bureaus and chairs with carved backs.'' She brushes
for a while longer, then smooths her hand over my short cap of hair.
``Too short to braid now,'' she sighs.

Sometimes I envy Aunt Hattie, the way she does what she wants no matter
what, the way she wanders through life like a tourist bent on seeing all
the sights. I wonder if I got cheated out of some chromosome that would
have made me more like her, sort of like a computer program with
reversed code numbers.

``Remember how, when you were a little girl, we went to Eastern Market
every Saturday?'' Aunt Hattie asks. ``You used to press so close to the
fish market window, your breath fogged up the glass. You kept wiping it
with tissues. Everything got so smeary you couldn't see.''

``That was a long time ago,'' I say, trying to remember myself as a
child.

``Not so long,'' she whispers, carrying her dirty dishes to the sink and
dropping them in the water. Grease beads on the surface. ``You wanted to
take those fish in the window and put them back in the lake. Never had
the heart to tell you they were dead.'' She dumps the dishwater, turns
on the tap, squirts in so much liquid that soap bubbles erupt from over
the side of the dishpan and cling to her arms.

``Very little in life that's as final as death,'' Aunt Hattie adds,
turning to me and flicking soap bubbles off her arm, aiming them toward
me. ``Some people just sit and wait around for it.''

``Harry,'' I warn, ``Don't talk like that.'' She clicks her tongue and
flicks more soap bubbles my way.

``Knowing you,'' I say, ``you'll be the one to decide when your times
up. You'll probably climb some ladder until you reach the top and then
step off, wearing nothing but your purple shoes with rhinestones.'' I
picture this and start to giggle. ``Well,'' I add, ``you might wear your
butterfly dress.''

``Only if Gentleman Death comes courting in the proper manner,'' she
says primly. ``And he needs to know the dance of the butterfly dress. A
very special dance, indeed.''

Aunt Hattie measures me with her eyes, from my toes to the top of my
head and winks. ``You are very special. Even back then at the Eastern
Market, you drew people to you. No other little girls got to help the
baker wrap the bread. While you two jabbered away, I slipped next door
and bought the trout you loved so much,'' she says. ``Once they were
coated in batter and cooked, you never knew they came from behind that
window. Sometimes I wonder if I should have told you.''

I hug her, feeling my body fit itself to her curves, kissing her cheek
that tastes like peppermint soap. She grabs my hand, dripping suds down
the sleeve of my white chenille robe. ``Don't ever forget who taught you
to waltz,'' she says, placing her other hand in the small of my back and
counting off the steps.

``How could I forget? And now, it's time,'' I announce. ``Let's go find
the monarch of butterfly dresses.''

While I'm getting dressed, I hear Aunt Hattie moving around my living
room. I find her study­ing the stack of computer printouts I was working
on last night.

``What do you do with this stuff? Looks like hieroglyphics.'' She lifts
up the top page. The pile unfolds like an accordion.

``I debug programs.'' She laughs, a staccato sound crisp as high heels
against tile. ``I fix the errors so the program can run.''

``Seems, then, like you could debug your life,'' she says, smoothing her
nylons, lamplight glinting off her rhinestone heels. ``No offense, but
even when you were a little girl, you were afraid of lightning bugs.
Always wanted me to catch them and put them in Mason jars. Never could
explain how you just can't put a lid on some things.''

I link arms with her and shut the door behind us. There's a boutique I
know that may carry what she's looking for.

When we arrive, Aunt Hattie refuses to try on the black dresses the
saleswoman brings. ``I'm not going to a funeral,'' she snaps as she owls
sequined dresses from the rack and holds them up to her. When she finds
the one with the red rose, its stem rising from the waist, she smooths
it over her bulges.

In the dressing room, she holds her hands at her sides as I zip up the
back. ``Goes with my shoes,'' she says, fingering the purple satin
skirt. ``Look how the butterfly on the petal glitters gold in the
light,'' she says. ``Gives me wings. Yes, it does.''\\
She steps out of the dress and whisks it in front of me. ``You try it
on,'' she says, shivering slightly in her purple slip. She pats her
arms, hands fluttering against her skin to warm herself. ``Go ahead. Get
the feel of it. Won't kill you.''\\
As I lift my arms to slip it over my head, I lose my balance and tip
sideways into the mirror. ``These dressing rooms are too small,'' I
grumble. ``No room for moving around.''

``Not over your clothes, silly,'' she prods. ``You need to feel it rub
against your skin.'' She flashes my grimace back at me.

Aunt Hattie picks up my blouse from the floor and drapes it over her
arm. ``Think of this as a costume.'' She stands behind me, hands on my
shoulders, and sighs. ``Feel the way that dress lifts you up?'' She
raises the silky material of the skirt and lets it drop again in folds
against my legs.

I twirl for Aunt Hattie, hands on my waist. ``We always were about the
same size,'' I say, still amazed that what fits her fits me.

``I owe you,'' Aunt Hattie says. ``You found me the dress. Now I need to
return the favor.''

``No favors,'' I tell her. ``I've had your kind of favors before. There
was that time you offered to cash my check and I never saw any of the
money. Or what about when you called in sick for me during fourth grade
and told the school secretary I had eloped with Rudolph Valentino, as if
she knew who that was. No,'' I repeat, ``I definitely do not want any
favors.''

``I'll think of something,'' Aunt Hattie says, as the saleswoman
packages the dress in tissue and folds it into a green cardboard box
with a carrying handle.

Aunt Hattie points the box toward a man dressed in a double-breasted,
charcoal gray suit. ``Now that's a nice looking man,'' she says as we
step behind him onto the escalator. ``But did you notice how his step
has no bounce? And those worry lines on his forehead are not a good
sign.''

``Shh,'' I warn her, hoping the man won't turn around. ``He's probably
got some problem he's figuring out. James always looked like that when
something at work bothered him.''

``See,'' she says. ``I rest my case. Definitely not your type of man,''
she says. ``And don't roll your eyes like your mother.''

As I unlock my silver Mustang, a present I bought myself when I was
promoted, Aunt Hattie pats the black convertible top. ``Wish we could
put the top down. Nothing like the wind in your hair,'' she says as she
runs her fingers through her pink curls. ``I still drive my old
Thunderbird. Gets me where I want to go and that's what counts.''

I don't confess to Aunt Hattie that I bought the Mustang on impulse this
past May and haven't had the urge to put the top down since. She would
insist that we roll the top down today, no matter what the temperature.

She settles against the chilled leather and pulls the seatbelt tight. I
wait for the defroster to start working before I zip out from the curb,
merging with the other traffic. We sing along with some of the songs on
the radio, her voice deeper, richer than mine.

Aunt Hattie says it's her treat when we stop by Carl's Chop House for a
late lunch. We're seated before I remember I left my purse locked in the
trunk. I'm irritated I have to go back into the cold to retrieve it.
When I return, I spot Aunt Harriet sitting at a different table,
laughing with a man about my age and holding up her new dress. She
doesn't see me moving in her direction.

She jumps as I put my hand on her shoulder. ``Oh, there you are.
Johnny's going to join us for lunch.'' She looks back and forth between
us, her eyes darting about like lightning bugs. ``Really, John is his
middle name but I like it so much better than Matthew.'' She folds her
dress carefully before tucking it back in the box and setting it on the
floor by her chair.

``Harry,'' I warn, glaring at her and refusing to look at the man, who
must think we're both crazy. ``It's time for us to go. That plane leaves
early tomorrow for Florida.''\\
``Sit down and rest awhile. Plenty of time. Go on, sit,'' she says,
motioning me toward the empty chair the man has pulled out for me.

``I'm sorry,'' I say to the man across from Aunt Hattie. ``It's just
that she's\ldots''

``A total delight,'' he finishes for me. ``Please do sit down,'' he
says, sweeping his arm toward himself as if he's scooping up air. ``Just
call me Johnny.''

As I run my tongue over my lips, I feel the rough skin, chapped from the
cold, and wish I had stopped to put on lipstick. My chest feels tight,
as if I might be coming down with something.

There's only one other couple sitting at a table against the wall,
leaning toward each other, their blonde heads merging over steaks that
may grow cold on their plates. The tablecloth glistens white as egg
shells or new-fallen snow.

``You two have a lot in common,'' Aunt Hattie says. ``You young people
today work too hard. You're so busy watching your feet move that you
step all over yourselves,'' and she clicks her tongue.

``How do you do that?'' Johnny asks, trying to imitate Aunt Hattie's
tongue clicking. ``My uncle used to make the same sound whenever he told
me to mind my manners.''

``And did you?'' Aunt Hattie asks.

``Never. No fun in doing what's expected,'' Johnny says, looking right
at me. ``Can't learn any­thing that way.'' He studies the wine list,
running his finger down the names, reading some aloud. ``Let's try this
Zinfandel from California,'' he tells the waiter who agrees it's a fine
choice for brighten­ing up a drab winter day in Detroit.

``So tell me why you work so hard,'' Johnny says, his eyes dark as the
convertible top on my Mustang. ``What is it you do?''

For some reason, I can't remember what I do. It's as if my mind's hard
drive has crashed and I lost all my files. ``I don't work hard,'' I deny
as he reaches across the table and turns my hand palm up trace the lines
with his fingertips.

``Cold from being outside. But no callouses yet,'' he tells Aunt Hattie.
``It's not too late. In her future, I see hope, a long lifeline, much
happiness, an aunt who gives good advice.''

As I start to withdraw my hand, he tightens his grip. Blood rushes to my
cheeks, a sudden warmth. Out of the corner of my eye, I see Aunt Hattie,
the stillest I've ever seen her and she looks tired, her face tanned,
lined and weathered, older than she looked this morning, I know I can't
let her down.

``How much happiness do you see?'' I joke. ``Enough to fill a wine
glass?'' I click my tongue against the roof of my mouth and Aunt
Hattie's sound emerges from my mouth. Both of them stare at me as if
I've conjured up ghosts. I do it again. ``Easy to do. Just takes some
practice,'' I gloat feeling his hand cup mine. ``Maybe I'll teach you
how.''

Aunt Hattie pats Johnny's arm and says, ``I know someone who has a
terrific dress she can bor­row if you like\\
to dance.''

``A man who can't dance sits on the sidelines,'' Johnny says sipping the
fragrant wine. He winks at me, the same kind of wink Aunt Hattie gives
when she's up to no good, the kind of wink that reminds me of a railroad
crossing.

I wink back, hearing in the distance the sound of train whistles and the
clickety-clack of big steel wheels against\\
the tracks.


\cleardoublepage
\chapter{Case Closed}

Turn out the light, mother hollers. Windows slam, doors lock on the
first floor, shut tight against hitchhikers, rabid bats who fly by
radar. At 12, the midnight age, I open shades she pulls, call forth
witches and the Seven Sisters from winter skies.

Heads bent low, we read mysteries. Under sheets bleached white with
dreams, we spot clues sprinkled like cherries over mother's buttercream
frosting. We dig for chocolate cake buried, layer upon layer. It
crumbles in our mouths. We guess it is the short bearded stranger or
even the vicar, never the girl in red polka dots. Motive is all, we say.
Where is her intent? We, like her recognize innocence. On the last page,
her sentence is life.

Mysteries crop up everywhere at school. Like what dress to wear or not
wear for luck. I solve algebraic equations, where x equals y squared and
unknowns become known. Answers are right, teachers say, or wrong,
depending on the answer key. I hope for multiple choice, a chance. When
I check Patrick's papers, I give him a couple of points for trying.
Trying to kiss me with mother flicking the porch light off and on like
some lightning bug. Still, his fingers touch bare breast and rest, a
period on the page.

So much blood flows every chapter. Hard-boiled PIs, names of metal and
luck like Mike Hammer and Sam Spade, go it alone. Tough guys, quick with
their fists, they pound flesh and pavement. I fall in love with these
characters. They've seen it all. And, they have done it all. At least
that's what I tell Thomas, years later, when he wants to know why I quit
my job and how we expect to pay our bills, much less make it and get
ahead. Ordinary questions, I guess, from someone who prefers reading
biog­raphies of dead men, models to live by.

Locked room mysteries confuse me since what can't be is. The key usually
is in the victim's pocket. When Thomas hires a security firm to install
electronic sensors, I can never remember the code. Mother says I'm lucky
he's such a good provider. He puts deadbolt locks on her front door and
back, showing her how to work them fast in case of fire. At my house, I
set off alarms, punching in wrong numbers. Finally, I'm fed up with all
the noise. No one understands why I want a divorce. Of course, he says,
you'll change your mind.

Miss Marple clicks her knitting needles, knowing all the time. Liars and
murderers are no more than microbes in a glass of water set on the
kitchen sink. I drink gallons of well water, live dangerously.



\cleardoublepage
\chapter{Computer Cupid}

Some things are tough to figure. If it were all happening to Cami, I
could understand. She's a cheerleader and real popular. But I'm just
average, nothing special

It all started yesterday. There I was in my computer class, minding my
own business. I called up last week's file and got a weird message:
\emph{I'm waiting for you, as always.}

I didn't let it worry me too much, just erased it, figuring someone got
my file mixed up with someone else's. But today when I log in, another
message pops up on my screen: \emph{Quit biting your nails. You're too
cute to have habits like that.}

I blush and look around. That creepy guy in the back row is watching me.
He's probably the one doing this. Just my luck. Too bad it's not someone
like Tom Ryan, smart and cute. But Cami is Tom's type. I'm only his next
door neighbor, a good buddy since kindergarten.

By the third message, I'm getting angry. It says: \emph{My eyes are on
you.} I glare at ``the creep'' all through class, but he doesn't look up
from his computer.

Telling Tom about it after school makes me feel better. He says he'll
help me catch the guy. Tom asks for my code, tells me he'll check out my
files, pats my shoulder and says, ``Don't worry. It'll all work out
OK.''

The next day Tom's waiting for me after his fourth hour computer class.
``It must be one of the guys in my class,'' he says. ``There's already
another message in you file and you told me ``the creep'' left class
before you did yesterday.''

``Maybe he snuck back later and did it,'' I say.

``I don't think so,'' Tom says, shaking his head. His hair falls
slightly onto his forehead, making him look like that guy on \emph{Miami
Vice. ``}But I'll follow him today to make sure.''

Turns out Tom is right. It's probably not ``the creep.'' He doesn't go
near the computer room except for class, and after school he rushes off
to the chess club tournament.

The messages keep coming some more taunting than before: \emph{Just look
around. I'm closer than you think.} Others reassure me: \emph{Don't
worry. You're not your parents. You are you.} Others make me shiver.
\emph{The light in your room went off late last night. Were you trying
to finish your English project?}

A couple of weeks later I get to compute class early and catch ``the
creep'' bending over my computer. ``What do you think you're doing?'' I
holler as I rush up behind him.

``Just need to borrow your manual,'' he says. ``No big deal.''

``Sure,'' I tell him noticing that his computer is still turned off.
Maybe I caught him before he could do anything.

Tom and I take turns following ``the creep.'' Nothing happens.

By the twentieth message, we're still not getting any closer to finding
the guy. Even Tom is stumped.

It makes me feel funny. It's as if someone is looking inside my head and
knows what I'm feeling and thinking, knows things I don't even tell my
best girlfriend.

This guy knows I worry about guys thinking I'm too smart, knows I bite
my fingernails when I'm nervous, knows I hate for my parents to argue.
Still it's kind of nice to have someone know the worst and like you
anyway.

I'm nervous as I slip into my seat in my fifth hour computer class. I've
changed my code before, but he's always been able to break it. I punch
some keys to display my file. Sure enough, there at the bottom of the
screen is another message: \emph{You'll do fine on the history test. I'm
with you all the way.}

I type in: \emph{How do you know I'll do fine? Who are you?} I haven't
admitted to Tom that I answer these messages. He might misunderstand.

I'm almost late to my history class. As I slide into my seat, I glance
over at ``the creep'' who sits two rows from me. I lean over and whisper
to Tom, ``I've got an idea. Maybe `the creep' has a friend do it.''

``Don't be on it,'' Tom laughs.

``It's not funny,'' I tell him as the teacher tells both of us to be
quiet and passes out the tests.

Tom walks out of class with me. ``How did you do on the test?'' he asks.

``Fine,'' I say.

``Knew you would.'' At the drinking fountain he puts his hand on my arm.
``Want to go to the Valentine Dance with me?''

``You don't have to do that,'' I answer, thinking he'd rather ask Cami.
``Maybe `the creep' will ask me,'' I joke.

``Your choice,'' he says, turning toward the second floor stairway and
taking the steps two at a time.

As I walk out the door, Tom honks.

``Want a ride?'' he shouts.

``You didn't have to wait for me,'' I tell him.

``I'm waiting for you, as always,'' he says quietly.

I gasp, cover my mouth with my hands. All my books drop to the sidewalk.
``It's been you all along!'' I scream. ``You tricked me.''

Tom jumps out of the car, helps me pick up my books. ``Had to do
something to get your attention.'' he says and winks. ``After all, you
still think of me as the kid you used to beat up in your sandbox.''

``Wrong. You don't know everything,'' I say, wondering what to wear to
the Valentine Dance.



\cleardoublepage
\chapter{Daily Entries}

I'm reading Sylvia Plath's \emph{Ariel} when Cornelius pushes my
shoulder, knocking my chest against the edge of the desk. And I feel
like I'm sliced in two but a lot he cares.

When I catch my breath, I turn around in my seat to hit him with the
book. All hour he keeps poking me in the middle of my back and
whispering things. (Why don't you see him and make him shut up?

He tells me Sylvia Plath was a sickie and says you look like the picture
of her on the back cover. Says I look like her too. Says it's something
in the eyes that gives sickies away. Says her poems are death poems by
the dead. Only dead people read dead poems he whispers.

I'd like to make him dead. I can do it for my demonstrations speech.
``How to Kill a Person Without Hardly Trying.'' Maybe I can hang him
from that fluorescent tube light that never works. The one that makes
the pitiful beeping noises, flickers, and goes off. Or what about
pinning hm to the bul­letin board with gold thumbtacks?

This room is a perfect tomb. Everything stays the same. Those ancient
books locked in the glass cupboards. And these old desks bolted to the
floor. Only the seats move and creak. Even the stale smell of chewed-up
Double Bubble and Juicy Fruit stuck under my desk reminds me of
gladiolas and chrysanthemums.

I run my fingers over the carved names on my desk, my eyes on the clock
that hasn't changed its time since the first day of class. If the bell
doesn't ring soon, I may just turn around and stab him with my pencil.
The headlines will read, ``High School Student Dies From Lead Brain.''

After class, Cornelius holds me in my seat. He says he's got the car
tonight and wants to take me for a drive and talk.

I tell him I'm busy.

He says I'm not busy and he'll pick me up at 8:00.

I try to spit in his face but nothing comes out. My mouth is dry as the
chalk on the board.

He doesn't give Sylvia Plath back but leaves with her tucked under his
arm.

Can you believe that he actually has the nerve to show up at my house? I
hear him talking to my mother in the living room. They both make me sick
talking about school and kids whose names she memorizes from the
yearbook, hoping one day she'll meet them all. But she hasn't learned
Cornelius' name and she's trying to find out about him.

Cornelius doesn't push my mother's shoulder and he's very serious as he
talks about his advanced placement math course and his plan to win a
scholarship to the University of Michigan.

My mother is impressed with him, with me for finding him. I know this
because she seats him in the overstuffed tweed chair. It's the chair she
usually reserves for the minister, my father, and the Avon lady. She
even gets a plate of cookies and a coke to keep him longer.

I get myself a coke out of the refrigerator and my mother, when she sees
it in my hand, says it'll make my pimples worse. So I quickly guzzle the
twelve ounces while she tells me how I'll get pimples and a stomach
ache.

Most guys would rather walk than take a girl out in a station wagon with
a bumper sticker that reads ``Vote for Nixon.'' And one underneath
saying ``God is Alive.'' The back side windows are cov­ered with decals
from national parks and state attractions.

Cornelius drives with both hands on the wheel. I turn the radio on full
volume and sing along. He turns it down and talks while I count the
seventy-five decals.

He talks a lot about you. Says you're too inhibited. Says you shouldn't
wear pants all the time even if you are ashamed of your legs. Says you
might be decent looking you'd get contact lenses and smile so your teeth
show. Says the way you cover your mouth with your hand is a sure sign of
insecurity.

The only reason I'm telling you this is because Cornelius says he's
going to make his entire journal an analysis of your\\
behavior. He takes on what you do in class. When he shows me the notes
(six loose leaf pages already) I tear them up. But he laughs and says he
makes carbon copies of every­thing.

When he pulls into lovers' lane, I scrunch down in the seat and hope no
one recognizes me. I've got a pack of cigarettes in my purse and two
packs of gum. If he tries to kiss me, I'll just light up a cigarette.

Cornelius turns on the overhead light and I have to lean over to shut it
off. I want to puke when my arm touches his. Instead, I start screaming
at him, telling him how dumb he is.

He just sits there smiling, saying I've got too much pent-up hostility.
He says Mary Ellen could give me some lessons on how to relate to
people. She's the one who sits in the third row, talking to some jock
and fluttering her eyelashes. She even talks to Jerry the whole time you
read Emily Dickin­son. (You know as well as I do that anyone who talks
during Emily Dickinson can't really feel anything.)

I tell him Mary Ellen's too dumb to know anything and through the
stickers on the windows, I can see other cars, couples with their arms
around each other, kissing. Well, I can't really see them but I know
that's what they are doing. And here I am cooped up with crazy
Cornelius.

The stuffing is coming out of the seat on my side. I pull out tiny
pieces, roll them into balls, and flick them off the rear view mirror.
All the time, I think about Mary Ellen and those parties she has.

Finally, Cornelius pulls Sylvia Plath's book from the glove compartment.
As he opens the book and shows me a page, I see he's underlined parts
with black magic marker. I'm so mad, I hit him in the arm with my fist.
He has no right to mark up my book and I tell him so (Would you like
some­one marking up your copy of Emily Dickinson?)

I tell him to take me home and he does. He doesn't even try to kiss me
good night. I'm glad of that. His braces would have shredded my lips.

But I do pause under the porch light to tell him about last night's
dream. He's roped between two desks, spread-eagled, and you're ripping
his journal into small, white slivers that turn into pins and pierce his
body. He's a voodoo doll tied to desks bolted to the floor.

Shaking his head Cornelius says he needs time to\\
figure that one out. Tomorrow night same time he'll pick me up he says.
I tell him I won't be here and slam the front door for emphasis.

In class, Mary Ellen teases me about Cornelius and invites me to her
pajama party. I don't really want to go so I pretend I don't hear her.

Somehow, my mother hears about Mary Ellen's party and I go just to make
her shut up. (I get really sick of hearing my mother talk about Mary
Ellen.) But I don't tell Mary Ellen I'm coming. I just show up.

Mary Ellen looks surprised to see me but she acts okay and takes me down
to the basement where the other girls have already spread out their
sleeping bags. Theirs look like they've just rushed out to Hudson's and
bought them. I really don't want to spread mine out next to theirs and I
wouldn't have until much later except Mary Ellen tells me too.

Mine belongs to my older brother and it still smells like stale smoke
from his Boy Scout campouts and has ketchup stains on it and a hole in
the bottom where he zipped it up and hopped around the campfire.

When I unroll it, the girls hold their noses and get\\
the Airwick to set next to my bag. I laugh with them. It is kind of
funny.

We listen to records, especially \emph{Saturday Night Fever.} Mary Ellen
says that John Travolta takes Cheryl Ladd to disco at Studio 54. Says
Suzanne Somers is dying to go out with him. A skinny girl with red hair
says she saw Travolta's movie seven times and could spend the rest of
her life watch­ing him. Mary Ellen puts her hair up like Olivia
Newton-John's and twirls her way into the bathroom.

The girls put on their pajamas---those shorty, see through kind. Those
nighties reveal everything and I'm glad they can't see my small breasts
through my flannel covering. My mother says not to worry about it, that
after I have children they'll get bigger. But by then, it won't matter
anymore. I've thought of stuffing Kleenex in my bra but I've heard that
a guy can tell when he dances with you. And the points on padded bras
always get pushed in and everyone knows. (Can't you tell that Jennifer
in the fourth row wears one?)

The girls teach each other the New York Hustle but I sit and watch.
After all, we're at that age and funny stuff can happen. Something like
that can mess you up for life. But even Mary Ellen dances, so maybe the
stories aren't true. No one doubts Mary Ellen. All the boys in school
are after her. (Even crazy Cornelius would give up a scholarship to go
out with her.)

One of the girls (I don't know her name but she's in my history class)
shuts off the record player and says, ``Surprise.'' From behind her
back, she holds out a small jar. Mary Ellen, who is yell­ing at her
because she scratched the record, stops in the middle of a curse word
and tries to grab the small white jar.

The girl drops down on her sleeping bag, still holding the jar.

Everyone is quiet.

She takes a movie magazine from her suitcase, flips to the back and
shows us a full-page ad. After the other girls get done looking, I pick
up the magazine and study the ad.

If I looked like the girl in the ad, all my problems would solve
themselves. She looks like me in the ``before'' picture-flat chest,
slumped shoulders, brown hair. She's even biting the inside of her lip:
at least it looks that way to me. The ``after'' picture reminds me of
Mary Ellen---tight sweater (wonder if it's cashmere), rounded in the
right places, blonde hair (maybe it's the lighting), and an easy smile.

All this from a jar of cream. Bust increaser that need only be applied
nightly. Smooths. Softens. Builds while you sleep.

The jar is passed around and hands rub under see-through nighties. Even
Mary Ellen scoops some out and applies it. When I get the jar, I look to
see if anyone is watching me. They're not and I take the whole jar into
the bathroom with me.

Behind the closed door, I massage in the cream. I can already feel it
beginning to work.

I'm going to buy a jar of cream for myself. If I knew Mary Ellen better,
I'd ask her to let me use her address. (Wonder how the school would feel
about receiving a small white jar in a brown wrapper.) Guess I'll just
use my own address and beat my mother to the mailbox. (Maybe I can wrap
my mother in a brown paper and mail her to Mary Ellen.)

Through the bathroom door, I can hear Mary Ellen. She's telling everyone
how Jerry (the one she talks to when you read Emily) calls her every day
and makes out with her every night in lovers' lane. She says she now
uses the pill. I can hear the other girls listening to her.

I look for the proof, the packet of pills behind the mirror of the
medicine cabinet. But there are only the usual aspirin, toothbrushes,
toothpaste, electric razor, and after-shave. I find a rusted safety pin
but no pills. They're not in the closet either. Not behind the shampoo
or the hair brushes. But on the second shelf, way in the back, behind
the new towels, under a pile of frayed yellow hand towels, I find a
diary. Mary Ellen's diary.

Since the key is attached by a red satin ribbon, I open the diary and
read. Pages filled with Mary Ellen's small backward slanting words.
Pages about her and Jerry. Three pages about their first kiss. Pages of
her parents' quarrels, of her desire to get her nose fixed, of going to
sleep and waking up, of boring days, of friends' insults, of daily
chores.

My hands sweat as I lock the diary and replace it.

I've been in the bathroom a long time but I don't want to come out until
I can think of some­thing to say. I start laughing to myself, hoping it
will help me remember a punch line to some joke, any joke. When I hiccup
between the laughs, when the laughs become sobs, when I can't stop no
matter how hard I try, Mary Ellen starts screaming at me.

Mary Ellen says, ``You're weird.'' She pounds on the bathroom door,
still screaming. ``I told my mother you were weird but she made me
invite you anyway. Your dumb mother made her pro­mise Your own mother
knows you're weird.''

I don't care if she screams. Who listens to anything she says?

I just unlock the door and walk out. They're standing there. Quiet now.
But I turn my back on them and walk up the stairs and out the back door.

I walk all the way to lovers' lane, sit behind a large oak where no one
can see me and listen to the cars come and go. There are no station
wagons tonight. Only cars with shadows inside that merge, then break
apart.

All night I sit breaking branches into small bits and thinking how I'll
write this up in my journal.

You're busy when I try to hand my journal to you after class, but I can
wait.

When he leans over and kisses you on the mouth, I think he's taken you
by surprise. But you don't move away.

I'm disappointed. I thought you'd understand.



\cleardoublepage
\chapter{A Dangerous Life}

On February 2, Eliza decides to give up men the same way she decides to
shed her winter weight: quick and clean, like snapping lima beans.

Maybe it's the sharpness of her mother's voice over the phone wires,
heavy as the ice that encases and drags down the lines, making them
touch in spots that are not supposed to come together. The static.

``Can you hear me, Eliza?'' In the background, the mixer whirs, its
motor whining, filling the empty space where she used to live with her
mother back when she was a young girl, just starting to notice the boys
in the neighborhood who stayed out late. ``I'm making your favorite
carrot cake. From scratch.''

The noise distracts Eliza, expands and feeds her hunger for quiet.
``Yes, Mother, I hear you.'' Another voice, deeper, in the background.
``Who's there with you?''

``Just your favorite uncle. That's what I'm trying to tell you. Charlie
just blew in from California.''

Her mother mutters something under her breath to Uncle Charlie, then
talks into the phone again. ``Brought lots of oranges filled with
California sun. Dinner's at 7:00. We're celebrating the end of winter.
You're invited.'' That tinkling laugh her mother uses when men are
around. It flies into Eliza's ear like a gnat and buzzes briefly.

``Sorry. You know I've got that new ad campaign for Branker Trucks I'm
working on.'' Eliza opens the refrigerator door. Empty except for a head
of wilted lettuce and a half-empty bottle of domestic red wine left over
from the New Year's toast she drank at 3:00 in the afternoon with a man
in her office she dated, a man who can't decide if he really wants his
divorce to go through, a man she's wasted two years on. She stoops down
and grabs a can of Trim Line from the cupboard.

``What do you know about trucks?'' Her mother laughs again. ``Remember
how you Uncle Charlie used to drive those tandem trucks through towns we
never heard of? Say, maybe he can give you some ideas.''

Sure he can. Ideas like how to pick up hitchhikers who hijack your load
or how to find the roadside park where a crap game is as good of an
excuse as a blown water pump for being late. Only took a week before he
lost that job. Yet, her mother always defends her baby brother, gives
him money from her waitressing job, even works double shifts when
Charlie needs money to pay off gam­bling debts.

Eliza can still hear her mother from back then chanting again and again
like some kind of mantra, ``He lives such a dangerous life.'' Then her
mother shivered with excitement, ``Don't ever say I said this, but those
men may kill him if he doesn't pay up.'' She always emphasized ``those
men,'' as if they were men she would like to meet, men from foreign
lands.

``Can't make it tonight, Mother. Besides, the trucks are toy trucks. For
kids.'' Eliza chugs down the can of Trim Line, a sweet-tasting
concoction that promises instant results. ``Got to go. Miles to go
before I sleep.''

Her mother talks without listening, talks until she wears Eliza down, a
sliver of herself. ``So what do you know about kids? You never had any.
You were never even a little kid yourself. Born a wrinkled-up adult, all
red and squalling.''

After 39 years of this reminder, Eliza knows it's senseless to refute
anything. On this day, she feels like the groundhog. She will emerge
from her hole and, despite the sun and shadows, scurry across the frozen
field, daring anyone to force her back into her burrow beneath the
earth's crust.

Once started, her mother can't stop. ``Always worrying and fretting
about things. Your father used to call you `little mother.' We both sort
of figured you'd have lots of kids, bless his soul wherever he is.''
Pause and rewind. ``Are you anorexic again? I make your favorite cake
and you can't bother to come over?''

Although she's often heard the story about her father, Eliza suspects
her mother made it up, something to justify her leaving Eliza alone when
she took off for the different restaurants she worked in, none of them
for very long. So many names: the Lighthouse, the Alamo, the
Alpines---all high places for making a stand. Her mother used to say she
would never work for a restaurant that had a common name or family name:
as if people can own the food they serve. For her mother, food is love.
To deny food is to turn down love.

Eliza can't remember her father, even when her mother shows her
snapshots, loose and slip­ping out of the corners in the photo album.
Instead, Eliza remembers the men her mother found eat­ing alone and
brought home. When Eliza woke, the new man was there, eating at the
breakfast table, trying to make jokes with her.

Only Uncle Charlie ever came to stay for more than a week. Then he would
take off, too, usu­ally on some rainy morning when the sun didn't hurt
his eyes. Eliza would wake up and he wouldn't be there. Her mother made
a game of it, looking around for him and asking her where he went, as if
Eliza could pull her Uncle Charlie from a hat, if she could only find
the hat.

In a while, her mother would shake her head, laugh a deep laugh that
rumbled around in her belly until it rose up and burst from the back of
her throat, an explosion. She always said the same thing, ``That
Charlie. Off for more of his adventures.'' She would cook up a big batch
of potato pan­cakes, smother them in her favorite apple sauce and wonder
aloud about where Charlie would go this time.

Her mother would even thumb through an old Atlas and they took turns
guessing where he might end up. Eliza always chose Alaska, a wild
territory, only recently named a state. She liked the names of the
towns, almost promises. Names like Anchorage, Fairbanks, Ruby, Point
Hope --- wide open spaces with only the wind to carry the prayers of men
who stood with water up to their knees, scooping up rocks from the
riverbed, hoping for the flash of gold.

``It's not good not to eat, dear. I never knew you to turn down carrot
cake, at least not the way I make it. Maybe you're working too hard.
There's more to life than working.'' Dishes clink in the back­ground.

Eliza imagines her mother, arms sunk up to her elbows in dishwater,
bubbles rising. One shoulder raised and head bent, cradling the phone.
Cord stretched taut across the kitchen, ready to let loose from its
anchor on the wall.

``Eliza, talk to me. I can tell something's wrong. I hope you're not on
one of those crazy diets again. Charlie and I are coming right over.''

``All right, I'll come,'' Eliza says, slamming down the phone. But she
doesn't want to go, doesn't want to watch her mother cutting pieces of
cake for her and Uncle Charlie. Doesn't want to watch her mother put on
more pounds.

But her mother doesn't care about things like weight, never did. Just
pats her wide hips and says men like a little flesh on the bones,
whenever Eliza chides her about her health. Never even cares when her
ankles swell most nights from carrying around all that weight. Claims
she still gets the biggest tips.

Just last week, she spent an hour telling Eliza how she enjoys watching
men eat, even puts extra scoops of mashed potatoes on the plates of her
regulars. Then she laughed, a sure sign that she found another man to
bring home and feed. Their coming and going never bother her mother.
Always says all a woman has anyway is herself to count on.

Eliza knows she is different from her mother. If she becomes weightless,
she will be able to float free from the pull of gravity, free from the
fleshy legs that hold her upright. She believes in more than food, more
than men. She believes in magic and she blames Uncle Charlie for that,
for pulling quarters from behind her ear when she was a kid, for taking
off for Alaska without her.

Eliza has a deadline to meet on the Branker Truck Account and she's
never missed a deadline in her life. She still has a couple of hours
before she has to be at her mother's house. Time enough. Opening her
briefcase, she takes out the sketches, mock-ups showing the series of ad
proposals. All she needs to come up with is the copy, words round as
cherries, shiny enough to seduce ten-year-olds.

But she crosses out each word she writes, obliterates it with lead,
moves the pencil point back and forth until the paper rips. Before,
words have never failed her. They blossom into fruit beneath her
fingers. Maybe it's the sketches, miniaturized, small versions of real
trucks instead of the thing itself. She wants her words to make the
trucks zoom, capture tire treads hot against concrete, the speed of an
empty tanker cresting the hill, coming home, large enough to fit into a
child's cupped palm. Eliza can almost feel the power of it all, the
engines throbbing.

Not good in snow, her own Camaro, new and bought for speed, for handling
when roads are dry. The sun has already set when Eliza slides into her
car, the cold leather warming with her body heat. It's still too early
to celebrate the end of winter with her mother and uncle. But she will
surprise them and show up before the appointed time. The days are
beginning to get longer.

Eliza will drive by her lover's house first, a confirmation of her
morning vow. Sometimes his wife does not draw the drapes against the
dark and she can look into their lives, see them watching television or
one time they were repainting the living room walls. She no longer wants
to sit there on his couch, performing daily rituals.

The Camaro's rear wheels slip as she turns right, onto her lover's
street. The rear end skids, but Eliza goes with it, doesn't fight it.
She brings the Camaro under control, keeping it from smashing into the
van parked in front of the colonial two-story that was just built last
spring.

During the opening of the colonial model, she traipsed through as if she
were a prospective buyer. Even asked lots of questions and took a
brochure about the new subdivision, wondering what it would be like to
live there. The real estate agent gave her a card, urging Eliza to call
as soon as she made up her mind. She never told her lover about doing
this. Even though he lived three blocks down in an older house, he
wouldn't have liked what she did. But, then, there were lots of things
she never told him.

As she pulls up in front of her lover's house, she turns out her
headlights and wonders why so many lights are on at his house. Wasteful,
she mutters to herself. As if energy were cheap these days. She sees
that he did not shovel the walkway and the Christmas lights are still
up. Thank goodness, she thinks, he has the sense at least to keep them
turned off. Why has he left the bike outside, leaning against the
garage, letting it rust as if it has no value to anyone? She wonders if
they are both inside, shut in by the deadbolt locks he said he installed
last month on both the front and back doors after their neighbor's house
was burglarized.

Eliza presses steady on her horn as she takes off, hoping her lover
hears the noise, unexpected as a shooting star. Her headlights streak
the night, catch another car unaware as she crests the hill and turns on
her brights. The roads are salted, dry, safe as she pushes the Camero
past 60, rejoices in the way it hugs the road and translates every curve
into a foreign language.

Eliza passes a bush: round, squat as her mother who is rooted,
self-contained within her skin. Not like me, Eliza thinks. Her mother is
a woman who has withstood droughts, tornadoes, and other natural
disasters, even her daughter. Even Uncle Charlie, who Eliza believes is
really her father. Maybe she just wants him to be her father. Out loud,
Eliza says, ``Father.''

Mother and father. The two of them together: earth and air. She will see
them tonight. She will eat her mother's cake, peel California oranges
and suck the sweet juices. Her words will flow, fluid and nourishing as
orange juice. She will show Uncle Charlie that she can pull silver
dollars from behind his earlobe, from behind both earlobes. She has been
practicing.

She who never speeds pushes up past 80 along these back roads. Now the
silence, the quiet darkness whipping past her. The night wide open.

Eliza breaks the sound barrier, slim, sleek in her new shape. Her hands,
magician hands, nimble enough to turn the wheel, rehearsed enough to
levitate the car in space, held to earth by the invisible strands of her
beliefs, by her desire to live a more dangerous life.



\cleardoublepage
\chapter{Double Date}

John sits across from me, dips French fries in ketchup as he thinks. I
drink my coke, trying not to break his concentration.

``Just ask them straight out,'' John finally says.

``And if that doesn't work?'' I ask, stirring the ice in my Coke.

``It'll work.''

I guess that kind of confidence comes with being the most popular guy in
school. Captain of the swim team and everything else.

``They might not go for it,'' I tell him.

``Not at first maybe, but they'll come around if you just sit and look
sad.''

``I remember how much my parents like John. ``A real level-headed boy,''
my dad usually says to my mother.

``You might not like camping,'' I say. ``Your clothes are always
wrinkled, the showers cold. Sand in everything. Ick!'' I slurp the end
of my Coke.

``With me along, maybe you won't mind cold showers,'' John laughs.

I pick up John's biggest French fry and feed it to him. Suddenly he
chokes, looks past my shoulder. I turn to follow his glance and see Mary
Beth leaning against the table behind me, watching John.

As usual, Mary Beth looks like she stepped out of a Clearasil ad, the
smooth-skinned cheer­leader who leads the crowd to victory.

Mary Beth steps to John's side of the table, her back to me. She tells
John about her California vacation, her voice soft as her hair. ``Miss
me?'' she says.

I excuse myself and walk to the restroom, my shoulders straight. I slam
the door, lock it.

Resting my forehead against the mirror, I remember John's and my first
date after he broke up with Mary Beth. When he first asked me out, I
turned him down. Figured he'd lost a bet, even though I can't picture
him losing anything.

But he kept asking and finally I said okay. Thought once he went out
with me that would be the end of that. He was used to the Mary Beths,
not klutzes like me.

Mary Beth probably stayed at some luxurious California hotel. Camping's
not her style. I worry that it's not John's either. What if he ends up
hating it, hating me?

I open the door quietly, smile when I see that Mary Beth is gone.

Before I'm even in my chair, John says he just knows my parents will let
him come along. Says he's so sure, he's already got his suitcase packed.
``Just remember to let them do all the talking,'' he cautions.

``Too bad it isn't California,'' I say, watching his eyes.

``And what's wrong with Traverse City?'' He touches my arm. ``You really
shouldn't let people like Mary Beth get to you.''

Sometimes his sureness gets to me. Things always work out for him; he
knows just what to do. For a moment, I hope my parents will say no.

John drops me off at my house, tells me to call after I talk to my
parents.

Before dinner, I stand in front of the mirror and practice what I'm
going to say. Nothing sounds right. If I mess it up, it's seven days
without John, plenty of time for Mary Beth to work on him.

During dinner, the words tumble around in my head. Dad teases me about
being in love. Dur­ing the apple pie, Mom asks if I've got everything
packed. She hands Dad the newspaper and starts to clear the table.

When I finally talk, my voice sounds hoarse. All the words leave me and
I say right out, ``Can John come camping with us?''

Mom sets down the plate she just picked up. Dad looks up from the front
page. Dad says it wouldn't look right; when he was young, boys didn't go
on vacations with girls.

Mom agrees but reminds him times have changed. ``Besides,'' she says,
``John's a nice boy.'' She stacks silverware on the plate. ``It might
give her something to do besides mope around the tent all day writing in
her diary.''

Dad remembers last year. I can tell by the way he combs his mustache
with his fingers. ``Couldn't even get her to go fishing with me last
year,'' he says.

Mom nods her head.

``Does John like to fish?'' Dad asks, pulling the sports' section from
the newspaper. That's his way of saying okay.

I pull the phone into my room and punch out John's number. When he
answers, I tell him the good news and ask him if he knows how to fish.

John says, everybody knows how to fish. I feel guilty for doubting him.

``Traverse City, here we come,'' I say.

As we talk, I polish my toenails pale pink. Reminds me of the inside of
a shell, the kind you hold to your ear and hear the sea.

I dance around the room, John's voice strong in my ear.

John teases me about being too happy, warns me I'll grow up one day. He
asks what makes me unhappy.

``Purple nail polish, the second day of school, and guys who can't
fish,'' I tell him as I stub my toe on the dresser.

I'm on my best behavior all week, do the ironing without being asked.
Even I begin to think I'm going to collapse from so much goodness.

But finally, after we wedge John's suitcase between the tent poles and
canned goods, we're all in the car---Mom and Dad in front, John and I in
back; sort of a double date.

I look out my window and try to remember last night's dream. John
catching minnows with his cupped hands, dropping them back, swimming off
for the white water lily by shore, leaping from the water like some
shining trout and putting the flower behind my ear.

But the fisherman of my dreams is talking to my dad about last night's
Tigers game. When I point out the deserted farmhouse, neither hears me.

We stop at a small diner and instead of sitting down at the table, I
continue walking to the rest room in the back.

When I come out, two guys about my age are sitting at the counter. One
punches the other. The punched one turns and whistles.

When I reach the table, John's looking at the menu, pretending he's gone
deaf. My dad's figur­ing the gas mileage on his napkin. My mother's
cleaning the spoon with her napkin before she stirs the coffee.

I stare at the blowfish hanging from a string over the cash register.
Probably gets fatter every time a sale is rung up.

John asks me what I'm staring at and my parents turn to look.

``Looks like you when you get mad and puff out your cheeks,'' Dad says
to me, laughing. ``Ever notice that, John? Get too close to a fish like
that and the spines puncture the skin. But it doesn't kill you, only
makes you more careful.''

``Yes, sir,'' John says. ``I'm careful when she gets mad.'' They both
laugh and I feel betrayed.

I tell John he's never really seen me mad and he reminds me of the time
the carnival guy conned us into playing the shell game. John says that
telling everyone within shouting distance that the man's a shyster was a
pretty good sign I was mad. I tell him that doesn't count; he says it
does.

Mom tells us we sound like an old married couple and I shut up.

In the car, I burn my thighs on the leather seat as I scoot closer to
John and mouth the word, ``I love you.'' He blushes.

We pass a metallic blue camper on the road, decals from all over the
United States plastered on the back window.

``Look at that,'' John says. ``Must have a television set and
everything.''

``Yeah,'' my dad mutters. ``Those things are taking over the
campgrounds. Last summer a long silver job came roaring in at night.
Leveled our tent just like that,'' and Dad snaps his fingers.

``It was the best part of the vacation,'' I say. ``We talked about it
for days. I mean, what if we'd been in there sleeping?''

``And I thought the best part was the Thompson boy,'' Dad says adjusting
the rear-view mirror so he can see into the back seat. ``You sure had
your eye on him.''

``I did not. Besides, he was only fifteen.''

John winks at me.

My mom says the Thompson boy was nice, even stopped by every afternoon
to show her the fish he caught.

``Sounds like a weirdo,'' John whispers to me.

``Anyone who panics at a couple of bloodsuckers on his arm has to be a
little off,'' I say.

``Bloodsuckers,'' John says, cracking his knuckles.

Rolling down the window, I shoo a trapped fly toward it. I get the fly
out safely, roll up the win­dow and lean my head against it. Soon I'm
asleep. A sudden jerk bounces me against the glass and I hear my dad
saying he can't help it if the boulder was hidden by tall grass. My
mother says she warned him to check before he drove in. John says it's a
nice place to camp and I open my eyes.

We all unload the trunk. Mom sets up the cooking area. I hand Dad stakes
for the tent and pull out the sleeping bags. John gets in the way until
Dad sends him to look for firewood.

When it starts to get dark, we cook hot dogs on an open fire.

``I thought you had a camper,'' John whispers as he swats mosquitos
landing on his legs.

In sleep, my mom and dad separate John from me. His snore, quiet at
first, then louder, surprises me. The breath leaves his mouth in spurts.

In the morning, I tell him he snores. He says he doesn't. When I ask him
if he ever hears himself sleeping, he says I twist everything around.

``Only old men snore,'' he says as I toss him his bathing suit from the
line.

``Let's go swimming, old man,'' I say.

He thumbs his nose and goes into the tent to change. When he comes out,
he looks glorious, his team suit trim at the hips.

My turn. In the hot tent, I pull at the bikini. Maybe Mom's right. Maybe
it is too skimpy, but like I told her, I'm not a little girl anymore.

When I open the tent flap, John looks up from my dad's fishing tackle
and grins. He asks if it will hold together in the water. I wonder the
same thing.

The lake is just over the hill. John picks a bouquet of daisies and
hands them to me, bowing. They wilt quickly in my warm hands and when we
reach the lake, I swish them around in the water and weight the stems
with a small rock.

John wants to race me to the raft. I tell him I'm game and start my dog
paddle, splashing water into my mouth. I stand up choking, spitting
dirty water.

Warning me not to swallow any of the water, John shows me how to
breathe, holds my arms so I can practice kicking. I kiss his arm.

``You'll never learn if you keep fooling around,'' he says.

I jerk my arm away and stand, the lake bottom solid against my feet.

``And if I don't want to learn?'' I say. ``If I don't want to swim like
Mary Beth, what then?''

``You're not Mary Beth.''

``And don't you forget it,'' I tell him, wishing I were Mary Beth,
wishing I knew how to swim and had a camper instead of a tent.

I swim alone toward the raft. He waits until I'm halfway there and goes
whizzing past me. He pulls himself onto the raft, lies on his belly, his
head over the edge.

``Here comes the mongrel pup. Sink or swim,'' he shouts.

He helps me onto the raft. I lean my head back toward the sun, waiting
for it to bake out the anger.

I tickle him and he laughs so hard he falls into the water. I jump in
next to him and dunk him. He comes up spitting. He tells me to grow up
and I tell him growing up doesn't mean you have to stop having fun. He
asks if I can imagine my parents tickling each other. I must admit the
picture just won't focus.

``Well, maybe they should try it,'' I say.

He ignores answers he doesn't like. ``This lake is kid's stuff,'' he
says. ``Muck and junk floating around. Indoor pools, that's the way to
go,'' He leans back on the raft, shading his eyes to look at me.

``I don't like chlorine in my eyes,'' I say sadly. I'm not sure I can
like anyone who doesn't like lakes.

``Like I've said a million times. You're just a romantic and they don't
get places.'' He rubs my shoulders, loosening the muscles.

``Here is someplace,'' I tell him, knowing Mary Beth would say it was no
place.

He laughs and dives into the water. ``Try to catch me.''

Even if I could, I wouldn't want to catch him, at least not now, not in
the water.

He's back on shore before I even leave the raft.

When I finally drop down beside him on the shore, I'm tired. I trace a
scar on his leg with my index finger.

``Bloodsuckers,'' I say, pointing to three on his leg.

He tries to brush them off with his hand. ``Don't just stand there. Do
something,'' he says.

``Like what? Amputate?'' I'm beginning to really enjoy myself.
``Spiders,'' I say, ``That's it. Spiders eat bloodsuckers. I'll go find
some.''

He grabs my arm. Tells me he doesn't like spiders, to find another way.
He keeps shaking his leg, trying to loosen the bloodsuckers.

``Don't move around so much, makes the blood flow faster,'' I warn.

He sits in the sand, his bloodsucker leg straight in front of him.

I feel sorry. He looks so helpless. But sorry or not, I need to know.
``Can you really fish?'' I ask.

``Looks easy,'' he says. ``Read some books about it once.''

With a stick, I gently scrape off the bloodsuckers. I tell him salt
works too.

John sits staring at me, pokes a finger in my cheek. ``Like your dad
said about the blowfish.'' He pushes a strand of hair from my eyes.
``Maybe your dad will teach me to fish.''

I hug him. ``What if you don't catch anything?''

``Nope I don't,'' he says. ``Don't like worms any better than
bloodsuckers.''

I tell him my dad doesn't either. That's why I always put the worms on
the hook. Bet that would make Mary Beth sick.

``And he calls himself a fisherman?'' We both laugh.

``Race you back to camp,'' John says taking a head start. ``I really
never did like Mary Beth,'' he shouts back over his shoulder.

I stand up and race after him.



\cleardoublepage
\chapter{Everything Was Different}

The room was a misty grey, but it seemed useless to turn on a light in
mid-afternoon. Valencia had tried to read the book, but that, too was
useless. She thought of how a college professor had once praised the
book for a whole hour. But she had found only words on a printed page.
Black on white. Nonexistent people created by a restless man. So she'd
let it fall open against her sagging breasts.

Her legs were propped on an orange-flowered ottoman to relieve the
pressure of her bulging stomach. She massaged her thighs gently,
stopping to trace where a blue vein had broken and spread its spider
branches under her skin.

A record was rejecting for the third time and she walked over and turned
it off. Outside the window, it was still raining, ugly, foggy drops
beating on the glass, making the green grass blend with the brown
patches. And a car passing, the wheels thinning the puddles.

She couldn't stand being boxed in her thoughts a minute more. She kicked
the book under the chair when she passed. As she slipped her arms into
the raincoat, she noticed a rip in the lining and didn't even care. She
walked across the street to Sandy's house.

``You little idiot,'' Sandy said as she opened the door and pulled
Valencia inside. ``Why, just look at your hair. It's soaked.''

And Valencia shook some of the dampness from her long, auburn hair as if
in obedience. Sandy took her coat and returned with a towel.

``Am I ever glad you came over. It's been one big storm in this house
all day. The boys can't get out with all this rain. They've both had
those summer colds, you know. And they're giving me a pain. Its' great
to see a civilized human being who can sit in a chair without pulling at
the buttons or slipping crackers under the cushion to see them get
smashed.''

``I'm not feeling any too civilized today. Just feel like I've got to be
away from myself and with someone. Let's go out to lunch and talk for a
while.''

``Sounds marvelous. But I can't get a babysitter in the afternoon and
Jack will be\ldots''

``Forget it.'' The words are clumsier than she meant them to be, like
opening a tin can with pliers. It frightened her that her mind was
growing clumsy along with her body.

Valencia looked at Sandy's short hair which was tousled as if she'd just
been awakened from a nap. She watched her quick-puff her cigarette and
turn it in her fingers as if she was trying to hold on to something.

Valencia folded the towel and pulled pieces of hair through it. She
wished she didn't have to pretend not to know that Sandy was concerned.
That was the pain of people.

``What do you think I'd look like with one of those new short
haircuts?'' she said to Sandy. ``I've been thinking about getting it all
chopped off.''

Valencia remembered how she'd had it piled on top of her head the day
she was married and the day at the beach when Ed had pushed it back
behind her ears. She listened to it squeak as she pulled it taut between
her fingers.

Looking at the kitchen clock, Sandy answered, ``You'd love it short.
Just whisk a brush through it in the morning.''

There were screams from the bedroom and the sound of metal hitting
something. Then crying and a door slamming. Always a lot of sounds.
Billy, the four-year-old, was laughing. ``That'll show him. He threw his
truck at me. But it hit the wall and the wheels fell off.''

He looked at Valencia and his face wrinkled. ``Gee, are you getting
fat.''

He held his gun up to his eye and pretended to shoot her.

``Hey, you gotta fall dead. You don't play right.''

``That's enough, Billy Get back in your room. Walk,'' she said as he
skipped back and left the lamp rocking. But it didn't fall.

The bedroom screams resumed and Valencia ran her hand lightly across her
ballooning stomach.

Her mind rocked back and forth as her hand reached out to straighten the
lamp. If only Ed would stop smiling. If only Sandy could\ldots{}

``Hey, Sandy, hasn't he ever asked that question before?''

``What question?''

``About being fat.''

``Sure, but he only asks it. Doesn't really want an answer.''

They looked at each other a long time, both wondering what to say. Sandy
lit another cigarette. She exhaled the smoke and split the humidity with
her question. ``You getting excited yet? Just a couple more months.''

Valencia started to tell her but instead bit the corner of her nail and
shook her head up and down.

``Getting the jitters?''

``No.'' And she wasn't . That was the ache. She wasn't getting anything.
``I think I will get my hair cut.''

``That's because you planned it all so carefully. That's smart. Both of
you working for a couple of years. Wish jack and I had done that. No
worries of any kind.''

Sandy drew her feet under her and yelled at the boys to keep quiet. Then
when they were quiet for a minute, she yelled again to ask them what
trouble they were building now.

``That's right, no worries. It's best to be practical.'' And she found
herself looking at the clock every few seconds as Sandy did. The cheap
gold hands slowly marking off another day, drawing closer.

When Sandy asked her into the kitchen while she fixed dinner, Valencia
said she had to get home and do the same thing. Sandy looked relieved.

As Valencia rose from the chair, the towel fell from her shoulders to
the floor. Quickly she bent over to pick it up and found her stomach
caught between her shoulders and knees.

``Damn,'' she muttered as she leaned over sideways and almost lost her
balance.

Sandy helped her into her coat, handed her a black umbrella, and said,
``Jack's bigger than mine. Don't worry about returning it right away.
The only time he remembers he's even got an umbrella is when he goes to
the back of the closet for his golf shoes.''

She laughed and Valencia smiled politely, almost embarrassed to be let
in on a man's quirk, the thing that made him human, the thing that
should be guarded.

The rain fell about Valencia, not touching her. If only it would touch
her, she thought, like when she was a child and put on her bathing suit
and built dams out of small twigs and the dams broke and the water
rushed over the brown lines and flowed down the sewers.

But she wasn't a child anymore. She couldn't just do things; there had
to be a reason. And she couldn't find the reason and Sandy couldn't tell
her; it wasn't fair to ask.

She folded the umbrella and stood by the car, looking at her house. She
didn't want to go in; it was too small for both her and Ed to be
themselves and now with the baby coming. Everything seemed to be
smothering her.

It was awkward climbing into the car, but then there was always
awkwardness involved in things like this.

In the city, people were driving. Looking ahead. Stopping for stop
signs. Making turns. And she performed the same routine functions.

Clicking on the radio, she listened to the music and commercials without
hearing. She was trying to think of a place to go where she could be
alone, could embrace herself and murmur lovers' secrets, lies that
explained so many things. Bust she couldn't think of any place.

``I've got to get back,'' she said aloud as a disc jockey announced,
``It's time for the 4:30 news.''

When she was home again, she stuck yesterday's roast in the oven,
knowing that Ed wouldn't say anything now. Then she plopped back in the
chair and tried to read the book again. But soon she was counting the
number of cars that passed the house; she could tell by the faraway
humming and clicking that grew nearer, and the splash that sent rain
sounds into her lap as the car passed. She waited for Ed.

Finally she heard his car. The door slamming and the footsteps slapping
the concrete. Ed was ugly as he stood in the doorway, large shoulders
blocking out the rain behind him, a tired puff about his eyes and the
black stubble on his chin that he shaved away twice a day. He reminded
her of a bull as he wiped his feet on the worn rug and she smiled with
her eyes for the first time that day.

``Here, let me take your coat.'' Her heaviness seemed almost gone as she
started to jump up and then remembered.

``That's all right, Val. Stay where you are.''

But Valencia took his coat and stood there, waiting. Ed walked into the
bathroom and shut the door; the water was running.

When he came out, she was standing by the door. They looked at each
other and she said, ``Kiss me,'' and he did and she wished he hadn't.

She listened to it rain the rest of the night and she could hear Ed
snoring gently, muttering something, and watched him as he turned away,
his face soft against the pillow. And she cried, enjoy­ing the
salt-sting on her face. Eventually she slept.

Dusty rays of sun hit her back the next afternoon as she ran a white rag
against the damp head­light. She was glad that Ed had been able to come
home early and thought about suggesting a picnic. She knew he'd go if
she asked, but he wouldn't like it, not in the middle of the week. Maybe
they'd go on Saturday.

She could feel him watching her and when she looked over at him, his
eyes were not laughing, but smiling carefully, as if they might break
her if the glance was too rough.

She caught a handful of water and tossed it at him; a few drops hit him
on the forehead. His green eyes pulsated with the old look. He pointed
the hose at her but then quickly turned it aside so that the jet of
water hit the tree to her left. She frowned, dipped into the bucket and
wiped the fender.

``Something wrong?'' Ed had stopped rinsing the car and was watching
her. ``Maybe it's too hot for youth be out here like this.''

Smiling, she shook her head. She knew he was trying, really trying, for
her sake. But she wanted things to be like they used to be, and she knew
they could never be that way again. And she knew no one could help her,
would want to help her. After all, they'd planned on the baby for a long
time; they'd both wanted it. But she felt cheated now; she was changing
her mind and she knew it was too late for that.

``Want to go shopping tonight for the crib?'' Ed was rinsing the front
tire off, no longer watch­ing her, offering his support.

``But you have to shop.''

``Not for the baby's stuff. That's different.'' He was watching her
again as she rubbed the hood dry.

And that was just it. Everything was different now. She emptied the
dirty sudsy water and rolled the bucket.

``It's like old times having you home early. Remember how we use to go
for long drives, out to dinner, and then to a drive-in?'' She laughed,
set the bucket next to him, and ticked him in the side as she continued
talking. ``You always spilled the popcorn.''

He turned to grab and tickle her, but instead kissed her forehead.
``Yeah, it was fun. But it wouldn't be comfortable for you at the
drive-in now. Besides, you always spilled the Coke.''

He bent to steel-wool the tires and she walked back around the car,
trying to remember if she'd really ever spilled Coke at the drive-in.

She waved to Sandy who yelled from across the street, ``Have you seen
the boys?''

Valencia shrugged her shoulders and nodded. Sandy never seemed to be
able to find the boys when she wanted them.

Valencia climbed up to clean the top of the car. The sweat was running
in small beads down her temples and inside her blouse collar. Using the
rag, she wiped off her face. It felt cool now.

``Hey, Val what the hell you doing? I'll get that. You just get the
lower parts.''

He put his arms around her waist and helped her down, smiling as if he
knew something.

``Damn it. Stop smiling like that. It makes me sick.'' She threw the
dirty rag into the bucket and rain into the house, slamming the door.

But she could hear Ed whistling and shut the kitchen window to stop the
sound.

If only he would be mad at her, she thought as she took some lettuce
from the refrigerator. She licked her finger and rubbed it against a
dirt mark on the door.

All the time she was washing and shaking the lettuce, she remembered
times he'd really been mad. That time she kept calling him a
son-of-a-bitch, he'd hit her, and once when she'd locked the bathroom
door, he'd smashed it in and had fixed the lock the next day. But now it
was that stupid smile all the time. He tried so hard and she felt sorry
for him.

Vinegar. Top shelf over the refrigerator. She moved a kitchen chair over
and remembered he didn't like her to climb. Only a couple of months,
he'd remind her. Well, let him count the time.

She glanced toward the window as she lifted her foot onto the chair and
kneed herself in the stomach\ldots{} balance lost. She was lying on the
floor. And all was quiet.

She hurt somewhere. She couldn't hear Ed whistling and it was impossible
to talk. She didn't know what she'd say anyway.

Nothing. The silence terrified her; the dead sparrow flies. Maybe she'd
heard it in a poem. Only her thoughts moving, all that there was of
life. Hushed velvet slippers crossing snow.

Then she heard the grinding of the electric clock. The baby kicked
against its confines. Time was passing and the baby kicked again and
then was quiet, the sleeping death of life, waiting.

Valencia felt better now. She'd just lost her breath for a moment. She
stood, pushed the chair back under the table, and opened the kitchen
window.

``Hey Ed. Can you reach the vinegar for me?''

She waited, smiling patiently.



\cleardoublepage
\chapter{A Fine Day}

The nurse rubs Vaseline on Juanita's swollen finger, twists the wedding
band over the knuckle. Juanita moans. She isn't ready for the next
contraction, can't remember any of the Lamaze class lessons, doesn't
want to remember.

Pinching her arm, Richard tells her to breathe. ``Why do you make
everything so hard?'' He sticks out his tongue, pants. ``Like this.''

She wants to yank out his tongue with forceps.

``Goddammit, breathe right."\\
She scratches at the starched sheets, holds her breath against the pain.

``You can do it,'' her husband says as she begs for the needle. ``Bite
on this.'' He slips a sponge between her teeth.

Her teeth close on his finger.

``Goddammit.'' He squeezes his sore finger, rocks it between his knees
and back to his chest. ``That hurt, goddammit.''

When the nurse comes with the needle, she holds out her arm.

``If you only tried,'' Richard says from the flowered chair by the
window. Newspaper pages crackle as he turns them; his voice floats above
her, reading the baseball scores.

After the doctor drops the baby on her abdomen, after he stitches the
wound and washes his hands, she is alone in a room. Curtains are pulled
around her and she sleeps.

Her breasts that leaked for nine months go dry. Her daughter cries
against her and the nurses bring her less often. Juanita is glad. Her
daughter reminds her of the chicken in her freezer---feathers plucked,
skin puckered.

In the middle of the night, the new mother in the next bed bites into
apples and sucks the juices from oranges. She whispers through the
curtains; Juanita feigns sleep. In the morning, the woman buys pictures
of her infant. Juanita orders none.

When Richard visits, the woman is nursing her new son. Her face, like
her exposed breast, is smooth, full. She laughs at Richard's jokes. Her
son's feet straddle her breast as she burps him against her shoulder.

Richard turns away, hands Juanita a potted geranium he is holding.
``Spot of color for the front porch.''

``I'll keep it in the bathroom.''

``Not enough light there.'' He kisses Juanita on the forehead and takes
the apple the woman in the next bed offers.

``Want a bite?'' he asks Juanita.

Juanita shines the apple on her nightgown and takes a bite. She gags and
spits the half-chewed pieces into a Kleenex. ``Wormy,'' she tells
Richard.

Richard holds the apple close to the light on her headboard, examines
it. ``Just your imagina­tion.'' He bites and the juices glaze his lips.

``What are we going to name her?'' Juanita asks.

``Who?'' Richard closes the curtains and turns on\\
the television.

``Your daughter.'' She pushes the off button and moves the control box
to the other side of the bed.

``I don't care. The mother is supposed to choose the daughter's name.''
Richard pats his pock­ets for a pack of cigarettes. ``Can I smoke in
here?''

Juanita's mother parts the curtains. Her fleshy arms jiggle as she moves
forward, holds out two packages. ``For the new mother from her mother.''
She hugs Richard until his body is lost in hers.

In the first package are nursing bras, the front flaps folded down so
the back tag shows.

``Thirty-six D,'' her mother says, winking at Richard. ``Didn't I tell
you pregnancy would make a woman out of her?''

``Mother please, Juanita says. ``I can't breast feed. There's no milk.''

``Nonsense. The women in our family always have enough to feed
triplets.'' Her mother unties the second box and removes the cellophane
wrapping. She eats the caramel in the center and picks out a
chocolate-covered cherry for Juanita.

``Let Mother pick the name,'' Richard suggests.

Juanita shakes her head.

``You got something against your name? Juanita is a nice name. Don't you
think so, Richard?'' Mother picks caramel from her back molar.

Richard does not answer. He's pushing in the tops of the candy to find a
soft center.

``All a girl needs is a pretty name. I read your name in a book.''

``What book?'' Juanita asks.

``One of those detective magazines. Your name was right on the cover.''

Richard's laugh is too loud for this room of mothers. A nurse pokes her
head through the cur­tain to remind them that visiting hours are almost
over.

Juanita closes her eyes and rests her head against the pillow. She does
not want to walk her visitors past the nursery where they will stop and
try to pick their flesh and blood from the other pink bundles.

During the eleven o'clock news, Juanita uses her cuticle scissors to cut
the nursing bras into strips. She cuts each thread holding the hooks and
eyes. Some seams she rips apart with her hands. The weatherman predicts
sun and a high of eighty as she puts the cotton strips into their box
and dumps it in the wastebasket by her bed.

She dreams names of childhood playmates---Lori, Judy, Cindy,
Katherine---but none seem right. Through breakfast, she says names
aloud: Esther, Ruth, Mary, Sarah, Hannah. In the book of names the
hospital provides, she finds Amanda, worthy of love. This is the name
she writes on the birth cer­tificate. The meaning of the name, she
repeats to herself---over and over, a litany.

Amanda develops jaundice and is kept from Juanita. Through glass, she
watches her baby. Although Amanda sleeps and does not open her eyes, the
nurses assure Juanita that she is doing fine. Juanita points her out to
other parents and grandparents. Tells them Amanda is starting to gain
weight, is losing the yellowish color that made her look like a chicken.
Juanita studies the nurse's movements as they bathe and feed other
infants. She watches how they pin diapers and is sure she can do just as
well.

Back in her room, she waters the geranium. Water leaks through the clay
pot, onto the congratulatory cards. The card from the office is on top;
the signatures smear as Juanita blots them with tissue. Their card is
the kind with a chubby baby smiling, pinks and whites blending in the
pastel borders. At the bottom, they tell her to hurry back.

Juanita reaches for the ringing telephone, knocking over a carnation
arrangement. The white mild glass breaks on the carpeted floor. Storks
with pipe cleaner legs and babies in yellow blankets are lost under
carnations and glass.

Yes, her boss says, they miss her. The new girl puts figures in the
wrong column and flirts with the married men. Not to rush Juanita, but
if she finds she can come back in a month instead of six weeks\ldots No
need to decide now.

The baby is fine, she is fine, everyone is fine, and maybe in a
month\ldots{}

A nurse bustles in. She takes a white cup from her tray and rattles the
pills against the paper sides. ``Hang up, dear,'' she says. ``Time for
our medicine.''

The nurse gives Juanita pills and water, and turns, almost tripping over
the aide who is sweep­ing the glass and carnations into a dustpan. The
nurse swivels on her foot and says to Juanita, ``We really must learn to
be more careful with a baby in the house.''

The nurse closes the door quietly. The aide leaves without looking at
Juanita.

The same nurse brings Amanda for her evening feeding, checks the
mother's and baby's bracelets. She tucks in the neckline lace on
Juanita's nightgown and says, ``We don't want to irritate the baby's
skin, do we?''

In her sleep, Amanda wrinkles her forehead. Juanita watches. Without
warning, Amanda opens her eyes and cries. Juanita sticks the bottle in
her mouth, rocks her back and forth. But the baby hits at the bottle
with her fists, drinks, instead, her own tears.

``Hush now,'' Juanita says. ``Pretty baby sleep, sleep.''

Holding the baby against her shoulder she sings, ``Mama's going to buy
you\ldots'' The baby's tears run down her neck, between her breasts. The
wails are trapped in her ear canal, bounce against the bones.

``Stop it.'' Juanita shakes her and tries to force the rubber nipple
into Amanda's mouth.

The toothless gums set against her. She cannot pry them apart with her
fingers.

Juanita rings for the nurse, again and again, until the nurse comes
running, her crepe soles silent against the carpet.

In the nurse's arms, Amanda burps and is silent. The nurse says nothing
as she walks out.

Juanita bites through the cotton pillowcase, sucks the plastic covering
until pine Lysol stings her tongue. The roof of her mouth is clean. She
sleeps with her hands tucked under her chin and dreams of spruce
forests, of pine cones nestled in branches.

At home, Juanita is alone with her daughter. No nurses with sterile
hands stand waiting behind glass. Richard at work all day, into the
evening. Later every night. Friends at their desk or at home cooking for
their own families. Her mother at the hardware store, counting out nails
for men. Only herself in the mirror every morning, her skin parched,
peeling around the lips.

She vacuums to drown Amanda's cries, watches soap operas on television.
Sometimes she rocks Amanda in the darkened bedroom and cries. She tries
to sleep with Amanda next to her in the double bed, but the baby kicks
against her, keeps her awake.

Richard says Juanita is too tense; the baby feels it. ``Girls cry more
than boys anyway,'' he says one night at the differ table. He takes
Amanda for walks in the buggy while Juanita sleeps.

Her mother calls and says it's colic; Juanita was the same way as a
baby. ``Is her stool runny?'' Mother asks Richard over the phone.

Richard relays the question to Juanita.

``I don't know,'' Juanita shouts over the television voices.

``Yes, it's runny,'' Richard says into the phone.

Richard goes to the drug store and buys a case of Enamel. ``Mother says
this is better for colicky babies,'' he tells Juanita when he gets home.

The doctor tells Juanita that most formulas are the same, says Amanda is
a normal, healthy baby. ``You just spoil her,'' he says. ``Let her cry.
She'll stop eventually.''

That night, Juanita calls the doctor's home number.

``Dr. Feldon speaking.'' A calm voice, warmer than his office voice.

Juanita lays the receiver in the crib net to Amanda's mouth. The cries
grow louder. She waits one minute exactly and picks up the receiver. It
is dead.

When Amanda cries again that night, Juanita dials the doctor's number
and listens for his voice before putting the receiver near Amanda's
mouth. On the third call, a distant voice identifies itself as Dr.
Feldon, tells the caller to leave a message when the beep sounds. Amanda
cries through the entire tape.

Exhausted, Amanda finally sleeps. Juanita falls asleep on the couch
watching the late movie. Sunlight wakens her and she puts her hand over
her eyes, pulls up the blanket that Richard put over her before he left
for work. The television is off and everything is quiet.

Juanita jumps up and stumbles on the edge of the blanket. She tiptoes to
Amanda's room, stares at her daughter whose face is already filling out.
She stands there until Amanda whimpers in her sleep.

By the time Juanita warms the bottle, Amanda is crying. Juanita changes
her diaper and wraps her in a blanket. She rocks Amanda and tests the
milk against her forearm before putting the nipple against the baby's
lips. Amanda sucks. Suddenly, she chokes on the steady stream. Juanita
lays her across her knees and pats her back gently.

Again she holds the bottle to Amanda's lips. Amanda will not drink more.

``An ounce isn't enough. Come on. Just a little more. `` She squeezes
the nipple so milk beads on the baby's lips.

Amanda turns her head. Milk drips onto Juanita's arm, sticky against her
sweating skin. Juanita wipes her forehead with the back of her hand, all
the time making clucking sounds with her tongue, coaxing Amanda to
drink.

Amanda's cries grow louder. Juanita hurls the bottle against the wall.
The plastic sac breaks and the milk drips down the blue walls, behind
the framed print of the little drummer boy.

``Stupid baby!'' she screams. ``Why can't you act right?''

The muscles in Juanita's arm twitch as she lowers Amanda into her crib.
She runs from the room, slams the door. The wood is cool against her
forehead.

``Dumb, ugly, little baby.'' She pounds the door with her fists. Pounds
until her fists hurt. ``Why don't you leave me alone?''

Juanita runs cold water over her bruised flesh. ``I'm leaving until you
stop crying.''

Juanita pours a glass of orange juice and picks up the morning paper
from the breakfast table where Richard always leaves it for her to read.
She shuts the heavy front door behind her and sits on the top porch
step, the concrete warm through her shorts. It's the first time she's
read the paper since she's been home. By the time she reads the comics,
she is laughing.

As much as she hates to admit it, her mother and Richard are right. She
needs to get out for a while, talk to other people.

She opens the front door and is surprised at the quiet. A good sign, she
decides as she puts on her makeup. She takes her time and blends the
rouge, wipes away the mascara smudges.

Even when she opens the bedroom door, Amanda does not wake up. She stays
asleep when Juanita puts her in the buggy, stays asleep as the buggy
bounces down the curbs and up.

Two mothers are sitting on the park bench rocking buggies. The first
mother wears a peasant blouse with intricate embroidery on the front and
on the sleeves. The second mother wears a halter top and moves her
position occasionally to catch the rays of the sun on her shoulders.
They are talking about a rash on the second mother's baby.

Juanita aligns her buggy with theirs and sits down.

``Look like diaper rash,'' the first mother says.

``Can't be. I've tried all the ointments on it.''

``What do you think?'' The first mother asks Juanita. She reaches into
the other buggy and points to a rash on the inside of the baby's thighs.

Juanita leans over and looks into the buggy where the baby makes soft
noises sucking at the nylon blanket binding.

``Diaper rash,'' she agrees.

``Try baking soda,'' the first mother says. Then she looks over her
shoulder at a small towhead on the monkey bars. ``Be careful!'' she
shouts.

Amanda wakes up and begins to cry. Juanita rocks the buggy harder. She
picks up Amanda and rocks her against her shoulder, offers her the
bottle.

``Let Ruthie try,'' the second mother says. ``She has a way with
babies.''

Juanita hands over her baby, watches her baby snuggle into Ruthie's
shoulder.

``Your first?'' Ruthie asks. ``My fifth. A boy,'' she says, pointing to
the baby asleep in the buggy. His light hair stands on end and his fair
skin is flushed. The blue stretch pajamas pull tight across his fat
belly. ``Girls are tougher,'' she reassures Juanita.

An ice cream truck rings its bell on the other side of the playground.
Three children jump from the sandbox where they've been digging
underground roads and run to the park bench.

``You promised,'' the towhead says to Ruthie. ``You promised.'' All
three children nod their heads.

Ruthie puts Amanda back in the buggy and winks at the second mother.
``We promised.'' To Juanita, she says, ``Do you mind watching the babies
while we get the kids ice cream?''

``Course not,'' Juanita says.

``Bring you anything?'' Ruthie asks, trying the red ribbon at her
neckline.

Juanita shakes her head and Ruthie laughs. ``Good thing. Probably melt
before we got back here.''

Juanita watches the mothers and children walk under the slide. She
brushes a fly from Ruthie's baby. While she's up, she wheels Amanda's
buggy next to the boy's.

Juanita does not think about it, just wheels the boy's buggy along the
path. She pushes faster until the path curves in to the center of the
park. She pushes over bicycle ruts and sings, ``Mama's going to buy you
a mockingbird.''

Concrete dolphins rise out of the fountain ahead of her. Water streams
from their mouths.

``And if that mockingbird don't sing,'' she hits the right notes, holds
them.

Juanita leans over the concrete edge and washes her hands in the water.
She rubs a moist finger along her lips.

She pulls the buggy back from the fountain, sits on the cement bench
next to an old man.

``Fine day,'' he says.

``Yes, a fine day.''

Your baby is sure enjoying it.'' The old man looks into the buggy.

``He's a good baby. Sleeps all night ad eats good.'' Juanita takes a
handful of popcorn form the box the old man offers.

``Reminds me of my son when he was a baby.'' The old man reaches into
his pocket and pulls out a wallet. Behind the scratched, plastic frame,
a blonde, middle-aged man smiles.

Juanita has no pictures to show. In the distance, she hears the ice
cream bell.


\cleardoublepage
\chapter{In Winter}

All day the words drift in Jeannie's ears, melt now as she lies in her
single bed, her grand­mother's crazy quilt holding down her body. The
words, like her legs, cramp in the winter heat.

Up through the vents come her mother's sounds: quiet, brittle. ``You
keep her,'' she says to Jeannie's father. ``She always loved you best.''

The furnace clicks on. Hot air dry against her eyes, the back of her
throat.

``A girl needs her mother. What can I say to a sixteen-year-old?'' Her
father's voice hovers near the ceiling, then curls around the lipstick
tubes and rouge she's worn since her thirteenth birthday.

He gave her the tube of strawberry kisses two years ago, after she'd
blown out the candles. From his coat pocket, not even wrapped.

Jeannie opens her window. Snow, cold under her fingernails, piled in her
hands. Still more clinging to the outside ledge.

Her mother's voice rises as she tells him the whole neighborhood knows
how well he handles the young ones. He tries to shush her.

Small snowballs, the size of pearls, lined across the window sill. One
by one, Jeannie flicks them off, packs the remaining snow into a ball.

In the box of photographs in the basement, her father puts his arm
around her mother: at the beach last summer, at the New Year's Eve party
after he set the crepe-paper tiara on her head. Before that even.
Jeannie, at two, held between them.

Silence. Her father may be holding her mother now, kissing her lips to
stop the words.

Kneeling at the vent, Jeannie pushes pieces of her snowball through the
metal slats. Snow water over her parents' heads, refreezing at their
feet. They cannot move.

Her father says he won't be home much when he finds an apartment.
``That's no life for a young girl.''

Her mother again, on cue, says it's no life for an old girl either being
suddenly split up like this.

The snow hand touches Jeannie's forehead, slips to her mouth. Muffled
shouts in her cupped hand. ``Never'' ricochets off her palm, onto her
tongue: a morning taste.

``What will I tell Jeannie?''

``Think about it.'' Her father closes the front door.

The car starts in the driveway. Grinding gears from park to drive. A
thump as metal hits the snowbank. Forgetting, he honks his standard
good-bye from the bottom of the driveway.

Has he already found a new place? Or will he come back later to sleep on
the tweed sofa that pulls into a bed? If her father chooses the sofa,
she can't have Cindy sleep over next week.

House sounds: television dialogue, canned laughter, the ironing board
click, the steam iron hiss. Her mother is probably ironing her father's
handkerchiefs, even his underwear, just as she's done for years. Take
the pale blue shirts and iron permanent wrinkles into the collar. Rip
off the buttons.

Turning her face toward the window, Jeannie sleeps, dreams the words are
ironed flat on stone tablets. Theme music: the clock clicking off each
minute, whining toward the next. Later, her mother's flesh against her
forehead; the dampness of lips and lotion hands, her mother leaning over
her, dim in the night light, mixing with the dream shadows.

In the morning, Jeannie's jaw is sore; she's a teeth grinder. Her
dentist suggests braces. Her mother claims it's nervous tension and
urges her drink warm milk before bed. Last night, her mother forgot to
heat the milk; this morning her mother sleeps late.

Jeannie sits alone at the kitchen table, watches out the window for
Gary's car. He's late and her feet are getting warm in the fur-lined
boots.

A horn honks. She runs, clutching her books to her chest. Halfway down
the walk, she turns and goes back to close the storm door. No one to
yell after her that she should wear her hat.

Like her father, Gary's at the wheel, tapping his fingers on the
dashboard. His head bobs to a tune Jeannie can't hear. His mouth opens
and closes, forms a circle to draw out the end of the song.

``If I'm late for that gym class one more time, I've had it,'' Gary says
as he throws the car into drive before Jeannie even has a chance to
close the door.

Jeannie rubs her jaw and straightens the books in her lap. At a
stoplight, Gary leans over and kisses her. She tells him she doesn't
like to be kissed at 7:30 a.m. She scratches a snowman into the frosted
window as Gary calls her a grouch, asks her what he did to deserve this.
Teasing, he sticks out his lower lip, waits for her to laugh. She does
as she gets out and locks her side.

Jeannie feels the cold between her legs, wishes she'd worn slacks. Her
boots are really no pro­tection against the snow drifts that lap over
the fur tops, melt against the soles of her feet.

In chemistry, she wonders which chemicals are responsible for sexual
attraction, vows to exper­iment with love potion formulas. In gym, the
volleyball reminds her of the back of her father's head. She vows to
serve with her fist when it's her turn. At lunch, she dumps her peanut
butter sandwich into a garbage can and starts a diet. By study hall,
she's light-headed and anxious to talk to Cindy.

They've been friends since grade school. Cindy has an older sister who
told her things which she told Jeannie; things about boys and how to
kiss with her mouth open. It was Cindy who went with her to buy her
first bra, talked her into a padded one trimmed with lace. Together they
watched the Girl Scout movie, ``You're a Woman Now'' and giggled. Cindy
already knew all about the blood, knew how to use it to get out of
playing softball. All she had to do was whisper to Mr. Whitlock, the gym
teacher, that it was ``that time'' and he'd let her sit under the tree
and watch the rest of the class play. It was Cindy she told when a boy
hit her in the chest and she was afraid she'd get cancer.

Cindy confided in her, the secrets girls can't tell their mothers, the
secrets mothers give away to fathers late at night.

In study hall, Mrs. Abbess sits behind the desk, watching, pushing the
sharpened pencil in and out of her tight braids. Sometimes she scratches
her scalp and examines the pencil point. Now she waves the pencil in
Jeannie's direction, motioning for her to get back to studying.

Jeannie drops her eyes and rereads the note she's written to Cindy;
really a letter---started in chemistry, added to in English, and just
finished. Nothing melodramatic. Her parents are getting a divorce. Does
Cindy think her dad is really seeing someone younger? How much younger?
Will she have to talk in court? Her mother threw the whole roast down
the garbage disposal last night, even the mashed potatoes. She ate Ritz
crackers and peanut butter. Did they turn off the heat in the study
hall?

When Mrs. Abbess starts sneezing and rummaging in her purse for tissue.
Jeannie gives the note to the girl in the next desk and nods towards
Cindy in the back. The note is passed.

Jeannie glances back and sees Gary reading the note over Cindy's
shoulder. He thinks it's about him.

``Cindy,'' a hissing through her teeth. Her jaw is still sore, the bone
hurts way back into her ear.

``Cindy,'' Louder this time.

Mrs. Abbess looks at her, looks at Cindy, at Gary, his chin almost
resting on Cindy's shoulder using the palms of both hands. She pushes
the chair from the desk, rises slowly.

``Cindy,'' Jeannie whispers.

Cindy watches Mrs. Abbess moving toward her. Slips Jeannie's note into
her English book. Pulls out her math assignment. Gary leans back in his
seat. Jeannie grips her pen like a child learning to write, digs her
fingernails into her palm.

``The note, please,'' Mrs. Abbess says, extending her arm.

Cindy explains she was doing math problems. Gary was helping her. She
puts the math paper into the outstretched hand. It slides off the
fingers, onto the floor. No one picks it up.

``The note, please,'' she says again, taking the pencil from her English
book, lays it on the desk.

``In my hand, please,'' says Mrs. Abbess. She takes the note Cindy hands
her and holds it between two fingers, waves it above Cindy's head, both
sides visible, filled with writing.

``As you students know, to reach such notes.''

Jeannie stands.

``Sit down. Perhaps you'll think twice about writing notes next time.''

Mrs. Abbess strokes the black velvet ribbon that holds her glasses
around her neck. Slowly, she lifts the glasses from her chest, places
the wire ends around each ear. She fingers the nose piece into place,
drops her eyes to read.

Jeannie cuts across the room, slipping through the spaces, pushing desks
aside when she must.

``Sit down, Jeannie.'' Mrs. Abbess doesn't move.

Sounds: The wind pushes snowflakes through the cracks. Mrs. Abbess' cold
rattling in her chest. Gary's crepe-soled boots against the wood. The
paper rattle as Jeannie grabs the note.

Pictures: Jeannie's hand raised, still holding the pen. Mrs. Abbess,
tripping, falling backward into Gary's arms. Cindy standing. The whole
study hall moving: red plaid shirts tucked into blue, pink sweaters and
yellow, even orange.

Someone calls her mother who meets her in the counselor's office,
embraces her. Now she sits stroking Jeannie's hand as the counselor
talks. Neither can understand. No previous blots on Jeannie's records.

The counselor talks of a note which Jeannie ripped up on the way to his
office, threw into the air like confetti. ``Mrs. Abbess was still
picking pieces of it out of her hair when she got here.''

The counselor looks at Jeannie's mother and offers her a cup of tea. She
shakes her head, starts crying. The counselor offers her a tissue,
pushes the box to the edge of the desk where she can reach them easily.

Her mother wipes her eyes, examines the tissue for traces of mascara.

After lining up the pencils on his desk, the counselor continues, ``Poor
Mrs. Abbess was so upset I sent her home for the rest of the day.''

Jeannie bites off a broken nail, wonders if Mr. Jenson will let her make
up the history test she's missing.

``Actually had her pen up, ready to strike,'' the counselor says.
``Vicious attack.''

Her mother is crying again and the tissues pile up in her lap.
``Everything at once,'' her mother says. ``Too much for her, for me.''
She tells the counselor about the divorce, about the girl. ``She's
probably not much older than Jeannie.'' Her mother's cheeks are flushed.

Using her foot, Jeannie moves the wastebasket by the counselor's desk,
pushes it toward her mother so she can get rid of the white mound in her
lap.

The counselor rolls the pencils across his yellow pad. Jeannie knows
he's dying to write it all down, stick it in her record. She asks the
counselor if the time she locked the teacher in cloakroom is in her
record.

``That was an accident,'' her mother says. ``You were only in first
grade.''

Her mother promises to look after Jeannie, to make it all up to her. She
agrees to take her for a physical exam, even to consider some outside
counseling. Yes, she'll keep her out of school the rest of the week and
yes, things do usually work out for the best. She smiles at the
counselor as she helps Jeannie into her coat and buttons it.

On the way home, the car stalls. Her mother tries to start it by holding
the accelerator to the floor even though Jeannie warns her she'll flood
the motor. Her mother folds her arms over the steering wheel, drops her
head, and cries.

Jeannie gets out her side, walks around the car, stopping to knock the
icicle of the bumper. She opens her mother's door and nudges her over.
Sitting in the driver's seat, she waits a few minutes, then turns the
key. The engine catches.

The rest of the way home, her mother talks quietly, explains that it's
her father's fault, deserting them both like this when they need him
most.

``When you were a baby,'' she says, ``I was the one who warmed your
bottles and changed your diapers.'' her mother takes one of the
counselor's tissues from her pocket and blows her nose.

Her mother's eyes grow brighter as she tells about his nights out with
the boys while she kept things going at home. The cooking, the
ironing---she did it all. And now what has she got? Nothing. That's for
sure. Not even her looks.

``Don't trust men.'' her mother says, brushing a strand of Jeannie's
hair behind her ear.

It's the first time her mother has spoken to her about men. Jeannie
turns the windshield wipers against the snow melting on the warm
windshield.

Who will do her father's cooking and ironing? Perhaps on weekends, she
will clean up his place, cook meals. The first weekend, she will
probably cook chicken. That's his favorite. She can even ask her mother
to teach her to make chicken gravy.

When they get home, her mother pulls the shades and makes her go to bed.
She pulls the blankets to Jeannie's chin and tucks them in at the sides.

The time Jeannie had pneumonia, her mother sat up all night sponging her
head with water, tucking in the blankets. When Jeannie started feeling
better, her mother made her a nurse's cap out of starched white cotton.

The vacuum cleaner runs in and out of her dreams; pots and pans bang
against the burners. Later, voices talk about her in tones and pitches
she recognizes. Her mother's voice. Her father's voice. Both droning on
and on.

There are words and more words by the two people Jeannie has brought
back together. She wishes they would both suffocate in all those
snowflakes so she could push in raisin mouths and carrot noses.


\cleardoublepage
\chapter{A Job}

Up until now, I've had it made. Things were easy. Don't get me wrong. I
didn't go around admitting it to just anyone.

But it's like I told Mary Jane when we decorated the basement for my
sixteenth birthday party a couple of months ago---I'm a lucky kind of
person.

My parents aren't divorced like Mary Jane's and they don't yell all the
time like Sara's. They buy me pretty much what I want except for crazy
stuff like the trip to Hawaii I asked for last year. They even like
Bill, my boyfriend. The only thing they really get on me for is not
cleaning my room and sleeping late on weekends.

But that's all changed. Now they're on me all the time. Every time I
turn around they're after me to get a job. It's getting downright
embarrassing.

At lunch today, I slide in next to Mary Jane. The other girls we hang
around with dash in just before the bell rings. We're all talking about
the good-looking guy in our English class who just moved here from
California.

I split my lunch bag down the middle like I always do and keep on
talking. Suddenly they're all looking.

``Taken to eating paper?'' Mary Jane jokes, pointing at my lunch bag.
``Guess that'll help you keep your diet.''

Folded between my apple and granola bar is a sheet of newspaper. I
groan, ``Probably an Ann Landers column.''

Mom's a big fan of Ann Landers. Since I was in junior high, she's been
cutting out columns and sticking them on the refrigerator door so I'd
read them. Stuff about dating, drugs and treating your parents right.
But this is going too far.

Sara snatches the paper and unfolds it. ``Aha. It's even better than Ann
Landers,'' Sara says. ``Here you go. Part time cashier at Sherman Drugs.
Wait, here's one that's even better. Responsible student to assist in
after-school playground supervision on Mondays and Wednesdays at Emerson
Elementary.''

She turns the paper around for everyone to see the ads circled in red.
``Plenty here for all of us,'' she giggles.

``Oh, no,'' Mary Jane says. ``We edit the Clarion on Mondays. You can't
just quit on us.'' She knows my mom has stepped up the effort to get me
working.

``I'm not getting a job,'' I assure her. ``No way. I mean it's not like
I have to work.''

``Bet you could get a nerd job like Cynthia's at McDonald's,'' Kim says.
``Then we could all drop in to see you Friday and Saturday nights.''

``That's not fair,'' I say. ``You know Cynthia has to work. Her dad's
still laid off.''

``She's a nerd, anyway,'' Kim says. ``No fun at all. She doesn't even
know anything that's going on.''

I look at Jill who's just leaning back in her chair the way she usually
does. She has that cool, sophisticated look that's beginning to get on
my nerves.

Jill leans forward and wipes the table before resting her silk-covered
arms. ``Do you think,'' she says, ``that Bill's going to just wait
around for you to get off work on weekends?''

I'd like to wipe that superior smile right off her face. Instead, I just
sit there seething at my mother, taking big bites out of my apple. The
sound of my teeth crunching through the red skin fills my head.

When I get home from school, I'm glad my mom's upstairs vacuuming. I
flop on the floor in front of the television to watch my favorite soap.
I turn it up to block out the vacuum overhead.

I'm caught up in the story when Mom just walks right up and turns the
television off.

``Hey,'' I holler, ``it's not over yet.''

Mom just stands in front of the screen, a dust cloth in her hand.

``Well?'' she asks.

``Well, what?'' I say. I know what she means and she knows that I know.

``Looks like some good job openings for you. Which ones are you going to
apply for?''

``None,'' I tell her. ``None of my friends work. Besides, I don't have
time for a job.''

Mom just looks at the television and shakes her head. ``You've got
time,'' she says in her matter-of-fact voice that warns there's no room
for argument.

I just pick up my school books and go upstairs to study until dinner.

At dinner, it's Dad's turn.

``I was telling a guy at work today how you're looking for a job,'' he
says. ``The guy's wife has this business where they call people on the
phone and try to sell magazine subscriptions.''

He asks my mother to pass him another roll. As he butters the roll, he
watches me, waits.

``I told him you're great on the phone, that you were born with one
glued to your ear,'' he laughs. ``Anyway, here's the number. His wife's
hiring some new people next week.''

``Sounds good,'' Mom says as Dad takes a slip of paper from his pocket
and hands it to me.

``I wouldn't be any good at selling people stuff,'' I say, wishing
they'd both just change the subject.

But my dad rattles on, tells me there's no need to be nervous. Adds that
he guesses everyone gets a bit scared about their first job.

But I can't stand it any longer. I ask them both to just stop talking
about it. I tell them I'm not afraid of getting a job, not afraid of
working. It's just that I need to work extra hard at school. The
competition to get into a good college is getting tougher all the time.

``There's more to learning than going to school,'' my dad says. I know
I'm in for one of his when-I-was-your-age stories.

Sure enough. I hear again about how he helped out on the farm when he
was big enough to walk. How he made extra money sweeping out the
hardware store when he was only ten. As if every kid should have to do
that.

``Times have changed,'' I tell him. ``And none of my friends have to
work.''

``That's just it,'' he says. ``They don't know anything about real life,
about what really matters. I want you to be different.''

``I don't want to be different,'' I scream and run up to my room,
turning up the radio full blast when I get there. I don't want to hear
them whispering about me.

They both come up later and tap at my door. Both stand in the middle of
my room. Mom hands me a piece of my favorite dessert, apple pie still
warm from the oven. I set it on my dresser.

``We know it's a change for you,'' Dad says. ``And we know you don't
want to do it. But it's time. We've decided to give you a month to find
something.''

My mother nods, straightens the hair brushes on my dresser. ``You'll
find something you like. And we'll be glad to help.''

``I don't need your help,'' I shout as they both leave, quietly shutting
the door behind them.

It's like I tell Mary Jane. It's tough to figure what makes your own
parents turn on you. But they all do it sometime or another. They get
that smug look on their faces and tell you it's for your own good.

At times like this, I wish I were Jill in her silk blouses and cashmere
sweaters. Her parents never make her do anything. Not even clean her
room. They have a maid who does all that stuff.

When I call Mary Jane, she's in the shower. But she calls me back in
five minutes.

I pour it all out to her and she listens. But then she laughs, ``It's
not all that bad. You'll have extra money and you might even meet some
cute guys.''

``What do you know about it?'' I hiss through clenched teeth. If even
your best friend betrays you, what's left?

Mary Jane is quiet for a minute, then says, ``Since my dad moved out,
there's only one thing my parents agree on. And that's that I'm too
young to get a job.''

Mary Jane never told me she wanted to work. I mean if you don't tell
your best friend some­thing, how important can it be?

``You're just saying that to make me feel better,'' I say.

``No,'' Mary Jane says, ``I really want a job.''

``You're crazy,'' I tell her. ``You don't know when you've got it
made.''

In bed that night, I plot it out. I know just the place to go. I'll show
them all. I'll get a job I hate. They'll see what happens.

The next day after school, I walk over to the Summerdale Convalescent
Home. It always depresses me. Every Saturday when I used to practice for
cross-country I had to watch all those old people being pushed around in
their wheelchairs. Sometimes, someone would even push them over to the
fence to watch us practice. I don't ever want to get that old.

The woman behind the desk is bright and cheerful, even though some of
the old people are sitting right there in the lobby playing cards.

I know they're watching me as I talk to the woman at the desk. She calls
the manager who takes me into an office to talk to me.

She says the job involves taking care of a group of senior citizens.
Says it's a tough job and not always pleasant. Says I'd even have to
empty bed pans.

It's just what I'm looking for. The perfect way to make my parents
sorry. So I respond to the manager's questions on cue. After all, I've
been in a couple of school plays and know how to act.

My parents are thrilled when I get the job, even tell me how proud they
are. When my mom hugs me, I do feel a little guilty but not much.

I start on a Monday, after school. As soon as I walk into the lobby, my
stomach turns over. Maybe I am a little nervous. Mostly, it's the smell.
Medicine and alcohol and maybe even death. It's the same smell hospitals
have.

The nurse on duty takes me around and introduces me to everyone, shows
me what I'm sup­posed to do.

Most of the old people are pretty nice but some are real cranky. Mrs.
Williamson, for exam­ple, yells at me all the time. Tells me I remind
her of her daughter who sends money but never comes to visit. Complains
that Mr. Bellingham cheats at cards. Whispers that Mr. Bellingham and
that new woman down at the end of the hall have something going. Says
she's seen them kissing and holding hands. I'm shocked.

``Now, dear,'' Mrs. Williamson says, holding my arm tightly, ``what
makes you think we're any different than you.''

I try to avoid Mrs. Williamson as much as I can. It's not easy, though.

Mrs. Williamson holds court in her room. The men and women from down the
hall and from other floors congregate there. She used to be a high
school English teacher and lectures them about the symbolism of {Moby
Dick} and the dark side of Nathaniel Hawthorne. The old people seem to
like it. Guess it reminds them of when they were young.

I joke about Mrs. Williamson and some of the others with my parents.
They listen and their eyes kind of mist over. I can't ever imagine my
parents getting that old.

One Saturday, Mrs. Williamson asks me to take her over to watch the guys
practice football. Everyone thinks it's strange because she hates to be
seen in a wheelchair.

As she once explained to me, ``I've got my pride, dear. Wheelchairs are
too much like baby strollers.''

But this Saturday, she's all dressed and even put rouge on her cheeks.

I wheel her around by the bleachers and sit on the third bleacher up, my
head level with hers.

One of the guys recognizes me and waves. I'm humiliated.

During the break, he runs over to talk to us. ``Two cheerleaders,'' he
says. ``Great. We need all the encouragement we get.''

Mrs. Williamson leans over to me when play resumes and says, ``Let's not
just sit here like two bumps on a log.''

She cups her hands around her mouth and shouts, ``Hip, hip, hurray. Put
`em away.''

The team salutes us and I wish I could just melt into the ground.

``Boom, boom, bah. Hit `em in the craw,'' she shouts louder, nudging me.

Pretty soon she has me shouting with her and we're giggling like a
couple of kids.

On Monday, the football player asks me who the old woman is.

I tell him she's not an old woman, she's Mrs. Williamson. And she thinks
he's kind of cute.

He blushes and says he'll see us both next Saturday.


\cleardoublepage
\chapter{Library Card in My Back Pocket}

When I was a little kid, my mom used to walk a mile with me every week
to the Jessie Chase Branch Library in Northwest Detroit. I sat at the
children's table thumbing through the picture books, deciding which I
would check out. At age ten my blue two-wheeler bike was my
transportation. Pulling out my library card made me feel like ``big
stuff'' since a mere flash of it led me into adventures with Heidi, the
Bobbsey Twins and Nancy Drew mysteries. I carried home whole new worlds
in the wicker basket that hung from my handlebars.

I promised myself that someday I would read every book, even those in
the adult section. I learned how to look up answers on my own, how to
use words, how to tell stories, how to expand my neighborhood by
crossing oceans and climbing mountains in my mind. The librarians
discussed books with me as if I were a grownup. They introduced me to
Black Beauty and, in my imagination, I jumped onto that horse and raced
the wind. I read everywhere: under the shade of the locust tree in our
backyard, at recess in school, under the covers at night, waiting for my
parents to get ready for church. I was that girl with skinned knees, a
ponytail, and a library card in her back pocket. On my twelfth birthday,
a cold day in November, my parents took me to the Main Branch of the
Detroit Public Library. It reminded me of castles I read about with its
big pillars and massive size. My neck hurt from looking up at the
ceiling. My shoes slid across the marble as I headed for the ornate
steps. I found room after room of books and people of all ages and types
reading at tables. We found the young adult section and I had never seen
so many books in one place. I stacked books all around me at the table
and flipped through them. I didn't know which to read first.

I collected as many books as I could carry and went to the checkout
desk.

My parents smiled at me as I pulled out my crumpled library card and
presented it with a flourish to the librarian. ``Ah,'' she said as she
stamped return dates in each book, ``I see your magic passport is well
used.''

``I work at the library at my school,'' I proudly told the librarian as
I cradled the books and tried to keep the top one from slipping.

Long before college, my first library card had worn thin. It finally
shredded in the jeans pocket during a wash cycle. Luckily a library does
not discriminate against anyone, even those who are care­less like me. I
received another card and I never left home without it.

The Main Library became my second home when I attended Wayne State
University. Many days after class, I curled up in an upholstered chair
in one of quiet nooks. There, I soaked up \emph{Leaves of Grass, Moby
Dick, Hamlet, Native Son, The Great Gatsby.} By then, I knew I could not
read fast enough to devour all the books in the library. Still, I could
not stop turning pages. From time to time, I looked down over the
railing to the main floor. I learned I couldn't identify serious readers
based on appearance. Reading seems to be one of those equal opportunity
activities that lure all ages, all ethnic groups, all religions,
everyone in the community. I watched so many people stream in and out of
the library and wondered what they had chosen to read, wondered if maybe
I had read the same book or should read it.

When I became an English teacher at a Detroit high school, I gave my
students extra credit for visiting the Main Library. I believed no one
could pass through those doors without succumbing to the lure of books.
Such a magnificent library breathes the history, the humanity, the
individuality of each person who dares enter.



\cleardoublepage
\chapter{A Lot She Knows}

The family said Irma should be the one. They argued back and forth on
the phone late at night and on the weekends when long distance calls
were cheapest. They talked about it when they got together for Sunday
dinner. They whispered about it in bed with their husbands.

After all, it was only fair, they said, that Irma should be the one. She
had no husband to care for (having been divorced ten years ago), no
children at home (Elaina, her youngest, was at Upper State College), no
job worth worrying about (being only a clerk in K Mart's wig
department), and in short, they said, she had always been Mama's
favorite.

Irma protested. She went to Sunday dinner, although she wasn't invited,
and told them she was willing to chip in for a nurse. She reminded them
rather loudly, that Mama took them to the Waldmere County Fair and left
her home to fix dinner. (``Mama always thought you could cook best,''
Ruthie and Antoinette said.) Mama called the doctor when they got
measles, and when Irma got them, Mama gave her what was left of their
medicine. (``Mama always thought you were the strongest,'' Ruthie and
Antoinette said.) Mama made her sleep in the attic while they had a room
together. (``She always said you were the smart one and needed time to
yourself,'' Ruthie and Antoinette said.)

But in the end, it was Irma who took the Greyhound bus to Hazen, Ohio
and promised to write every day to let Ruthie and Antoinette know how
Mom was getting along.

Once there, Irma straightened her long-haired wig and picked up her
suitcase, still not quite ready to walk up the hill to Mama's house.
Instead, she turned and walked across the street to the Soda Stop. The
red sign was freshly painted and clay pots of marigolds were on each
side of the door. It had always been like this, never changing. Old man
Warner insisted the sign be painted every spring before the first heat
bubble showed. When Irma pushed open the heavy glass door, she heard the
familiar sleigh bells announce her arrival.

Old man Warner laid down the \emph{Hazen Record} and pushed his bifocals
to the top of his head. ``My far sight's still as good as when I used to
go crow shooting. And I do believe it's Irma who's finally come back.''

He went behind the counter and scooped vanilla ice cream into a frosted
metal container. Next, blueberries followed by milk, mixed on the
machine and, ``The Blue Bomber Special for you. Don't nobody ask for it
much anymore.''

Irma had named the football captain the ``Blue Bomber'' when he made the
winning tackle against County West. Then she'd gone to work and
concocted the Blue Bomber Special which she served him and a
cheerleader. After that, she toasted him silently with the drink and
picked bits of blueberry skin from between her teeth.

Old man Warner leaned over the counter and watched her drink. ``How's
your Ma doing?''

``Okay, I guess.''

``If your sisters was home, they'd be down here to fetch you.''

Irma wondered how many times she'd read her school books at the corner
table after she was off work or sat whispering to one friend or another
about the Blue Bomber. It was at that table that old man Warner
convinced her to take the scholarship to Michigan State, where she
dropped out the first semester to get married. Mama had never forgiven
him for that.

``When you get settled in, come back on down and I'll show you our new
bank. Even has those new cameras in it. A subdivision is going in behind
Lankford's farm and we got us a new schoolhouse. Lots has changed,'' he
said, straightening the boxes of licorice whips and jawbreakers. ``Maybe
enough that you'll stay this time.''

``Nothing changes that much,'' Irma said, wiping her mouth with a paper
napkin.

Irma walked the path in the woods where she and her sisters had buried
their pet duck after it died. She looked for the crooked tree she used
to sit in to read or write in her diary. She looked for the crooked tree
she used to sit in to read or write in her diary. She couldn't find it.
The path was grown over with wild ivy and Queen Anne's Lace. The
dandelions had turned to white puffs, and , every so often, she stopped
to make a wish and blow the white seeds. Irma guessed that no one walked
the path much to Mama's house anymore.

The county nurse opened the door and led Irma past the bookcase
containing \emph{Reader's Digest} condensed books, into Mama's bedroom.

Mama was sitting up in bed, wearing the quilted pink bedjacket edged
with lace that the three daughters sent when they first heard of her
illness.

``This lace itches my neck. Too still,'' Mama complained, pulling the
lace until it tore in one corner.

``You can tell they're getting better when they start getting cranky,''
the nurse said to Irma, as if Mama weren't in the room.

``Hi, Mama,'' Irma said, kissing her papery lips, fearing they would
come off on her like the tattoos in cereal boxes.

``What took you so long?'' She asked.

``The bus was late, Mama.''

``Don't lie to me Irma Jean. You been talking to old man Warner. Always
listening to him, never caring what I said. Isn't natural.''

``Ruthie and Antoinette send their love. How are you feeling, Mama?''
Irma sat down in the nurse's chair by the foot of the bed.

The nurse answered, ``She's doing fine. It was only a mild stroke. A
couple of months and she'll be good as new.'' She handed Irma a list of
instructions and a phone number, and then left.

Not knowing what else to say to Mama, Irma stared at the faded brown
photographs of her great grandparents framed in golden oak. At the
bottom of the frame, marching in order, are school pictures of Mama's
grandchildren. Small metal Kresge frames for the three daughters. Daddy
serious on the tractor, the same look in a studio portrait, a hand
resting on Mama's shoulder. Mama with Irma and Ruthie, holding
Antoinette in her arms. All jammed on the low chest of drawers by Mama's
bed, edges of frames overlapping bodies blocking each other.

``Good as new,'' Mama snarled. ``A lot she knows about weeding the
garden, with arthritis in my hands.''

``She is a nurse, Mama.''

Irma smoothed the patchwork quilt covering Mama. She'd helped make it.
Bits of matching Christmas dresses with holly sprigs. The old lemon
print apron Mama wore to make wine. Papa's favorite plaid flannel
hunting shirt. Sunday school dresses of pink, yellow, and red calico.
Irma's gradu­ation dress. Squares of years quilted together with white
background fabric. Irma traced the lines, the small, even stitches.

Irma remembered her uneven stitches, Mama yelling at her, showing her
time and again how to gauge the length. She couldn't see the patches
she'd done. Perhaps they were on the other side of the bed, against the
wall.

``What did you do to your hair?'' Mama wanted to know after she'd asked
about her daughters and grandchildren.

``Nothing, I haven't done anything to it.'' Irma wished she hadn't worn
the black wig. Maybe the short blonde one or even the braid, but not the
long, straight black one.

``You always had such pretty hair. So curly and thick. Like mine used to
be.'' Mama stuck another bobby pin into the thin gray hair twisted onto
the top of her head. ``My daddy used to make me braid it so tight my
eyes hurt. Said curls was the devil's tool.''

Mama's childlike voice told of her father, who'd wanted a boy, who'd
been pleased, though, when his only child was big-boned and strong,
strong enough to slaughter chickens, drive the horses, clear the field
of rocks, pitch hay.

``I milked the cows so Mama wouldn't have to. Some said she was a
Jefferson and married beneath her; I always believed that.''

Irma picked up the portrait of her great grandmother and studied her
waist-length, straight hair, moonstone eyes, fragile hands, the dabs of
pink high on the cheekbones, delicately painted in later by the
photographer.

``Mama, are you glad I came?'' But her mother was gently snoring, like a
clock's quiet murmur at midnight.

To help the days pass, Irma cleaned house: polished the nurse's
handprints and coffee stains from the dark walnut table; straightened
drawers where she found her sisters' and her own old report cards tied
in blue yarn, locks of their hair pressed between brown paper, and
Valentine cards they made for Mama in first grade. Irma found years of
her life jammed in cardboard boxes and wooden chests. In a storage
cupboard in the attic, she discovered letters in which her father had
written words she hadn't known he knew, words to Mama at sixteen, at
nineteen. It surprised Irma that her parents had been lovers.

That evening, she took golden brown chicken and biscuits to mama on a
sterling sliver tray that had been a wedding present. She had never seen
mama use it. She put both the Irish linen napkins with their original
folds, which she'd found in the cedar chest.

``You've become a better cook since you left home,'' Mama said. The
color was returning to her cheeks, and she'd gone outside to check her
tomatoes that afternoon.

``I've been cooking a lot since I left home,'' Irma said.

After she scoured the pans, Irma usually read to Mama. She'd read
through all the \emph{True Confessions} the nurse had left, surprised
that Mama hadn't wanted her to throw them out.

Irma removed the pins from Mama's hair and brushed it. Then, she sat at
the low dresser, removed her wig, and brushed her own hair one hundred
strokes, as she'd done every night since she was a child, hoping somehow
the nylon bristles would pull it straight, tame it.

``It's a wig,'' Mama laughed, pushing the quilt from her shoulders,
standing behind Irma in the mirror. ``Let me try it.'' Mama slipped the
wig onto her own head and stood, hand on hip, examining herself. ``Don't
I look like Maybell Stinger?'' Mama asked, as Irma tried to recall who
Maybell was.

It was the way Mama stood with her head thrown back that made Irma
remember Maybell, the woman at Carlton's Hardware. She was the pretty
woman who measured out the nails, who rung up purchases on the cash
register. The one who always gave Irma a red sucker when she went with
Papa to buy chicken feed. She remembered Mama and the other ladies
whispering about Maybell at the church's Saturday Potluck Suppers. Mama
was right. She did look a bit like Maybell with the wig on.

Irma got the other wigs from her suitcase and rejoined Mama at the
mirror. The two gray heads bent close together as they tried on wigs,
struck poses, and tried to guess who they were now.



\cleardoublepage
\chapter{The Mapmaker}

When Rayna Travis Gullavor gets a notion into her head, it sticks like a
grain of sand caught in an oyster shell, a gritty irritant calcifying,
growing, and transforming itself into a hard, round pearl. It nestles
there, her third eye, the eye she can't close.

Rayna sees things. Lately, she feels her watchful eye moving back and
forth, up and down, pupil dilated, open wide, vigilant, waiting for
something big to happen. It scares her, yet excites her, churns up her
stomach as if she's riding the Tilt-a-Whirl at the county fair and can't
get off. She can't tell Eddie, she must not tell Eddie, no matter what.

The little things she sees are piling up. It's okay to tell Eddie about
these. Things like Uncle Henry driving all the way to Cranston, the next
town west, to buy new Jockey bikini shorts, a whole dozen that he kept
in a toolbox in the trunk of his ten-year-old Chevrolet Impala and
washed twice a month at the Easy Ways Laundromat over in Brownstown,
just south of Raintree. Things like the little boy down the street
writing math formulas on the inside of his forearm before the semester
exam, or Jennifer at the Cranberry Café fudging the amount of her tips
at tax time, or even Eddie, her husband, whose leg will end up in a cast
when he slips off the roof next year.

``What did you see today?'' Eddie asks jokingly every night before he
falls asleep.

And Rayna tells him, only him, like she's done since seventh grade.
Funny things, silly things. She sees them before they happen. Things
about people they know, about people who move behind her eyes, who come
and go when they want and refuse to take any direction from her.
Sometimes, for weeks at a time, she sees nothing to tell. Still, Eddie
remembers to ask and his response is always the same, ``Rayna, you're a
woman to be reckoned with.''

``I am, indeed,'' she always replies. And when she gets right down to
it, she thinks of herself as a woman of action, a woman who gets things
done. Lately, she's trying to see her way clear to what she must do.

Like she tells Eddie, she can't help it if she sees things. Is it her
fault that sometimes she doesn't see the whole thing? Besides, how's she
supposed to know when something is just a brick instead of the whole
house? Okay, so Uncle Henry's Jockey bikini shorts were part of his
strategy for getting Jennifer's attention. It worked didn't it? They
used the tip money she didn't give the IRS so both of them could run off
to Miami and fry their flesh in the Florida sun while the rest of us
froze back here in Raintree. It's not as if Aunt Mable cared much about
Uncle Henry's bikini shorts or Jennifer for that matter; she just hated
to miss out on going to Miami.

Rayna wouldn't mind going to Miami herself, with Eddie, of course. She's
spent all 55 years of her life in Raintree, Michigan, right near the tip
of the rabbit's ear. Or, more accurately, within a 33 mile radius of
Raintree since she likes to grocery shop at the Farmer Jack over in
Brownstown (but she always made sure not to go grocery shopping when
Uncle Henry was at the laundromat).

``Don't like my eggs runny,'' Eddie reminds her, the same way he does
every morning. ``And don't skimp on the salt.''

``Bad for your blood pressure.'' Without even looking, Rayna reaches
into the right-hand corner of the cupboard, pulls out the salt shaker
and measures out a teaspoonful, tossing it into the skillet with the
eggs. ``If you're going to eat with so much salt, you ought to at least
get more exercise.'' But no way Eddie's going to listen to her. Seems to
crave the stuff. Won't even read the newspaper articles she clips and
posts on the refrigerator.

``It's your life. Guess if you're determined to do yourself in, isn't
any way I can stop you.'' Rayna glances out her kitchen window, checks
the thermometer she got free at Grady's hardware and tacked to the
bird feeder, close enough for reading, for seeing what kind of day she
can expect. Hotter than normal today, near 70 degrees already and humid.
Miami is probably cooler with the ocean breeze.

``Can't be,'' Rayna mutters, shivering, rubbing her arms to get the
blood circulating. ``Just can't be.'' She grabs the Windex from under
the sink, sprays the window, wipes it with a paper towel. Must be her
own image reflected in the glass.

``What? You know I can't hear you when you mumble.''

``Nothing, I didn't say anything.'' Maybe something is wrong with the
glass. Hadn't she been telling Eddie she wants new windows, the kind she
can clean from the inside? She doesn't move, doesn't blink; her eyes
sting, dry from the air, from lack of moisture.

``Adele,'' she whispers, ``can't be.'' But it is Adele. Adele kneeling,
reaching, yanking, turning, and uprooting the crab grass, dandelions,
chickweed and even the violets breaking ground in the flower­beds behind
her house. Adele whose arthritis bothered her so toward the end that she
couldn't walk down the back steps without her cane.

Rayna raps on the window, calls out to Adele, then quickly puts her hand
over her mouth, hoping Eddie isn't listening. She bites at a torn
cuticle as Adele rises with the weeds cradled in her arms, then fades
like the watercolor of Lake Superior Rayna painted from memory so many
years ago in her high school art class.

Rayna rushes to the back door, opens it, shouting, ``Get away from
there. Go on.'' She claps her hands, hard, angry that the Simpkins' coon
dog is loose again. ``Darn dog,'' she mumbles. ``Always digging up other
people's yards. Adele never could stand that dog.''

``So? Doesn't bother me none. Doesn't bother Adele none anymore
either.'' Eddie scoots his chair back, a screech against the linoleum.
``You never tinkered with this weird stuff before. I don't like it. You
think you can pretend I didn't hear you call Adele? Why can't you let
the woman rest in peace?''

``Now don't you go getting up and getting in my way.'' She pops the
toast onto a plate and reaches over his shoulder, setting it in front of
him, kissing the back of his neck as she draws back.

``You're up to something. You think I can't tell when something's not
right with you?'' Eddie cuts off the crusts, darker than the rest. His
jaw clicks as he chews, the sound of a socket slipping out of the joint.
``Next it'll be flying saucers and little green men you're seeing in
Adele's flowerbeds.''

Rayna stares out the window again, concentrates until Adele begins to
take shape, her hair long, grey, coarse, too heavy now to stay pinned in
a bun, loose and tumbling down her back, legs bent against the grass,
her hands buried in the fertile soil, digging there.

Rayna dares not look away. Hope rises in her like the sour dough bread
she made early this morning, covered with a damp dish towel and left to
ferment while Eddie slept, sheet wrapped between his legs.

``You hearing what I'm saying? Gets spooky when you start seeing people
who aren't there any­more. Even you have to admit it's a lot different
than Uncle Henry's bikini shorts. Not natural.''

``No need for you to be shouting when I'm just two steps from you.''
Rayna scoops eggs onto his plate, wishing he would listen to Dr. Metpath
about his cholesterol and glad he hasn't discovered she's been using
Eggbeaters to make those coconut cream yellow cakes he loves. ``Ever
thought about going to Miami, Eddie?''

``Nah, might run into Uncle Henry and Jennifer.'' Eddie pours ketchup on
his eggs, stirring them around until they're covered with red. Next the
salt, specks of white like stars. ``Reminds me, is my uniform pressed
for the parade tomorrow?''

``In 37 years, have I ever forgotten?'' She can't wait to see him march
by in the Memorial Day parade, handsome as ever, just a bit more of him
than when he came back from San Antonio, wearing that very same uniform,
stepping off the Greyhound bus, full of places he saw and she didn't.

``Darn fly,'' he waves it away from his eggs. ``Hope they're not out
full force for the picnic. You know, Aunt Edna refuses to eat potato
salad anymore.'' He points his egg-loaded fork at the fly. ``Not a bite
since last Memorial Day.''

``Remember how she had a big chunk of potato half way to her mouth
before she spotted the fly stuck to the potato with mayonnaise? Rayna
laughs, smoothing her green dress over her belly, a well-fed woman.
``Who knows how many she already swallowed.''

``That's one you should of seen coming,'' Eddie chides Rayna. ``Sure you
didn't even call those flies? You never did like Aunt Edna.''

``Eddie Gullavor, you know darn well I can't conjure up whatever I
want.'' She fears Adele may be just a fluke, the same kind of luck as
her two winning cards at Bingo last week. Rayna stares across the table
at him, her thumb scratching at the green Formica tabletop as if she can
get below its shiny plastic surface, right down to the thing itself.
``If I could do that, I'd see us in Miami. Yep, Miami. Probably on a
yacht.''

She runs her tongue over her teeth, licking away the salty taste, lifts
her hair from the nape of her neck as she starts humming then singing
full voice, buoyantly, ``O spacious skies\ldots from sea to shin­ing
sea.''

Eddie takes up the chorus, an echo: ``From sea to shining sea.'' His
voice breaks reaching for the high note. ``So what's the difference
between a sea and an ocean? `From ocean to shining ocean.' Doesn't sound
right.''

He sets down his fork, watches Rayna's hand turned palm up, fingers
moving as if she is sifting a fistful of sand. ``Rayna, stop it.'' He
leans across the table and catches her hand in his, her fingers
scuttling across his palm like a ghost crab without its shell.

``Let's do it, Eddie. If not Miami, then San Francisco or Seattle or
anyplace near the ocean. I need to see an ocean. Salt instead of fresh
water. Bigger than Lake Superior. Fish the colors of crazy quilts.'' She
pauses, then adds, ``You know, some people think life started in the
ocean.''

One at a time, his words like hard steel balls in a pinball game,
``You're not yourself. This has got to stop right now.''

Rayna waits for his words to bounce off corners, hit the target and
settle into her pockets, weigh her down. She can't explain. It's as if
she can hold Miami in the corner of her eye, a speck of dust at first,
then a house and lately even whole city blocks, and she walks those
streets. The hot sun at mid-day makes her perspire. Her pale face has a
soft shine like satin, smooth as the pearls clasped round her throat.
She is there.

Maybe it's just all the books on Miami she's been checking out from the
Bookmobile. Or maybe it's the \emph{AAA Tour Guide} she sent away for,
especially the photos of pink, aqua, yellow, beige office buildings,
hotels and houses, with greyhounds and flamingos and even peacocks and
nymphs carved into limestone, etched on glass. She knows she must go
there. She must leave soon.

She wonders if it's the same feeling Eddie got years ago when he worked
the mines, back when he rode the elevator down the shaft, stepped out
into the darkness, with the light on his hardhat to keep him from
blindness. Sometimes he used to bring her chunks of iron ore, chipped
from rocks, from miles of tunnels blasted deep beneath the earth's
crust. But that was before 23 men were killed when one of the tunnels
collapsed, before the strikes, before the mine closed for good.

``We've never seen the ocean. Just think, we can go right to the edge.
We can walk along and feel the land drop away.'' She shows him the book
on Miami she got yesterday from the Bookmobile. It comes once a month,
39 miles from Calendar and back again. The book opens to her favorite
photo, the sun filling space between two palm trees, slivers of blue at
the edges, ocean or sky. The map at the back opens into her hands. She
points out Haulover Beach, just a short ways down Collins Avenue. Right
on the Atlantic Ocean.

``Ever feel the ocean between your toes?'' She wiggles her toes inside
her shoes, against the curved leather that binds her feet.

``Drug dealers, sand, palm trees and retirees. That's all Miami is.''
Eddie slips on his work jacket, his name embroidered across the back
under the Mobil logo, a walking advertisement. ``Some­thing wrong with
Raintree all of a sudden?''

``Wait,'' but she can't stop the accident. His jacket swings loose and
catches on the orange juice carton, knocking it to the floor. ``You're
late. Go ahead. I'll get this.''

Rayna squirts Joy liquid detergent into the dishcloth and kneels down on
the linoleum. The orange juice puddles, the shape of the Grand Canyon, a
canyon too wide to jump across, just like she saw on the Channel 9
documentary last night. But you can get down inside with a helicopter or
go right to the very bottom by riding a mile or walking. Close enough to
touch slate ledges streaked the color of marigolds.

She can see straight down, all the way to the canyon floor and feels
dizzy. She stands up, sways and keels over, forehead smack in the middle
of the orange juice. Hard enough to knock her out. The fly feeds off the
orange sweetness, a slight buzzing beside her ear when she opens her
eyes, orange pulp stuck on her eyelashes, fluttering there like
flypaper.

She comes to slowly, not sure where she is. Then she remembers. She dabs
at her dress with the dishcloth, wipes her face, then the beige
linoleum. A faint outline remains, two streaks for the canyons' edges.

The rest of the morning she works in the garden, clearing away dead
leaves, turning soil. Earth worms rise to the surface. Too light for
slugs who leave their trails of slime after dark. She fertilizes the
lawn, a holiday ritual: Memorial Day, Fourth of July and Labor Day. Days
of fireworks and sparklers.

On Memorial Day, Rayna and Naomi stand shoulder to shoulder, as they've
done year after year, watching the parade on North Central Avenue.
``Here comes, Eddie,'' Rayna says, waving her flag at him and the small
group of veterans who left Raintree and returned to stay after World War
II, after Korea, after Vietnam, after the Persian Gulf.

``Still, the world's not safe,'' Naomi sighs, tugging her bra strap back
up on her shoulder and tightening it, fingers moving deftly under her
flowered dress. ``Man knifed his own wife last night over at Carson
Ridge.''

``I know,'' Rayna says, hugging Naomi and pulling her back from the curb
just before Charlie Dawson got so busy waving and looking back at Miss
Raintree that he forgot to watch where he was headed. He lost control of
the red convertible on loan from Courtesy Chevrolet, ran right over the
curb where Naomi had been standing a second before, tires crushing the
cooler and the picnic lunch Naomi packed that morning.

``Nothing's safe anymore,'' Naomi says, aimlessly lifting the flattened
cooler with her foot and letting it drop again onto the concrete
sidewalk. ``Not even in Raintree. Can't do a darn thing about it either.
Just got to take it.''

``Maybe we can. Do something about it, I mean.'' After all, hasn't she
just saved Naomi's life? Well, maybe not her life but at least kept her
from the bruises and broken arm that she would have had. It's the sign
she's been waiting for, maybe nothing worth putting up in neon lights,
but her sign nonetheless, round as a pearl plucked from an oyster, black
as her dilated pupils full of light.

That night, Rayna tips her head back against Eddie's shoulder as they
and the rest of Raintree watch the fireworks burst silver, gold, red,
yellow, white against the starless sky like unset gems displayed on the
jewelers' black velvet cloth.

After the fireworks, Raintree neighbors tell each other good night, walk
slowly back into their homes, not wanting the day to end. Rayna climbs
into her bed beside Eddie, lies with eyes wide open, knowing what she
must do, waiting for Eddie's ragged snores to tell her when she can
dress again with­out waking him.

On her way out, she picks up the books on Miami, the maps where she
traced different routes. So many ways to get there. On the familiar
Michigan back roads, she drives slowly, watching for deer and raccoons
as she rounds crests the hills, flicking the headlights to bright. Rayna
passes through Brownstown, past the Farmer Jack, past the laundromat,
past the drive-in movie theater where she and Eddie went every Saturday
night when they dated, past roads she recognizes.

Rayna pulls into the first Rest Area and parks under one of the lights
turned on for late-night travelers like her. She studies the pull-out
map in the library book she showed Eddie, tracing with her finger the
wiggly lines that lead to Miami. She has folded and re-folded the map so
many times, the creases are starting to tear. But she locates the
interstate, a junction where U.S. 23 connects in Briarwood.

Although she's never been to Briarwood, she knows it's about thirteen
times the size of Raintree, a very large city she hears about on the
news. Like other big cities, it has its share of crime. But she's not
frightened, not at all. And, as she leaves the Rest Area, she pushes
hard on the accelerator, knowing that she's over the speed limit, but
she likes the sound of the road beneath her tires, the sheet metal
encasing her, protecting her, cutting through the night.

She stops again in Mettalwood to check where she is. Hard to see
Briarwood, right on the fold. In Ironsville, she checks again and sets
her odometer to make sure she doesn't roar right through Briarwood. Only
twenty more miles, then the interstate. She watches the miles pass. The
odometer shows twenty miles but no Briarwood. Too big to miss and she's
on the right road.

No traffic this time of night, actually this early in the morning. So
Rayna just stops smack in the middle of the road, doesn't even pull off
to the side. Darkness all around her. No gas station, no stores, no
houses. Nothing but dark. She flicks on the dome light and pulls out the
map. Can't even find Briarwood now. Looks like Briarwood has worn off in
the crease, too much folding and refolding the map, or maybe it was
never there.

Nonsense, Rayna tells herself. Briarwood was there. So where did it go?
She knows she's not lost. Briarwood is lost. Maybe getting lost will
scare the town enough to clean itself up. Like Naomi says, ``Nothing's
safe anymore.''

Rayna rests her eyes. Behind her eyelids, she pictures Briarwood. As it
was before she came. Now the void, the empty space where it should be.
Rapid blinks. Her third eye, focusing, capturing the image, turning it
upside down, holding it there against the retina.

No different than one of those souvenir plastic towns inside a snow
globe: Briarwood. Turn it over, shake it, set it up on a table and fake
snow floats down, covering the town in white flakes suspended in fluid.
An embryonic sac. Shake out the guns, the knives, the anger, the greed.
Break the water, let it rain torrents until the streets are clean. She
looks again, her vision clear.

Briarwood, the sign says, population is 15,434. Rayna rolls down her
window to watch the chil­dren walking to school, listens to them as they
pass, giggling and telling their stories. Their faces are clean and
moist as if they've just stepped from their morning showers. They carry
books clutched in their arms. She knows that they will also learn
someday about their birth, how the waters broke before they were born.

She wonders if Eddie will think she's los when he wakes up in the
morning. Probably take him awhile to realize she hasn't run to the store
for more eggs or walked over to Naomi's for an early morning cup of
coffee. Should have left him a note. How can she leave a note, though,
when she doesn't know what to say, how to explain it? She trusts him to
figure it out by himself. When he starts looking, he will see the books
are missing the books on Miami.

Rayna checks her map. She finds Briarwood, back on the crease, bold
letters as if the name is freshly typeset. She licks off the smudge of
ink on her finger and passes through Briarwood. She picks up the
interstate with no problems.

Traffic is heavy. So many cars returning home after Memorial Day. Cars
from Georgia, Ohio, Kentucky, Tennessee and even Florida pass by with
carriers on top packed full of tents, lanterns, camping gear. Vans
towing boats, still dripping water as if their owners snuck in one more
ride across the water before heading home. Pickup trucks loaded with
fishing gear, tanker trucks carrying more fuel, and the eighteen
wheelers carrying goods of all kinds across the country.

She stops at a Mobil station and pumps her own gas. Eddie will never
stop anyplace but Mobil, even when they get really low on fuel.
Spreading out the map on the counter, she asks the man who gives her
change the name of a town she has accidentally put a hole through with
her ballpoint pen. He doesn't know. In fact, he thinks there's no town
there, at least it doesn't show on the map pinned to the wall right
above the cash register. But Rayna knows differently.

When she arrives at the hole in the map, she stops the car and pulls off
to the side of the road. Nothing there but cattails, furry brown spikes
reaching up, rooted in the marshland beside the road and a red-winged
blackbird rising from the shallow creek that cuts through the field,
over by the weep­ing willow. Reminds her of Raintree, the spot down at
the bottom of the hill, right where Everet Mason's property ends, a spot
no one owns. She and Eddie often go there to picnic and pick wild
strawberries in season.

Rayna opens her car door and steps out onto the road. She has come to
trust that eye that no one sees, that eye that moves within her head.
She pictures the town, the café where she will eat lunch, the houses,
the people. The town begins to evolve, gradually, just as she envisions,
emerging like an oil painting, layer upon layer until it is complete,
until she can stand back and admire its depth, its detail and as she
squints and moves closer, she sees the brushstrokes that bring it to
life: Flagsville, Ohio.

She opens her map. The hole is gone, covered with a black dot and the
name, Flagsville. She climbs back into her car, turns the ignition key
and drives up and down streets. When she spots the café, she pulls right
into the parking lot, leaves her car unlocked and goes on in, seating
herself in the first booth, liking the way the tufted black vinyl feels
solid against her back.

``Ever notice how everybody's in a better mood during a holiday?'' the
waitress asks. ``You know, I didn't even mind coming to work today.
That's sure not like me. No way. I was thinking of leaving this place.
But, you know, somehow it looks different today.''

Rayna says, ``Yes, I know,'' and orders a hamburger with everything on
it and fries, just like she used to do before she began worrying about
Eddie's cholesterol.

The waitress sets the hamburger and a bottle of ketchup in front of
Rayna, and asks, ``Have everything you need?''

Rayna nods and pulls out her map, worn thin and torn in spots, some
names of towns and cit­ies illegible. She knows she must pass through
them all before she gets to Miami. A drop of ketchup oozes from her
hamburger and drops on her map, covers the name of another spot she will
go.

When the waitress returns to refill her coffee cup, she asks Rayna where
she's headed. When Rayna tells her Miami, the waitress sighs, ``I always
wanted to go to Miami. Are you going on vacation?''

``No.'' Rayna puts a spoonful of sugar in her coffee and stirs. ``I'm
going on business.''

Rayna looks around the café. Everything glistens, captures and reflects
back her own image. The chrome chair legs, the window hung with brown
gingham valances, even her coffee spoon that she wipes with a napkin.
She sees herself: a middle-aged woman, hair still brown, the color of
mud she often thinks, skin reddish from veins too close to the surface,
her faced lined and slightly weathered, lips parted, ready to smile. No
jewelry, except, of course, her wedding ring that she never removes. All
in all, a practical woman, solid, the kind of woman strangers ask for
directions.

``On business,'' Rayna repeats. ``I make maps. You know, trace in the
new routes and bypasses. All the changes.''

``You must travel a lot.'' The waitress sits down on the other side of
the booth and sets the cof­fee pot on the table, taps her long pink
fingernails against the table, the sound of rain against a tin roof.
``Sometimes I think about leaving. Just never seem to get around to it,
though.''

``Why not? You look like a woman with vision.'' Rayna opens up the map
and holds it out. ``All these places. So much to see. So much to do.''

The waitress reaches for Rayna's map, knocking over the salt shaker,
spilling the salt granules. ``Oh, no.'' She brushes the salt off the
table into her right palm, then tosses it over her left shoulder.
``Better safe than sorry,'' she says. ``Who needs bad luck? Not me,
that's for sure. Had enough bad luck already to last me a lifetime.''
She sighs, ``The stories I could tell you.''

``Someday,'' Rayna says, looking at the waitress' red plastic name tag,
pinned to the brown ruf­fled apron, worn over a brown gingham dress with
puffed sleeves. ``Yes, Barbara, someday the notion will hit real hard.
Then, just like that,'' Rayna snaps her fingers, ``you'll do it.''

After paying the bill, Rayna asks for directions to the nearest jewelry
store. She buys pearls, the real kind, not freshwater pearls or
cultivated pearls, but pearls pried from oysters by a fisherman, pearls
round and perfect, lustrous white pearls to clasp in a single strand
round her neck.

Rayna Travis Gullavor wears these pearls, warm against her skin.
Sometimes she fingers them as she drives. She sees Eddie pumping gas and
cooking eggs, sees him spread out across the whole bed at night. For
Eddie, Naomi bakes sourdough bread, cans the strawberries he picks and
goes on televi­sion asking Rayna to come home. But Rayna cannot go home.

Not until Rayna drives through all the towns, until she sees the way to
real Miami is safe for any traveler, until she walks out into the ocean
and waves crest, wash over her body, pick her up and float her out past
lifeguards, out to where salt water buoys her up and the pearls burst
from their string and dissolve to sand---until then, she must travel
alone.



\cleardoublepage
\chapter{Minor Adjustments}

Our desperation drove Brittany and me full speed in our hunter green van
right out to the army surplus store, the one down there on Telegraph,
next to the construction company. Heck, we're both pacifists. But enough
was enough.

For once Brittany, my fifteen-year-old daughter, and I agreed on
something: men turn your life up­side down and shake it as if you're
only plastic figures in a snow globe. Well, we would see where the snow
settled.

Brittany pounded on the door of the surplus store as I coughed and spit
bits of sawdust into Bradford's oversized, monogrammed handkerchief. The
kid with sinking ship tattoos on both arms waved us away.

``This is an emergency,'' Brittany explained, tilting her head and
staring right into the kid's green-metal eyes as he backed up to let us
in, locking the door behind us.

``Just call me Rocko,'' he said after Brittany told him we needed
whatever it took to survive environmental warfare. Didn't take him long
to get into the spirit of things as he yanked out battle fatigues,
helmets and gas masks left over from some military maneuver.

``Cool,'' Brittany said twirling in front of the three-way mirror,
sleeves hanging down past her knees and the tube on the gas mask
dangling. Rocko nodded his shaved scalp that had green-and brown-tinted
stubble poking through on top. Brittany rubbed her sleeves over his
head. ``Great camouflage.''

Took two Visa cards to pay for all that gear including the canisters of
Mace Rocko convinced us to buy for self-defense. ``Be on guard,'' he
shouted as we climbed into our van. ``It's a war out there.'' As if we
needed to be told.

On the way home, all Brittany could talk about was Rocko, Rocko, first
and last name the same. When I cautioned her that Rocko was a guy, no
more and no less, she made the error made by young women everywhere.
``He's different,'' she said. ``But Bradford wouldn't approve of Rocko
would he?''

``Don't call your father Bradford. He doesn't like it,'' I responded
automatically as the garage door closed behind us. ``Never know. Maybe
Rocko carries one of those military handi-mechanic tools on a
keychain.''

Our house was still under siege. Bradford was squatting with his head
inside the bottom kitchen cupboard, door open, sanding away at something
and clogging the air with sawdust. My pots and pans were heaped in a
pile in the middle of the floor, filling up with sawdust.

I yanked at Bradford's ankles. ``Get out of my cupboard now. You
promised. No more pro­jects.'' I forced myself to breathe in and out
simply and slowly and visualized Bradford roped to a straight-back chair
while every single Craftsman tool he owned paraded back to Sears.

``Wasn't closing right,'' he muttered, shutting off the sander. ``Just
trying to help.'' He climbed back into the cupboard. The whirring,
whining sound bounced up and down my backbone.

``Closed fine for me. Just fine.'' As if he asked. As if he cared. He
could never say I hadn't warned him.

He poked his head out, looked right at Brittany dressed in her fatigues
and gas mask and didn't say anything except, ``Get the Dust Buster will
you, Sweetheart? Not the shop vac, but the small one. I need you to hold
it here and catch the sawdust.''

I saluted Brittany before putting on my gas mask. Felt good to know she
was covering my flank as I marched into the family room, turned the
volume high on that new comedy everyone was raising about and we tilted
our helmet-heads against the beige velour cushions, ready to laugh. Like
a geyser, water spurted out of the bathroom faucet as Mr. TV Fix-It Man
stroked his chin.

``Hey, Bradford must have crawled out of his cupboard right onto the big
screen.''

I took my gas mask off to get a better look. When Mr. Man rummaged
through his box of high-tech tools and each gadget was bigger and had
more dials than the one before it, I had to admit something in his eyes
looked like Bradford. When he kicked the vanity and shouted and cursed,
I knew the language. When the blonde in her bikini swam in to rescue him
by turning the faucet off, I would have sworn it was Bradford except I
could still see Bradford's legs hanging out the cupboard door in our
kitchen.

``Just leave everything where it is,'' Bradford said the next morning
before he left for work, dressed in his pin-striped navy blue suit with
white shirt and sedate red-and-blue striped tie that fooled everyone at
work into thinking he really was a chief engineer like his business card
said, a man in con­trol. ``I just have a minor adjustment to make.''

Sure, sure. How many times had I heard those very same words? Minor
adjustment translated into ``done this year some time unless I get
interested in something else first.'' That's why last summer Bradford
never finished putting the eaves on the lop-sided gazebo that had an
overhanging roof on only one side of the octagon. Of course, no sense in
painting it until the eaves were up. Not that it was Bradford's fault
since Glenda next door yoo-hooed him right on over to fix her garbage
disposal just as he was getting ready to go to Home Depot to get the
eaves he should have gotten when he bought the wood to build that stupid
gazebo that none of us wanted anyway except him.

I stumbled over the turkey broiler. ``I hate to cook anyway,'' I
screamed, heaving the broiler against the cupboard door and raising both
arms in victory as it left a big round dent in the center.

``Bullseye, Mom. Way to go.'' Brittany stubbed her toe on the cast iron
skillet and threw it to­ward the turkey broiler, leaving her own dent.
``That ought to keep Bradford busy for a while. You know what they say
about idle hands.'' She rolled up about twelve inches of pant leg on her
fatigues.

``You can't wear that to school. It'll give your principal conniption
fits.''

``Yeah,'' she sighed through the gas mask. ``Guess it will,'' I heard
her say as she leaped over the pile of pans, her pant leg catching on
the lid of my crockpot and dragging it outside behind her, wiping out
her own footprints in the late March snow.

I can't say I was one bit surprised when Mr. Leary, the school
principal, called me at work and told me she was suspended until she
apologized for stirring up trouble. He sputtered and said that in his
day girls knew how to dress appropriately. ``Sure,'' I muttered, ``in
mini-skirts and platform shoes that ruined their arches.''

Silence at his end for a moment, then the lecture. Made me feel like I
was back in school my­self when he said, ``I can see where Brittany gets
it from. You need to pick her up. Now.''

``Oops, gotta run. Late for a meeting. You do have her father's number,
don't you?'' And I hung up.

Bradford enlisted you-hoo Glenda in his battle to convince Brittany and
me to discard our gear which I must admit was a bit smelly by April. She
did her best. Even lured us into her kitchen by baking our favorite
chocolate-frosted brownies. Raving about Bradford's talent, she
demonstrated how he fixed her garbage disposal so it chewed up a whole
carrot in two seconds and said, ``A man like Bradford is a treasure.''
Brittany and I just tightened our gas masks and let her back door slam
when we left.

Fact is that Brittany and I had become comfortable in our fatigues. Each
colorful battle patch Rocko helped us pick out signified a rear flank
action. We awarded ourselves a star above the right pocket on the day we
disabled the garage door opener so that even after two days of
tinkering, Brad­ford couldn't figure out what was wrong with it.

By the time May rolled around, Brittany knew all the words to the Army
marching song, talked me into going back to the surplus store for mess
kits and combat boots, and developed a Rocko-induced craving for
dehydrated potatoes and beef jerky and for Rocko himself. Even Bradford
couldn't complain about my not cooking since my pans took up most of the
kitchen floor and he was still waiting for a back-ordered hinge for the
kitchen cupboard.

In the meantime, he was in the process of painting the bathroom in an
oil-based, enamel white when he discovered he was a gallon short (and
left drop cloths and open paint cans on the sink because as soon as the
store could mix it without putting too much starlight white in with the
ermine and egg­shell, he would finish the back wall). ``Don't care if
they do it a hundred times, they're going to get it right,'' he said.
``I won't settle for less.''

So, each night, he dabbed some paint on from a new can and the
commercial exhaust fan he installed in the kitchen last week sucked the
paint fumes throughout the house. That fan sure did work, even sucked up
all the purple blooms from the African violet I had setting on the
window ledge above the sink. ``Now that's what I call a real exhaust
fan,'' Bradford said. Most days I wished it would suck him right up,
paintbrush and all.

At exactly six o'clock on a Wednesday, May 13 (no way I'll forget that
day ever) Brittany, Brad­ford and I sat in the unmowed grass with our
backs against the lopsided gazebo eating our rations which Rocko kept in
stock for us.

It was then that Rocko came roaring up the driveway on his
Harley-Davidson. When he spot­ted Bradford, Rocko braked too quickly and
catapulted over the handlebars.

``Hey, guy, what did you bring us?'' Brittany said at the same time
Bradford started yelling, ``Get out or I'll call the police,'' and I ran
over, telling Rocko, ``Don't move. You might have a con­cussion.''

Bradford stood over Rocko, holding a hammer. ``Don't try anything. I'll
use this if I have to.'' Bradford raised the hammer and gripped it with
both hands. Who knows where he came up with a hammer on such short
notice. Then again, Bradford doesn't look fully dressed unless he's got
a tool in his hand. ``I mean it,'' Bradford said, lowering the hammer as
Rocko wiggled his legs. ``Don't make me hurt you.''

``This is ridiculous. Give me that.'' When Bradford wouldn't let go, I
unclipped my canister of Mace from my belt and aimed at his left ear.
I'd been wanting to do that for some time. When the wind blew some into
his eyes, he let out a yowl that set the neighbor's black Lab barking
and Bradford dropped the hammer like it had teeth. ``It'll make it
worse,'' I warned as he rubbed his eyes with his knuckles. Then, I
introduced Rocko.

Maybe it was the ship tattoos sinking more on Rocko's rippling biceps as
he clenched his fists or the two-inch, skinny orange braid that fell
over his forehead or the black rubber marks his motorcy­cle left on the
concrete driveway that provoked Bradford. ``Leave. Now,'' Bradford
hissed, still rubbing his eyes.

``Or what? I know all about you,'' Rocko said, sitting up on the grass.
He handed Brittany two canteens. ``In case the old man breaks the water
pipes.''

``What old man? What water pipes? And who are you to give my daughter
anything? Get out.'' Bradford and Rocko ran for the motorcycle at the
same time and each took hold of the seat and a handlebar and pulled
opposite ways. The motorcycle rose slightly and fell one way and then
the other, again and again, as the two men pulled in opposite
directions, each screaming at the other, ``Let go.''

Don't know how long this would have continued if I hadn't pointed out
the twisted seat and bent fender. ``Oh, man,'' Rocko and Bradford moaned
at the same time as they let go of the motorcy­cle which landed, again,
with a solid smack on the concrete and laid there.

``Always wanted one of these hogs,'' Bradford said. Looks wounded laying
there. Not to worry, though, I've got the tools to fix it.'' He slung
his arm over Rocko's shoulder as they walked in step toward the garage.

Brittany stamped her foot, ``Get back here, Rocko.'' Rocko just kept
moving, nodding his head at whatever Bradford was saying. ``You're a
deserter; that's what you are: a deserter. And a traitor, too.'' Her
gold-flecked brown eyes, shaped like mine, challenged me. ``I thought
Rocko was different.''

I hugged Brittany to me and patted her back. ``I know,'' I murmured.
``Oh, how I know.''

A slight breeze from the north brushed petals from the cherry tree that
drifted onto Brittany's shoulder and mine. For a moment, we mistook them
for snow. Then, we smelled the fragrance.

Didn't take us long to decide what to do. And no way were we going to
leave a note for Bradford or Rocko since they could be holed up in that
garage for days.

Brittany and I booked the most expensive room available at the
Ritz-Carlton for a week. We stripped off our fatigues and gas masks,
changed into flowered spring dresses, packed our suitcases, drove the
few miles to Dearborn, signed in, and immediately found the whirlpools
where we laid back in the churning water and breathed clean, moist air.



\cleardoublepage
\chapter{Not Fast Enough}

I may have been only ten years old but I could run faster than any boy
or girl on our street, faster even than the twelve-year-old boy across
the street who was always running out of his house, up the street, away
from his drunken father who sprinted after him shaking his belt. Faster
than my friend Joyce who never even walked fast because her hair might
get messed up. Faster than Bobby who preferred riding his bike
everywhere.

When I saw the notice posted on a tree at our school playground, I knew
someone nailed it there just for me to see. See it I did. I stared at it
so long it must have tattooed itself on the back of my eyeballs.

Picnic, Games \& Prizes\\
Everyone Welcome\\
August 23, 1953. Noon.\\
Baseball, horseshoes, checkers, dominoes, more\\
Running contest. Ages 10 to 12.

There was more but I stopped reading. Running contest was what caught my
attention. I ran around that maple tree at least twenty times for good
luck. When I ran, my mother always told me to slow down, told me that
ladies walked. I knew right then I wasn't a lady and never would be. I
was a runner and that was that. I went into strict training. Only two
weeks to prepare. Short but time enough. I ran everywhere: to the store
to get milk, up the stairs when I wanted to beat my sister to the
bathroom and down the stairs when I was done and right out onto the
concrete sidewalk and across the neighbor's grass, even though he shook
his fist and threatened to tell my parents if I didn't stop. I didn't
even slow down. I couldn't. I wouldn't, no matter what. Even during
sleep, my legs churned, kicking the stuffing right out of my mattress.

Neighbors shook their heads as I whizzed by. All the dogs in the
neighborhood broke free and followed behind me yapping and ripping at my
heels. The birds flew lower in the sky as they tracked me.

Often Dad stood on the sidewalk, yelling, ``Lift those knees higher.
Higher.'' He shouted until he was hoarse but I still heard him croaking
when I passed. ``Left, right. Higher.'' And I did. I lifted them so high
I figured I was going to knock my own mouth shut.

Mom's mantra was: ``Stop that right now. You're going to get a heat
stroke. Then, what will the neighbors think?''

On August 23\textsuperscript{rd}, at high noon, I stood tall at the
start line. I knew I could beat each and every kid there. My flat chest
puffed out and I shuffled my toe along the line, trying to appear
modest, some­thing all girls should be according to my mother. Just
before the start signal, a big guy showed up and toed the line. I'd seen
him around, playing baseball, football, hockey, soccer and being cheered
on by a bunch of high school girls. If you had been watching from the
sidelines, you would have seen me cross the winner's line on the heels
of the big guy. Despite what you might have thought, it was sweat I
wiped from my eyes with the palm of my hand. But when the big guy went
up to get his prize, a Siamese Fighting Fish, I saw the disappointment
in his eyes even though the girls marveled at the pur­ple fan tail,
giggled when the fish bumped up against the small round fishbowl meant
for guppies, not for the magnificent Siamese Fighting Fish.

I wanted that fish in the worst way. I would name it Johnny, just like
the guy who sat next to me in math, the guy I had a crush on. I pictured
all the room Johnny would have to spread his tail in my ten-gallon tank.
I offered the big guy a dollar for the fish. He just sneered at me.

To this day, I'm convinced that the big guy was really thirteen. He
cheated. I know that in my heart. Just like I know he dumped that
glorious Siamese Fighting Fish down the sewer after the girls left.



\cleardoublepage
\chapter{The Party}

Karen stared at her husband across the room. She watched his lips move
and leaned forward in her chair trying to hear what he said to the other
men. But his words were lost in Ruth's laugh.

``The Pailmans aren't coming tonight. Their little boy has a cold,''
Ruth said, touching Karen's arm.

``That's too bad.'' She lit another cigarette and blew the smoke
forcefully between her rigid, slit lips, all the time trying to remember
who the Pailmans were, trying to make the connection.

But tonight her mind wouldn't twist in the right directions. She
reminded herself that she'd known the Pailmans for years and she could
visualize their faces and things they'd done together, but she still
couldn't bring them close tonight, any more than she could draw closer
to Ruth, who'd shared her secrets since high school days. Even Ruth's
son, whom she loved but did not yearn for, was only a mass of recalled
sensations at this moment. And her own unconceived children were buried
some­where inside her, consciously denied but waiting still. In this dim
room, her mind was beyond manipulation.

She was being spoken to. She knew it. The laugh, the words, and the warm
fleshy hand touch­ing her arm reminded her. Karen forced herself to say
the expected words. ``I hope it's a boy. You'd like another boy,
wouldn't you?''

``Come on, Karen. What are you thinking about tonight? You know we want
a girl this time.'' She shook her head at the other women and turned
back. ``You'll find out someday what hellions little boys are.'' Ruth
smiled, confident of her womanliness, yet not even showing a slight
stomach bulge.

The women talked for a while about their children, about their
children's accidents, about their children's babysitters, and about
their pregnancies.

Karen heard clearly a woman's voice laughing in the dimness at some
story another woman told. She knew the voice; she knew all the voices;
she'd heard them for years at high school parties, at college parties,
at summer swimming parties, at New Year's Eve parties. It was as if they
all belonged to one family, a sort of group marriage.

She turned to look at Ed, the man they called her husband, the man who
looked at Ruth's whiteness. Then she spoke, surprising herself. ``Well,
Ruth, I hope it's a beautiful girl. A beautiful girl who will live
forever in her tomboy treehouse.'' She blew a screen of cigarette smoke.

``Please, Karen, blow the smoke the other way.'' She parted her lips
exposing small teeth clenched tightly together. The words fought their
way from behind the hard whiteness. ``I just don't understand you
tonight. Are you drunk?''

``Yes, that's it. Tonight I am drunk. No one can understand a drunk. No
one is expected to. I don't expect you to.''

She heard Ruth say something to her, something that was meant to be
reassuring. She watched her twist on the arm of the chair, moving her
legs more to the right and smoothing the skirt over her thighs,
carefully so the life inside would not be disturbed, the life that
hadn't even made itself felt yet except in morning sickness.

The women were still telling funny stories about their children, each
trying to outdo the other. She listened to Ruth speak of her son,
watched her methodically stroke the tufted wool of the thick chair arm.
If the party had been quieter, she would have heard a soft, slurred
sound, the sound of Ruth.

``But I did read it somewhere,'' Ruth protested.

She must have nodded her head ``no,'' but it hadn't really been to what
Ruth was saying.

``It's true, Karen. Children are upset with the blood and cruelty of
fairy tales. Like the mean old witch who eats children in `Hansel and
Gretel.' It would upset him if I read that sort of thing to him.'' Ruth
looked toward the other women for support.

The damp blackness of Karen's eyes suggested a smile. ``You're right.
Children shouldn't know they get eaten.''

Ruth hadn't heard. Her legs were swinging in beat with the record that
was piped into the base­ment, speaking words of fleeting truth to the
congregation below. Her heel caught on the fabric, bend­ing her forward
and revealing more of the small white roundness that struggled to free
itself from the flowered dress. Ruth's temples were damp and her hair
curled carefully around the water droplets.

Karen smiled at her, the just-right smile of departure. She would play
the game, the game of hide-and-seek that they all played, the game that
even children knew.

``Excuse me, Ruth. I need another drink.'' She stood, pushing her slip
strap back under the summer's sleeveless dress.

``Don't you think you've had enough?'' Ruth's face wrinkled in
disapproval. ``Well, there's nothing worse than being sick the next
morning. It makes you weak to get sick. After all, I should know.''

``Sure, I've had enough. But what else can a person do when she's had
enough except to take another drink?''

She could feel Ruth's eyes trace her ascent, watching for a visible sign
of sickness. The tape­worm eyes measured the body flesh from top to
bottom, checking the dimensions of femininity.

As Karen passed the screen door, the coolness of the summer night
reached in and clutched the back of her neck, strangling her with its
freshness, its dark clarity. And in the kitchen, she wiped the wetness
from her forehead with the palm of her hand, a man's gesture. She set
the empty glass on the sticky Formica counter and read the labels of the
liquor bottles. There were ten different brands but the same ones that
were lined up at every party.

She sat down on one of the bar stools that was used in the kitchen nook
and, crossing her legs, snagged her nylons on the underside roughness of
the counter. So easily her earlier tedious grooming had been marred. The
washing, the filling, the brushing, the ironing, and the final touch of
perfume, the warm smothering final cloaking and she had been ready. Now
it had been made imperfect by the bare skin under the torn nylon fibers.

Her husband, if he saw her now, would say she was drunk. She filled her
glass with Scotch and water and took two ice cubes from the ice bucket.
Then she sat alone with her imperfection.'

She removed a bobby pin from her hair, feeling the strands stiffened by
hair spray. She used to wear it in a ponytail in high school and had
loved the feel of it as it bounced off her shoulders when she walked.
She was reminded of the graduation party that most of tonight's people
had been at. But Barbara wasn't here tonight. She had gone away to
school after the graduation party. No one had really liked Barbara. She
was too fat, her hair was always dirty, and dirt was under her
fingernails in the summer because she liked to garden, really liked to.
Her fingernails were even dirty at the graduation party. No one had seen
her since but she probably wasn't at a party tonight and her fingernails
were probably dirty.

Looking down at the counter, she saw what she had done. She had
scratched her and her husband's names into the counter's wax with the
bobby pin. She tried to erase them but knew it didn't really matter. No
one would ever notice. The light had to be hitting the counter just
right before the scratches were even visible.

Suddenly she cupped her hands around the sweating glass as she heard
someone coming up the steps. As she waited, she pressed the side of the
glass against her lips, licking the moisture from the smooth roundness.
She could smell Ruth's lily-of-the-valley perfume. She took the last
cigarette from the pack, sliding her fingers along the whiteness. She
forced her eyes to look at the scratched-in names, fearing some hand
would suddenly fill them with wax if she turned to look at Ruth.

Ruth spoke her name for she was a woman comfortable only with noise.

She liked the sound of Ruth, a woman of God. ``Are the walls sky blue,
Ruth?''

``Yes. Yes, Karen, they are.'' Ruth was moving the dirty glasses from
the counter and rinsing them in the sink.

``You're wrong. They're robin's egg blue.''

``All right. What's the difference? You really are drunk.''

She could feel Ruth stiffen and even Ruth's name became a growling sound
deep in her throat as she tried to call to her. Instead of calling, she
dropped her head on her arms, admitting defeat; Ruth was much stronger
in her unyielding certainty.

Ruth wiped her hands on a paper towel and moved above her to stroke her
forehead, all the time murmuring sounds. Ruth encouraged her to cry, the
woman's remedy. But she wasn't able to cry; she hadn't cried for a long
time. And Ruth had once called her cold or maybe she had chosen the word
``restrained.'' It was when they were at a movie when they were freshmen
together. Ruth, at that time, had licked popcorn salt from her lips and
wiped tears from her eyes with a green Kleenex. And all the time she
kept saying that Karen should cry. Somehow, Karen felt she'd let her
down, then and now.

Ruth handed her a Kleenex, commanding her in a quiet voice, to wipe the
sweat-smeared makeup from her face. In her defeat, she obeyed.

Ruth was grasping at words and reasons, anything to fill the silence.
``It's okay, Karen. It's not a party unless someone gets drunk. Eddie's
not mad. He asked me to come and see if you were all right.''

There was no answer from Karen and Ruth continued hurriedly, probing
softly. ``Eddie's worried about you. Says you've been depressed a lot
lately. Is something wrong, Karen? You know I'd like to help.''

Karen lifted her head, looked at her best friend, and wished Ruth could
help.

``Is it the talk about kids, Karen? Is that it?''

When Karen just continued to look at her, she relaxed with the reasons
that she could accept. ``We forget about you sometimes, Karen. I'm
sorry. We're all sorry. Your day will come.'' She chuckled to herself
and added, ``Besides, you'll have the benefit of our experience. We can
tell you what works with kids and what doesn't.''

Karen smiled, realizing that Ruth needed her agreement. She finished the
Scotch and water and wiped her lips with the side of her finger,
allowing it to feel the smooth pinkness, drawing every part of herself
inward to the redness, slowly.

``I'm okay, Ruth,'' she said quietly. ``I'm really okay.''

Ruth put her hand on Karen's elbow and although Karen couldn't feel the
touch, she saw the hand and heard Ruth's voice droning on. She thought
how soothing Ruth's voice was and wondered if the words really mattered.
Sometimes they seemed powerful, but tonight they'd lost their force; she
was too tired. Ed's eyes had been drawn to Ruth's assured whiteness and
Karen found herself hating her own seasonally tanned skin, her firm
body. And yet she couldn't cry. And Ruth was waiting.

``We're alike. Good God, but we are.'' Karen wasn't sure if she'd spoken
the words aloud until Ruth responded.

``Of course; we've been friends a long time. Come on now. Let's join the
party.''

Karen excused herself to repair her make-up but waited in the kitchen
listening until Ruth's sound on the steps was gone. She ran her finger
over the scratches on the counter, drank another glass of Scotch and
water, and started downstairs.

The screen door was still open. It had started to rain and for a brief
moment she thought about her open windows and the drapes that would get
water-spotted.

Then she opened the door and walked into the garden where the flowers
all looked black at night. And she stood among the flowers, feeling the
mascara streak her cheek with darkness, feeling the black rain soak into
her skin, knowing that someone would come to get her soon.



\cleardoublepage
\chapter{The Picnic}

She sits at the window watching for him. Still, she isn't looking when
he swerves up the drive­way, going too fast, always too fast.

Her eyes stay fastened on the little kid next door, wondering if he's
the one tearing up her flowerbeds. She's never caught him at it. Never
seen him pull up the marigolds, the asters, the bache­lor buttons, the
zinnias, leaving the roots exposed to the summer sun.

She wipes sweat from her forehead, feels it escaping beneath her
fingers, trickling down her temples, her neck, and pooling between her
breasts. Not a breeze anywhere. Even the fan malfunc­tions during this
heat wave.

``Why didn't you keep the shades down?'' Jim shouts from somewhere near
the back of the house, his voice rushing ahead of him. When he reaches
her, he leans down and tilts her chin up, kiss­ing her lips damp with
sweat.

She pushes away from him. ``Too hot,'' she says. Last year at about this
time, she remembers how her mother fanned her as she slipped on the
wedding dress, fixed the veil on her head before they left for the
church. It was hot, then, too.

``That's the kid digging up my flowers,'' she says, pointing to the
little boy playing in the sprin­kler. ``It's not right digging up my
flowers like that.''

``No, it's not right,'' he says. He straightens the picture on the wall,
picks up the glass filled with melted ice, wipes dust from the table, a
flurry of motion as if he's trying to create a breeze that will cool
her.

``Ever notice how when it gets this hot, it's like you're looking at
everything through a telescope? Things move closer and lose their
shape,'' she says and looks past him as he stands at the window.

He leans his hands on the window sill, notices the paint peeling around
the outside frame. If he has time this summer, he plans to strip the
paint to the bare wood and start over.

``When I was a kid, my old man always painted the house white,'' Jim
says. ``Every year he touched it up. Said we had the whitest house on
the block. Made him proud just to look at it.''

``Not white, she says. ``I don't want a white house.''

``What do you want?'' he asks.

``Stop shouting. I hate it when you shout,'' she says. ``It's this heat.
Makes everyone edgy.''

``It's not the heat. It's you. It's me.''

``Stop it,'' she says as he pulls her from the chair, toward him. ``I
can't breathe.''

``Let's go on a picnic,'' Jim says, still holding onto her. ``Out to
Proud Lake. Like we used to do all the time before anything happened.''

``Before what happened?'' she shouts. ``Go ahead, say it.''

``Just before,'' he says, already moving toward the kitchen.

She hears the familiar kitchen sounds. He's opening and closing
cupboards, rummaging in the freezer.

On the drive out, he tries to talk to her but she turns up the radio and
rests her head against the window. He pulls down the visor against the
sun. Pools of water on the road form and disappear as he speeds toward
them.

He pulls into a side road. It's hard to see anything with the dust
they're stirring up. The pines lining the road are covered with dust.
Even the sparrows seem gray, weighted down. But around a curve, she sees
a shack and past that a canal leading to the lake.

The woman at the shack tells them they only have an hour before dark and
that the dark comes suddenly. Warns them how easy it is to get lost so
the y need to watch for the signs that will guide them back. Loans them
a flashlight to read the signs, just in case. She tells about a couple
that got lost one foggy night years ago. Says no one ever found them or
the canoe. But some people say they see them during Harvest Moon.

``No,'' the woman says as she swats at a mosquito that settles on her
arm. ``I never seen them.'' She asks them to sign the canoe rental form.
``But sometimes maybe I hear them. Maybe. But no way to be sure. Just
hear canoe paddles and something like singing.'' She pauses, tilts her
head, listening. ``But maybe it's nothing but the water against the
shore and the wind in the trees. Hard to tell.''

The woman leads them to the canoe, waits for them to get settled before
handing them the flashlight and the paddles. ``Remember,'' the woman
says, ``the canals aren't deep, only about three feet. It's the lake you
got to watch. Over your head. About fifteen feet, no one knows how deep
in the middle,'' she warns as she shoves the canoe off the slanted
landing ramp.

Paddling hard to move through the canal, Jim says, ``It's okay,
Sweetheart.'' His breath seems measured by the paddle strokes. ``You
don't need to paddle.''

She faces forward, her back to him, and holds the paddle tightly across
her knees. She watches for submerged logs. ``Jim,'' she shouts, ``Watch
out. There's a log on the right. A big one.''

``What?'' he shouts back, too loud, almost like a gunshot. ``Can't hear
you.''

``Jim,'' she shouts louder. But he still can't hear her and the canoe
bumps against the log, rocking and turning the canoe slightly. She sits
rigid, certain it will tip over. Surprised when it straightens out.

They move out onto the lake and watch the fishermen standing on the
shore, casting their lines and reeling them in when they hook a trout.
Near a small island, they spot geese honking and arching their necks
when the canoe moves too close to shore.

``Protecting their nests,'' Jim says. ``But they don't seem afraid of
people. Amazing isn't it how they get used to people taking over their
lake in the summer.''

``Getting hungry?'' Jim asks. But she's not hungry. Maybe the picnic can
wait until another day.

As the sun starts setting, she urges Jim to go back. But he wants to
watch the sun set, feel the cooler breezes wafting over the lake. The
fishermen are fading on the shore. She's not sure if they've gone home
or she just can't see them in the dusk.

``Peaceful, isn't it?'' Jim says, letting the canoe drift. ``It's like
we're the only people here. And you don't feel the heat on the lake.''

She starts paddling. ``Let's at least get back to the canal. The lake's
too deep. What if we can't find the signs?''

``We'll find them.''

``You're always so sure of everything,'' she says, paddling harder,
rocking the canoe form side to side.

They make it to the canal but she's not sure it's the right one. Her arm
muscles hurt, not used to the paddling. But she doesn't stop. A bird
swoops low across the canal, skimming her hair and she screams, letting
go of the paddle.

``It's okay,'' he assures her, grabbing for her paddle and missing it as
it moves past him. He starts paddling again. He breathes hard as he
fights the wind pushing them backward.

``It's not okay,'' she says. They don't seem to be moving. The water
slaps against the canoe, a hollow sound like a ripe melon.

Picking up the flashlight beside her, she directs the beam over the
side, squinting hard to see. But the water, like wrinkled skin, shows
only its surface.

Over there, on the outer circle of the light, she sees her own face,
breaking in the waves. Another face forms behind hers. She watches,
quiet waiting, a fisherman letting out the line. Suddenly her mother's
eyes, shiny as minnows, rise to the surface.

Even now her mother's eyes spawn secrets, swimming there just out of
reach. Her mother's eyes have looked like that before---at birthday
parties before presents were unwrapped, at her wedding before Jim put
the ring on her finger.

Jim reaches out for her as she stands, but she's too far away. ``Sit
down,'' he says as he grasps each side of the canoe and moves forward,
stepping over the metal seats in the middle. She pulls back. The balance
is off. The canoe tips, flipping them both into the canal.

Jim reaches her side, puts his arm around her shoulders and pulls her
upright. ``It's okay. Just put your feet down. It's shallow here.''

``Don't move,'' she whispers. ``She's here. Do you see her over there?''
she asks, staring hard trying to find the place again.

``It's just the moon reflecting off the water,'' he says.

``No, it's her. I know it's her,'' she says. ``Find the flashlight and
I'll prove it.''

``It's gone,'' he tells her. ``The flashlight is gone. Your mother is
gone.''

She remembers how she and her mother had shared many secrets---a new
dress she couldn't tell her dad about, an extra \$50 tucked in her
jacket pocket when she went off to college, the first yellow crocus
under the melting snow. But there were some secrets her mother wouldn't
share.

``Some things you just have to find out for yourself, like a first
kiss.'' She leans back, moving her arms through water, making angel
wings. ``That's what my mother used to say.''

``Do you think they're really here on the lake?'' she asks, spitting the
canal water from her mouth.

``Who?''

``That couple who drowned out here when their canoe tipped over.'' She
caresses the side of the canoe as she speaks, cutting her finger on the
rough aluminum edge. She looks at the cut. ``Do you think maybe they
canoe every night and only let themselves be seen at Harvest Moon?''

``Let me see that, ``Jim says, reaching for her hand. ``Just a small
cut. I'll put something on it when we get to the car.'' He turns the
canoe right side up. ``It's just a story. A story that woman likes to
tell people like us.''

``Oh, Jim, it shouldn't have happened.''

``What?'' he whispers. ``What shouldn't have happened?''

``She shouldn't have died. Not while I was gone,'' she says so quietly
that he's not sure he hears it. ``She wasn't even sick.''

They listen to the water lapping against the canoe, against the shore.
The night sounds.

``I mean she could have waited until we got back from the honeymoon.
Maybe I could have done something. Maybe we should have waited to get
married.'' She tilts her head back to look at him, her hair floating
like seaweed atop the water. ``Maybe she knew and just didn't tell me.''

``She would have told you,'' he says.

The water is cool, cold. Too cold now that the sun has gone down.

``It was hot that day, really hot,'' she says. ``Didn't it break a
record of some kind?''

They move toward the shore, pulling the canoe beside them. They can see
lights on the land­ing ramp ahead. Not much further.

``I miss her,'' she says, remembering how everyone tells her she has her
mother's eyes, dark as garden soil.

``I know,'' Jim says.

When they pull the canoe up to the landing, the woman runs out of the
shack, waving at them. ``I was so worried about you two,'' she says,
``What happened?''

They're both too tired to explain. They just motion toward the canoe.
When they try to pay her for the lost paddles, she refuses, saying,
``Wouldn't do to have two young couples lost on the lake. You would have
had to wait for the Hunter's Moon,'' she jokes. ``And Hunter's Moon
isn't the time to be out on the lake.''



\cleardoublepage
\chapter{The Puppet Show}

My life is filled with scenes. The puppeteer improvises, sticks her hand
inside a memory and wiggles it until it's center stage.

\textbf{Act 1, Scene 1}

\emph{As a Daughter}

My mother and I stand in line at Hudson's. I am nine years old, too old
for the thermal under­shirts she holds in her hand.

``The ones from last year don't fit,'' my mother says to the woman
behind her in line. She holds the shirt to my chest and turns me toward
the woman.

``Do you think if I take a tuck in the shoulders, they'll be okay?''
Mother asks the woman.

The woman nods. The little girl holding her hand giggles. I turn to the
front of the line, my braids heavy against my back.

``She's the tallest in her class. If I didn't take tucks in things, we'd
go broke.''

``Mother.'' A growl in the back of my throat. She doesn't hear.

``Not sick all the time either like others in her class.''

The line moves forward.

``Smart, too,'' Mother says over her shoulder. ``Show them,
Sweetheart.''

The saleslady rings up the prices.

``Don't total it,'' Mother says. She turns to me. ``Okay, what's the
total?'' Everybody looks at me, waits.

I stand, my shoulders slumped.

``Well?''

``I don't know,'' I whisper.

``Don't be shy, Sweetheart.'' She looks at the woman behind her, then at
the saleslady. ``She always gets A's in arithmetic.''

Blood in my mouth. My tongue hot, thick between my molars.

Mother's fingertips nudge my shoulder. ``Give us the answer.'' The balls
of her fingers press to the bone.

``\$7.29,'' I say.

The saleslady totals.

``See,'' Mother says, moving her fingers inside her coin purse. ``Right,
including tax. She never misses.''

The next day at school I go into the girls' lavatory and pull off the
sweater my mother knitted, pull off the undershirt. After wadding it
into a ball, I put it into my lunch box, under the peanut butter
sandwich my mother packed, under the apple and celery sticks.

\textbf{Act 1, Scene 2}

\emph{As a Daughter}

Twelfth grade graduation. My mother in the audience. Her green cotton
dress tight across her breasts, her hair frizzy from last week's
permanent. She waves as I walk to the lectern to give my valedictorian
speech.

``Good afternoon,'' I say.

``Good,'' my mother mouths, one word behind me.

At home, fanning herself with my graduation cap. She sits on the front
porch swing and talks to Julie's mother, Mrs. Brandon. I pour lemonade
into their glasses.

``She's got a scholarship to Antioch,'' Mother says as she unbuttons her
top two buttons.

``Some name for a college. Sounds like a vegetable.'' Mrs. Brandon holds
the cool glass to her temple. Looking at me, she says, ``Thought you'd
be getting ready for the prom. Julie's home taking her beauty nap.''

I shut the screen door quietly behind me and pick up my yearbook from
the hall table.

Their voices come through the holes in the screen.

``That Julie. Broke her date with Jim Parker so she could go with Ronald
Thompson.''

The swing cracks. ``This heat gets to me. Seems to make my blood
pressure worse.''

``That's because you eat too much salt. It bloats you up.''

I stare at Julie's picture in the yearbook, trace her oval face with my
finger.

``When my daughter's a doctor, she can take care of me. Won't have to go
to that quack anymore.''

``If she wants to go to the prom, bet Julie could get Jim Parker to take
her.''

I don't listen to my mother's answer. I concentrate on drawing a
handlebar mustache under Julie's nose. Ink freckles splatter against the
white face. The curl goes in her hair, solid blue beneath the ears.

\textbf{Act 2, Scene 1}

\emph{As a Mother}

I spend hours making my daughter's ballet costume. It is perfect: each
pink layer of net stitched carefully; the sequins sewed one by one to
the bodice. No last-minute cuts or tucks.

Dress rehearsal. My daughter sits beside me in the darkened auditorium.
Two rows up with the rest of the class.

``Why don't you go sit next to Christine?'' I know all the girl's names.

``Don't want to,'' she says, rummaging in her school bag.

``What are you looking for?''

``Nothing.''

Finally she pulls out a flashlight and book. A permission slip marks the
book at the halfway point. She opens the book and begins to read.

``What's this for?'' I take the permission slip. ``A hayride. I used to
really like those,'' I lie.

She doesn't say anything.

``You look silly holding that flashlight. What if your teacher looks
back here?''

``I don't care.'' She turns the page.

``You should.''

I reach for the flashlight; she holds it tight. I pull. It falls to the
carpeted floor, shatters. The light goes out.

Her teacher turns and waves her forward to join the class. As they mount
the stage, their legs are tinged with green until the stage lights shift
color. Now, white and pink porcelain figures, stiff in their tutus, arms
arched above their heads. Twirl like jewelry box ballerinas.

My daughter, at the end, looks into the spotlight, freezes there while
the rest turn right and exit.

Locked in her seatbelt in the car, my daughter finishes a chapter in the
book.

At the first light, I straighten her tiara, say, ``Everyone goofs in
rehearsal. You'll do fine tomorrow night.''

``I know.''

``There's nothing to be nervous about. You were the prettiest one up
there.''

``I'm not nervous.''

``You don't care do you?''

She starts another chapter.

``You don't care that Christina and the others laugh.''

She doesn't look up from the page. ``It's only a dumb ballet recital.''

``But the others care.''

``Yeah,'' she says.

\textbf{Act 2, Scene 2}

\emph{As a Mother}

A puppet show at a birthday party. A makeshift stage: two chairs and a
cardboard box draped with black velvet. The puppeteer is concealed.

On stage: a butterfly: a boy chases it with a net. Sequined wings snag
on the black velvet. One of the mothers sets her coffee cup on the
Formica end table and moves forward. By then, the butterfly is free.

I watch my daughter. She looks straight ahead, unmoving, trying not to
respond to the baby who has its hand tangled in her long hair. Maybe she
doesn't feel the groping hand. Again and again, she breathes deeply, her
chest straining against dotted Swiss ruffles. Does she smell the musty
silk? Does she dissect the butterfly, pulling the sequins off one by
one?

Whatever she is doing, I will never know. She doesn't even flinch when
the baby wraps fine, blonde strands around its finger, when he sticks
them in his mouth and sucks.

Other children scoot closer to the stage, try to touch the puppets.

My daughter sits alone in the group, untouched by them, untouched by me.
The space around her is clean like her room at home. She changes the
sheets, makes the bed, vacuums the floor, and locks her diary with the
key she wears on a gold chain around her neck.

The butterfly gives way to a wicked witch, a flying fish, and a dancing
girl in feathers. Still, she watches. I wonder what I have done to her
or not done. I get tired of trying to explain her to myself, to others.

``You're not listening,'' Peggy says as she jabs me in the ribs.

I apologize and readjust my face.

``This is some party. And to think they almost cancelled it.'' Peggy
wipes caked lipstick from the corners of her mouth.

Peggy knows what is going on. I wonder if she has one of those
electronic gadgets that picks up conversations ten miles away.

``Ricky stuck a tape recorder in Katherine and Don's room.'' She giggles
and covers her mouth when her daughter turns around and glares.

``Ricky played it for all the neighbor kids. Didn't Sarah tell you about
it?''

I shake my head.

Peggy nibbles at a corn chip. ``Wish I could have heard it.'' She hands
me the bowl.

I wish she would go in the kitchen to refill her coffee. My head aches
from the noise and smoke.

``If only my kids could keep their mouths shut like Sarah.''

Peggy's two girls are poking each other, each jab harder than the one
before. For a brief moment, I wish they were mine. Then I look at my
daughter. She could be a color photograph of herself taken a few minutes
earlier. She hasn't moved and her expression hasn't changed. I want to
shake her.

``I'm ready for them to go back to school and they've only been out
three weeks.''

``Yeah, I know what you mean.''

I look at my daughter and feel sad. My agreement is based on my two
older sons' behavior, not hers.

Suddenly I want Peggy to keep talking.

But the show is over. Older children push the younger ones from in front
of the stage, grab at the puppets. The black velvet is pulled from the
chairs. The stage is gone. The puppeteer sits two of the smaller
children in her lap and slips the cowboy over one girl's hand, the cow
over the other's.

The birthday boy steps on my toe in his rush toward the kitchen. A group
of children follow, each wanting to be the first in line for ice cream
and cake.

My daughter stands slowly, unaware of the movement around her. I call to
her and she turns, startled to hear her name.

As I put my arm around her shoulders, I ask her how she liked the show.

``Fine. Just fine.''

I know she's thinking of something else; I've asked a stupid question.

``Which character did you like best?'' Again, it's not what I want to
ask, but at least I try.

``The princess,'' she answers too quickly, saying what a nine-year-old
might say.

I've forgotten the princess; I suspect she doesn't even exist, but I
smile, play along with her. She reaches for my hand.

When the crowd around the stage thins out, I ask her if she wants to
play with one of the puppets.

``I don't need to,'' she answers.

I wonder what she does need. At her age, I needed Richie Beard to stop
taping ``kick me'' signs on my back every time he went to the pencil
sharpener. I needed my mother to stop asking Ellen Farber in every time
she knocked on the door. I wanted my mother to quit straightening my
under­wear drawer and throwing out my pencils without erasers.

I look around for Peggy and her children. My daughter plays with them
sometimes and I'd like to see them playing together now. But I can't
find Peggy and I must avoid looking into my daughter's face.

My hand squeezes hers tightly. There are only two children playing with
the puppets now; one is behind the stage talking for the butterfly. High
pitch, too shrill.

``Try the puppets. Maybe the princess will be fun,'' I mock her, ashamed
for both of us.''

She tries to withdraw her hand but I'm ready for her. No escape. She
begins to cry, silently, but it's too late for tears.

We are by the stage now, alone; the two boys lose interest in
make-believe. They turn to more substantial things like ice cream and
the swing set outside. Over the mothers' chatter, the noise of chil­dren
playing.

Grabbing the butterfly, I force her hand into its gauzy innards. The
sequins sparkle as she struggles against me. Her wrist no more than a
butterfly wing in my hand, fluttering against the net. Tighter. The
fluttering stops. Her bones knobby, hard against my skin. She slumps
against my hip and I am ashamed. When I release her wrist, she stands
quietly, looking at me.

She is not able to understand that I fight myself and I do not try to
explain. She is not aware of the lies I tell for her and for me.

``Would you like some cake?'' I ask her, smoothing her hair.

She avoids responding directly but moves nearer to me. ``Let's go
home.''

I find myself perspiring, getting angry again at this girl who won't
try, who has inherited my weakness.





\cleardoublepage
\chapter{Queen Anne's Lace and Dandelions}

If my mother finds out what I'm doing, she'll have conniptions. (I can
just hear her. ``I don't work all day so you can ride the bus and go
alone to places like that.'' The sad look.) She doesn't approve of my
going to the shopping malls either, but says since I'm with friends,
nothing really bad can happen. But after a winter of bumming around
shopping malls, I need something to get me out of my rut.

I first notice it in the ``Where to Go, What to Do'' column of the
newspaper. Maybe it's the strange-sounding food--- baklava, spinach pie,
grape leaves. Or the picture of the belly dancer with all her spangles.
Whatever it is, I know I have to go to Greektown.

Sometimes things hit me like that. No matter what, I have to do them.
Nothing else matters. Not even my mother's voice or her finger as she
points to a newspaper article about a mugging on our old street. We
moved out to Westland when I was in the first grade. But I do remember
my mother saying she'd never go back to Detroit, not even if her life
depended on it. And she never has, not even to visit old friends.

But Detroit doesn't worry me. The only thing I worry about is catching
the express bus. I keep checking the paper where I've written the
directions. The lady at the bus company kept muttering something about
Grand Circus. I kept telling her I wanted to go to Greektown, not a
circus. When I asked her to repeat the directions for the third time,
she hung up. Luckily, I got a nicer lady the sec­ond time I called. She
even spelled out all the street names. (Mother voice: Don't talk to
strangers.'')

The bus is crowded and the man next to me keeps hitting my shoulder with
his newspaper every time he turns a page. I clear my throat and my voice
sounds squeaky when I ask him if we're almost downtown. He looks over
his glasses and shakes his head. I ask him if he's ever been to
Greek­town. He ignores me.

To make sure the bus driver knows, I pull the cord twice and get off in
front of Hudson's. I remember coming here with my mother to see Santa
Claus and getting lost behind some racks of winter coats. It's tough
when you're small and can't see over things.

I make my two transfers okay and walk over three blocks like the woman
at the bus company told me. And there it is: Greektown. Not that I
expected a whole lot, but I thought there'd be some­thing besides a
bunch of buildings. Maybe the dancers only come out at night. Maybe
they're all inside someplace drinking wine, waiting.

At least I can eat some of the food. I sit at a table in a hot bakery
and eat baklava, the honey dripping through my fingers. The clerk, who's
about my age, asks me if I'm from out-of-town, says they get a lot of
tourists in the summer. I tell her I am, that I'm from Texas, that my
parents are next door looking at vases. (Mother voice: ``Is the truth so
bad?'')

The clerk says her name is Maria, says I don't sound like I'm from
Texas. I tell her I just moved there, and I try to slow down my speech,
draw out my vowels. I usually pretend I'm in a class play at times like
this. I sit straighter in the chair, dab at the corner of my mouth with
my napkin---a real southern belle. She says she used to have a Greek
accent because her parents still speak Greek at home. Says she went to
speech therapy to get rid of it. I tell her that if I had a Greek
accent, I'd never want to lose it. She laughs.

I'm so busy looking at the Greek crosses and trying on blouses, I forget
the time. By the time I remember, it's almost too late. Even though I
hate rushing things, I run to catch the buses, to make my connections.

I'm all out of breath by the time my mother walks in the door. Maybe
she's too tired to ask me much or to notice the Texas drawl I've been
practicing on people all day.

I tell her the gang and I went shopping. She says I should do something
more constructive with my time. She rattles off a list of chores for me
to do tomorrow and I write them down. I can do most of them tonight
while she takes a nap. That'll leave the day free for Greektown. There's
something about the place that's gotten into my blood.

After a week of hanging around Trapper's Alley, I'm now one of the
regular crazies. We're all here today. Prometheus with his magic torch
that joins glass and metal. Apollo drawing her circles and filling them
in with yellows and oranges. Mr. Antilopolis selling popcorn from his
red and white cart. The old woman playing the piano, drinking beer from
a paper cup. And all the rest who open their suitcases and sell coin
bracelets to the tourists who come to look and sometimes buy.

I'm sitting on the concrete watching the old woman play when I first
notice him. His eyes move from my bare toes, to my red halter, my lips,
and back again. He's not like other guys I know, not with the eagle
tattoo on his right arm, the wavy hair, the dark eyes that pin me
against the brick wall. I tug at my white shorts and smile, wondering if
I remembered to put on lipstick. When he flips over the wooden Coke
crate and sits next to me, I can't think of anything to say. (Mother
whispers in my ear: ``Ignore him.'')

Doesn't matter, though, because he's full of wisecracks about the old
woman. Says she danced so long in her white satin dress it turned
yellow. Says those dangling rhinestone earrings gave her a tin ear. Says
lots of things that make me laugh.

Suddenly, the old woman stops playing and pulls on elbow-length, white
gloves. She walks over to us and screams in Greek at the boy beside me.
He screams back in Greek. She pokes a gloved finger against his chest
and says more words before spitting on the ground beside him. He gets up
and kicks the Coke crate. He goes, still shouting.

I can't understand any of it. She must really be crazy. We weren't doing
anything. But she stands above me, muttering. She crosses herself and
pulls me up. She's taller than I thought and big­ger. My stomach feels
funny. Probably because I didn't eat breakfast. (Mother: ``Don't I beg
you to eat right?'') After all, no old woman can scare me. Not even when
she reminds me of one of the witches in \emph{Macbeth}. I picture her
stirring her brew, chanting the magic words.

With my free hand, I try to pry her fingers from my arm. I shout at her
to let me go. I shout to Mr. Antilopolis for help but he shakes his
head. Prometheus doesn't even lift his goggles. (Mother rings her
hands.)

The old woman pushes me down on the piano bench and sits beside me. I
stand up. She pulls me down again. My legs are warm against her satin
dress.

She asks me if I play and, when I shake my head, she says, ``I will
teach you.''

Up close, she looks at least ninety. As she slowly peels down the gloves
and pulls at the fingers one by one, I tell her I don't want to learn. I
only want to talk to the guy she chased away.

In English, the old woman says the boy is no good. Says she knows him
like she'd know her own son if she had one. Says Greek girls won't be
seen with him. That one is dangerous. Knew it when he had that eagle put
on, she says more to herself than to me. Says not to ever trust boys
with tattoos. Bad sign, she says. (Mother nods her head in agreement.)

It's easy to peg the old woman. Probably a stripper who read palms and
played the piano dur­ing intermissions years ago. Sometimes I think I
have E.S.P. the way I can just guess people's jobs and problems. I do it
to pass the time on the buss or in my fourth-hour Geometry class.

``Like this,'' the old woman says, arching her hands and stretching her
fingers. ``Get the feel of it. Let it grab you here,'' and she taps her
chest. ``You try it.''

I shake my hands, stand up and bow to Mr. Antilopolis. Then, with my
fists, I pound at the keys. The sound is too loud, even for me. (Mother
shouts: ``You want to go deaf?'')

``So much anger,'' the old woman says, taking a drink from her paper
cup, rolling it around in her mouth. ``In the old country, there was a
story about a girl like you. She breathed in everyone else's anger; it
was the gods' way of testing her. But she did not know how to get it all
out again. She tried coughing, hanging upside down from the rafters in
the barn. She tried kissing and drinking buckets of water to drown the
anger. She tried to cry it out until there were no more tears. But she
couldn't stop the anger from growing. Finally, she grew so big with all
her anger that she floated into the sky and became a thundercloud.''

Old people telling stories always makes me laugh. My grandma, before she
died last year, used to talk about the witch woman who made cures and
love potions. Grandma said she drank something once, made special for
her by the witch woman and that's how she met my grandpa. Even my mother
would laugh when she told that one.

I picture myself getting fatter and fatter. My clothes split at the
seams. A fat thundercloud with blonde hair. I rain on a
watermelon-eating contest. I laugh so hard I can't catch my breath.

The old woman passes me her paper cup and I take a drink. I gag and spit
the warm beer on the concrete.

``Try again,'' the old woman says, lifting my hands to the keys.
``Easy,'' she urges as I press the ivory. ``That's it.''

I sway with the old woman as her shoulder leans into mine and she begins
to sing. I play all the wrong notes. It sounds lousy. But I've got the
beat and sing ``Ninety-Nine Barrels of Beer on the Wall.'' I'm surprised
to hear her join in. Then Mr. Antilopolis, with a deep voice that loses
count of the barrels.

Tourists stop to watch, to sing. I get to laughing and hugging the old
woman. The tourists drop coins in the paper cup. Beer splashes against
the sides. Watchers move on, still humming the tune.

Taking the coins from the cup, the old woman wipes each one on a
Kleenex, drops it into her satin shoe, and finishes the beer in the cup.

She plays songs I've never heard. Most of them are about lost lovers,
but the one I like best is ``Angels of the Wind.'' It's about two young
lovers who run away together and, when they die, they ride the wind
across the canyons and sing to each other.

``I know,'' the old woman says. ``Watch.'' She alternates her feet on
the pedals and plays two black keys, four white keys. Over and over.
Faster and faster. ``You try it,'' she says.

I scoot over so my feet can reach the pedals and hold my hands like she
showed me earlier. The pattern is easy but sometimes I get confused and
hit three black keys. (Mother, proudly: ``Give it your best.'')

The old woman bobs her head until I've got the rhythm. Stands. Takes a
lace-edged hand­kerchief from her low-cut dress. Waves the handkerchief
in circles above her head. Twirls.

I keep playing. The old woman takes Mr. Antilopolis' hand, making him
drop a bag of pop­corn he's handing to a tourist.

``Don't worry. The pigeons will eat it,'' the old woman says.

Apollo wipes her hands on her smock and takes Mr. Antilopolis' other
hand. The growing line grabs a man from the crowd. Then Prometheus, who
pushes up his goggles and turns off the torch. I play faster. The line
swerves in between the tourists, picks up more people. The old woman
waves the handkerchief and sings. Directs them all to sing. Leads up one
side of the alley, down the other. Someone knocks over a garbage can.

The old woman collapses next to me on the piano bench, wipes her
forehead with the hand­kerchief, tucks it back inside her dress.

``Good. Very good,'' the old woman says. Her rhinestone earrings shine.
Maybe they're really diamonds that some dark-haired man gave her when
she was a stripper.

She takes my hand like a little kid's and leads me through an alley
door, into a bar, up to the counter where an old man is washing glasses.
She sits on a tall stool and motions toward the one next to her. I sit
down.

``What'll it be, diamond girl?'' the old man asks.

Taking money from her shoe, she orders a beer for herself and a Coke for
me. She's blushing.

The old man sets our drinks on the counter, picks up her hand, raises it
to his lips. To me he says, ``This one's a beauty, eh?''

They talk in Greek and I look around at the old men sitting at the
tables, playing checkers, talking. Reminds me of the time I visited my
great aunt in an old folks' home. Kind of depressing to think of all the
old people in the world just sitting around, waiting to die. (Mother:
``Have you no respect?'')

``Where are all the kids?'' I ask the old woman.

``Gone,'' says the old woman. Says the young have to leave sometime. In
the end, they come back, she says.

I tell the old woman I have to go. My mother will ask me tons of
questions if I'm not there when she gets home.

The old men do not look up from their checkers and the old woman is
talking to the man behind the bar. No one says goodbye.

On the bus ride home, I watch the street names, trying to remember where
I used to live. Three times, I play the game where I have to find all
the letters in the alphabet on street signs. Each time, it takes me
longer, so I start counting out-of-town license plates. I look down into
cars and try to guess where people are going. Sometimes, if there are
kids in the back seat, I wave and they wave back.

At a light, I see a guy lean down and kiss the girl beside him. I
pretend I'm the girl Jim used to kiss me like that this spring. But we
broke up just as school ended and it was too late to find a boyfriend
for the summer.

When I walk down our street, I always look toward the open screen doors
and picture windows. If I see someone moving around in the house, I feel
good.

I unlock our front door and start the potatoes and pork chops my mother
has left out for dinner. I'm in the shower by the time my mother gets
home. She tells me to hurry up; the food's on the table.

Buttering her bread, she tells me about the woman who bought
tiger-striped cotton to make her husband pajamas. ``Can you imagine,''
mother says, ``what her husband will look like in four yards of tiger
stripes?''

Today is one of her good days. By now, I can tell the difference. On bad
days, she talks about how her legs hurt, how her boss schedules her for
weekends because he has something against her, how the new girl messed
up a whole bolt of velvet, how she's about to quit and find another job.

But tonight, she says she might be able to get me a part-time job there
this winter. That way I can have some extra money for school clothes.
Says she knows how important that is to a girl.

We're eating the chocolate chip cookies we made Sunday night when I
remember all those fancy things my mother used to make for dinner before
my father went away. I remember how I used to hate the stuff but she'd
smile and say, ``Try it. You'll never know unless you try it.''

Sometimes I worry about my mother because she doesn't smile like that
much anymore. She doesn't clip food recipes from the paper and doesn't
even take cookbooks out of the library.

I get her old cookbook from the shelf in the kitchen and we look through
it for Greek recipes.

``How about kidney pie?'' Mother asks.

I make dying sounds. ``Eat kidneys? That's disgusting,'' I say.

While we're doing dishes, I tell my mother she should go out more, have
some fun. She says I sound like a mother. She lifts her hand out of the
dishpan and blows soap bubbles my way. I put my hand in the water and
flick my fingers at her. She dips in both her hands and gets me back.

While we're watching television, the detective tries to trap the killer
in a disco bar. My mother says the dancers are more disgusting than
kidney pie. I get up and do a few steps, tell her I'll show her how. She
says she's too old. Her body won't move that way anymore. I tell her she
really should get out more. She says it's my bedtime.

It's really a hot day and all the way to Greektown, I worry about my
mother, wonder if she's happy. Sometimes I pretend my father comes back
and we're all at the dinner table, laughing.

I sit on the piano bench next to the old woman who keeps wiping the
sweat from her eyes. Finally, she takes me to her apartment where she
has a fan. She clicks it on and stands in front of it, lifting her satin
dress to cool her legs.

The old woman tells me to get a Coke from the fridge and to bring her a
beer. In the fridge, I find three cans of beer, a hunk of cheese, a box
of cereal, wilted lettuce. No Coke.

She hollers from the other room for me to watch out for the dish on the
floor. Says she likes to put something in it for the stray cats now and
then. Says she owned a parakeet once but it flew out a window she'd
forgotten to close. That's why there's a bird cage in the living room
with no bird, she says. Says to help myself to something to eat if I'm
hungry. I tell her I'm not hungry and pour her beer into a glass.
(Mother: ``Make sure it's clean.'') I fill a jelly glass with water.

The old woman takes the beer and looks at my water. ``Watching your
figure, I see.'' Rubbing her hands over her wide hips, ``Used to be more
careful myself, especially when I was on the stage.'' Warns me never to
develop a taste for beer. Says it was an Irishman who got her drinking
beer. Warns me against Irishmen.

She wants me to see everything in her apartment and lifts the curtains
that block the bedroom from the living room. ``Made them myself,'' she
says, but I see a worn tag down at the bottom. She shows me the satin
dresses in the closet, old and musty like the furniture. Says she feels
guilty having so many clothes. Then she holds up a rhinestone necklace
from the cluttered bedside table, places it against her satin chest. She
says she saves it for special occasions. Then she shows me a bouquet of
Queen Anne's Lace and dandelions she picked behind the Grecian Urn.
(Mother, remembering: ``Reminds me of the bouquets you used to bring
me.'')

Leading the way back into the living room, she tells me to sit down and
cool off. I flop down on the couch, my head against the armrest. The old
woman just keeps talking and showing me things. Takes a photograph from
a three-legged table and says it's her with her father in Greece. Lived
there until she was ten. She shows me another picture of a girl in an
organdy dress, a bow in her hair. Says it was taken shortly after they
moved here. I looked at the picture, then at the old woman. I can't
imagine ever being that old.

The old woman tells me about her mother. Says all the men in town wanted
to marry her mother. Even after they came here, men turned to watch her
mother, helped her carry groceries up to the apartment. That is, until
her father put his foot down. The old woman tucks a thin strand of
yellow hair behind her ear and pinches her cheeks to give them color.
``Even my father said I got my mother's looks,'' she says. I remember my
father telling me the same thing the night of the Spring Fling.

Now I'd be the first to admit that the old woman isn't what anyone would
call pretty, but in her own way she looks nice. And, now that I've had
time to think about it, she must be younger than ninety.

I find myself telling her about my mother. About how she works so hard,
how she's tired all the time. How sometimes I get so mad at her, I
decide to never tell her anything. Then, afterwards, I feel bad. I tell
her how last night my mother came home in a bad mood and got mad at me
for playing the radio so loud. Then she screamed about how I have a bad
attitude and threw my socks under the bed instead of in the hamper.
(Mother winces: ``And what if I tell my friends everything you say?'')

``In the old country, we had a story about the ugly duckling who grew
into a beautiful swan.'' The old woman props her feet on the couch and
tells the story I've heard a hundred times before. When my mother told
it, she made up different voices for all the ducks.

When the old woman gets to rambling like this with her stories, I know
she's tired. Usually I listen to two or three and tell her I have to get
home. Then she tells the story about a girl who's always in a hurry.
It's a funny one where the girl turns into a clock or ends up running on
quicksand. We both laugh and I leave.

I stop in at the bakery to talk to Maria. We've gotten to be good
friends and usually check out all the boys who walk by. Some come in and
talk for a while. But she doesn't know the guy with the tattoo and I've
never seen him again.

Maria has me cutting slices of baklava as she kneads dough. Once she let
me make baklava, kept shouting directions and handing me things to put
in. But I must have forgotten something because it turned out a runny
mess and we had to throw it away.

The bell rings. Maria wipes her hands on her apron and goes out front.
When she comes back, she asks me if I still hang out in Trapper's Alley,
if I still see the old woman.

There's something in the way she says it I don't like. I pretend not to
hear.

``That old woman is crazy, you know,'' Maria says. She's a liar, too.
Bet she told you how she was born in Greece,'' Maria laughs like she's
been saving it up for a long time. She tells me that every­one knows the
old woman was really born on the east side of Detroit, knows she worked
for years at that old hotel on Third Street. ``Everyone knows it. She's
no more Greek than you are.''

I tell Maria to shut up and leave before I've even finished cutting the
second pan of baklava.

My mother comes home in a rotten mood and when I mention her birthday,
she tells me to forget it. Says she's getting to old to have birthdays.
But ever since I was a little kid, I liked birthdays. Anyone's birthday.

When I tell my mother not to plan anything for Saturday, she gets that
funny look mothers get when their kids do something nice. She tries to
guess, threatens to tickle me until I tell.

Saturday comes. I shake my mother awake, hand her the tray with toast
and coffee. She starts crying. I tell her she looks sixteen and she
laughs.

My mother says all she wants to do is sleep all day, but I tell her I'm
taking her on a bus some­where for her birthday surprise. ``I've got big
plans for you,'' I say.

My mother says it's been ages since she was on a bus. I get a little
nervous when we enter Detroit. But my mother doesn't say anything about
it. She's just looking out the window and telling me how everything's
changed.

When we get to Greektown, she says it's like another world. I buy her a
piece of baklava.

The old woman is in on the surprise. She's watching for us and winks
when she sees us in the crowd. She stands up, pulls on her white gloves,
smoothing them to the elbow.

``That old woman could have a stroke,'' mother says. ``All those heavy
clothes in this heat.'' She shakes her head.

The old woman takes my hand, leads me to the piano. I play. The old
woman dances, alone at first. Then, she starts the line, pulls my mother
from the tourists. Mr. Antilopolis takes my mother's hand. I hear my
mother protest, her words louder than the music. Finally, her laughter.
In the end, my mother and the old woman sit next to me on the piano
bench.

To the old woman, my mother says, ``I don't know how you do it. Teach me
your secret.'' My mother gasps for breath, takes the old woman's hand.
To me, she says, ``And when did you learn to play the piano?''

``I will teach you to play,'' the old woman says to my mother.

My mother laughs, argues, but the old woman soon has her playing.
Nothing fancy, of course, but playing all the same.

When my mother gets tired of playing, the old woman kisses her on both
cheeks. I can tell my mother's embarrassed by the way she picks at her
nails; she does that when she doesn't know what to say or do.

``My mother's thirty-five today,'' I tell the old woman.

``So young,'' the old woman says, taking my mother's face between her
hands. ``But you should get out in the sun more. You're so pale. In
Greece, young women like you danced every day under the sun.''

My mother moves closer to the old woman, puts her feet on the pedals,
begins to play notes on the high keys.

``So young to have such a grown-up daughter,'' the old woman says after
drinking beer from her paper cup. She smooths the satin material over
her thighs. ``If I had a daughter, I'd want her to be like yours.''

My mother tells the old woman how I throw my clothes on the floor, how I
ran down the street naked when I was three.

The old woman laughs and tells my mother about the Greek boy with the
tattoo. They're still talking about me when I slide off the piano bench
and leave.

I come back with two paper cups of beer and a bouquet of Queen Anne's
Lace and dandeli­ons. I hand each of them part of the bouquet.

``Beer is good for living things,'' the old woman says, putting the
bouquet in the cup. Mother does the same.

I link arms with theirs and lead them through the back door of the
Grecian Urn. As we sit at the table, my mother looks at me like she does
when I forget to tell her where I'm going. The she looks at the old
woman and says, ``It's embarrassing, but I didn't get your name.'' She
looks at me again, and I realize I don't know the old woman's name
either.

``Call me Yaya,'' she says, flicking a small spider off the Queen Anne's
Lace. That's Greek for grandma.'' She sets her cup in the middle of the
table, places my mother's beside it. ``There. Reminds me of the time I
was singing with Frankie on tour. He picked a rose from the Queen's
garden and put it behind my ear. Completely forgot about the thorns
until I started yelling.'' She tucks her slip strap back under the white
satin.

``Who's Frankie?'' I ask.

The woman tells me the young forget too fast. My mother agrees.

The old woman snaps off a Queen Anne's Lace and pins it in my mother's
hair. ``For the birthday girl.''

My mother says she hasn't had flowers in her hair since her wedding day
when she wore a wreath of baby's breath.

I think of how she looks in her wedding pictures, standing next to my
father in his tuxedo. ``Remember how you always told me to wash my hands
when I looked at those pictures?'' I ask my mother. ``I used to think
you were a fairy princess and someday we'd all go back to your castle.
Remember?''

My mother laughs like she did in the old days. ``How could I forget? You
know I saved my wedding dress for you to wear someday, if you want to.''

I'm surprised. She'd never told me she saved it; I never thought of her
as sentimental. After all, she's always teasing me about saving ticket
stubs and shells from long-ago vacations, tells me to clean up all the
clutter.

``Enjoy,'' the waiter says as he sets the spinach pie and salad on the
table.

The old woman waves a piece of cheese on her fork and says, ``Feta
cheese. It makes you sexy, or so my father always said. Enjoy.'' She
winks at my mother and leans over and takes the cheese from my salad.
``You're too young to worry about that.''

The old woman invites us to her apartment to see her new parakeet that
likes beer. Says she bought a female because they sing prettier. ``This
one got loose in the store and took a whole day to catch. Knew right
then it was the one for me,'' the old woman says, picking up her bouquet
from the table.

After we tell her we'd like to but we have to go, she kisses us and I
feel her heart beat under my cheek as she hugs me first, then my mother.

``Come back next week to see Aphrodite. That's what I named her.'' My
mother and I looked at her. ``The parakeet I was telling you about. I'll
make sure she doesn't get loose before then,'' the old woman says.

We promise we'll come, and I tell her I'll even bring some beer.

The old woman waves at us until we turn the corner.



\cleardoublepage
\chapter{Raspberry Surprise}

It isn't that I'm not appreciated. It's just that like the early morning
newspaper, I'm missed more when I'm feeling unappreciated. I play games
with myself: I wonder what would happen if I were to go back to school,
or stay in bed all day, or get a job, or run away. The running away is
the best game because I predict what I'll pack. Sometimes I leave with
only the clothes I'm wearing. Some­times I buy a whole new wardrobe.
Always I take the car. Always the family is frantic. Tom is pictured
telling the police that I am a good wife, too dependable to run away.
And always Tom looks puzzled when the police tell him thousands of wives
run away each year.

But today I'm tired of these games and I need something more solid. I
need to sit at the kitchen table with my cup of coffee and actually read
the newspaper. It's a small rebellion, perhaps, but I'm not silently
planning the day's schedule, or worrying about the dirty dishes crusting
over with oatmeal, or rushing through to the grocery ads so I can plan
my shopping. I just sit here sipping my coffee, pulling my stained robe
closer, and reading.

I don't feel angry at the slender models who wear the latest hair
styles. I don't promise myself I'll go on a diet, or get my hair cut, or
try a new shade of lipstick. I don't even cut out any recipes although
some, especially the raspberry surprise cookies, look good. Maybe it's
that my first babysitter is getting married. Maybe it's that Julie next
door told me yesterday she is getting a divorce. Or maybe it's the phone
call this morning when I promised the efficient voice that I'd be a room
mother again next year. Or it could be my husband's talk about the
efficient new secretary. All I know is that I've been feeling out of
place recently, as if I didn't belong anywhere. The reason or reasons
really don't matter.

What does matter is the inconspicuous announcement about a women's
consciousness raising group that is meeting tonight at the local
community college. I'm not sure what it is about, but I do know my
consciousness needs raising, at least beyond the level of dirty dishes.

I lift the phone receiver from its hook on the wall and feel the oatmeal
stickiness from when I answered it quickly this morning. I punch out the
number, listen to the busy tone for a while, and then quickly push all
the correct buttons again before I can change my mind. This time it
rings and on the second ring is answered by a young, confused-sounding
voice. Somehow, this gives me comfort.

``Woman's Center. May I help you?''

``I'm calling about the meeting tonight of the consciousness raising
group.'' I hope I don't sound nervous or incompetent. I am glad she
can't see me as we talk.

``Yes?''

She sounds rushed, and I comment on it. She sighs and complains briefly
about all the calls. She says the female caller before me, who just told
her off for giving women the wrong ideas, had said a woman's place is in
the home. I wonder if the caller is right, but I don't say anything.

``Can you believe it?'' the telephone voice says. ``We even have to
fight our own. Maybe they're just scared.'' Her voice sounds uncertain.

``Maybe,'' I agree and hope she can't detect the fear in my voice. I
continue quickly, ``Anyway, I just wanted to check about registration.''

``I'll take your name, and then you just sign in tonight.''

``Veronica Vale,'' I lie. With a name like that I could be a movie star
or a newspaper writer, someone whose name, when it appears in print,
looks good.

She gives me directions before we hang up. I had meant to ask her what
they do at the meet­ings, but there hadn't seemed to be a right time to
ask. I glance at the clock and know I have to rush through the chores if
I am to be ready in time to go.

First I call Cathy, the woman across the street, to ask if her oldest
daughter can babysit until Tom gets home. She thinks I have a doctor's
appointment because she knows Tom likes me home for dinner. It seems
easier to agree with her and to construct a story about a badly infected
throat. She sympathizes and says her daughter will babysit.

Although I don't usually lie, both of the ones today are necessary and
seem to add to the excitement. For Cathy is one of those efficient women
who makes the rest of us all feel inept. She always looks at me
sympathetically whenever I forget and complain about the kids, or
housework, or feeling tied down. I know she wants to help me and I don't
want to give her the chance of making me doubt my decision to go
tonight.

Ironically, I spend the whole afternoon cleaning and straightening and
preparing a dinner that the babysitter can just heat up and have ready
for Tom. At the same time, I know the house will be messed up when I get
home, the kids will be tired and whining, and Tom will be irritated I've
been gone so long. I feel the consuming and all too familiar resentment.

I leave Tom a note telling him I am at a meeting and that I'll be home
late. No specific time, just late.

I look at my pantsuit in the mirror and decide it will do. And, as I
leave the house, unlatching the back door for the babysitter, I think
how it is like going for a doctor's appointment. I am always afraid I
won't be able to explain the symptoms clearly enough, or, worse yet, he
won't find anything wrong with me. And what if that happens at the
meeting? What if we all have the same symptoms, and there is nothing
wrong? What if it's normal to hurt, to cry out in pain? What if we are
told it's all in our minds?

But, like a doctor's appointment, I have to go through with it, I tell
myself as I park the station wagon and enter the building. I find the
room with no trouble. When I look in and see other women who look like
the women at last week's P.T.A. meeting, I feel an immediate affection
for them.

I turn to the woman at the desk next to the door. The woman smiles
encouragingly, and I wonder if it is the same one I spoke to on the
phone this morning.

``Name, please.''

``Veronica Vale,'' I lie again. I watch as she writes the name I've
chosen for myself. It flows across the paper and gives me confidence.

I walk into the crowded room and find a seat in the semicircle. The
woman next to me is twist­ing the clasp on her purse. I smile at her and
she leans over to me and confesses she hasn't been to a consciousness
raising group before and isn't sure why she's here tonight. When I tell
her that I feel the same way, she asks my name. I hesitate, confess that
my name is Janet, and start to giggle.

She looks puzzled, but I don't know how to explain that sudden urge to
deceive, to be glamor­ous. The meeting starts and I settle back to
listen.

The speaker stays seated toward the front of the semicircle. Only her
hands are in constant motion as she explains that last year she started
feeling depressed and had attended one of these meet­ings. With the
other members of the group, she probed her dissatisfaction and had come
to see her­self differently. Because the group helped her, she took
training to become a group leader.

Her right arm gestures in a half arc drawing the women closer to her as
she confides, ``But I began as some of you are beginning, not quite sure
why I attended the first meeting. In fact, for the first meeting, I told
my husband I was going to a meeting for Girl Scout leaders.'' I hear the
soft laugh­ter of others who had lied tonight. And I know I'm not alone.

The leader calls off the names. I don't answer when she calls Veronica
Vale. It sounds silly when she says it out loud. Then the group leader
asks if anyone's name is not on the list. I give her my name, Janet
Monroe.

Next we play a game. We must pretend that we're someone else and choose
one thing that person would have done today that we did not do. I choose
to be Cathy, the woman who lives across the street. I have no doubt that
she would have baked the raspberry surprise cookies today.

The leader asks me to pull my chair into the middle of the semicircle.
She remembers me because my name wasn't on the list, and I think she
suspects why.

I'm nervous, but I do as she asks. She pulls her chair out next to mine
and asks who I've chosen to be. I tell her briefly about Cathy, the
perfect wife and mother.

The leader instructs me to speak as my neighbor would and answer all
questions as she would. If not in real life, at least in a game I get to
be the woman I'm so jealous of.

``Why did you waste time baking cookies today?'' the leader asks.

I'm startled for a moment and then remember it's a game. I play along.
``It only took an hour, and I know the children would enjoy them when
they got home from school.''

``But you could have been reading a book or taking a nap,'' the leader
says. ``The children will eat those cookies, and what will you have got
for your time and effort?''

``But I like the children to be happy. That gives me pleasure. There'll
be time enough for me when the children are grown.'' I'm proud of my
response.

``But how do you know there will be any you when the children are
grown?''

If I were really Cathy, I would know what to say. But I'm not Cathy, and
I'm stuck for an answer.

The group discusses the last question and then I exchange places with
another woman who tells about the person she's pretending to be. We only
get halfway through the semicircle when it's time for the meeting to
end. I feel comfortable with these women and know I'll be back next
week.

It's only seven o'clock and I've still got time to try the raspberry
surprise cookies, which will probably make me ill. I decide to tell Tom
about the meeting. In fact, I'll even tell him about Veronica Vale.
It'll be good for a laugh.



\cleardoublepage
\chapter{Real Enough}

Miriam is cleaning her closets. She throws out the old stuff, saves some
for Goodwill.

On the top shelf in the back, she finds her blue beret under her summer
straw hat. She shakes off the dust and tries it on.

In the mirror, she sees. The blue has faded. Her husband, daughters, and
grandchildren watch her from their silver frames on the dresser.

Thick carpeting shuffles her steps on the stairs. Voices play in her
head, a musical jewelry box. Anyone can lift the lid, watch her twirl.
But the gears have rusted, voices sound hoarse.

Her husband's voice from the bedroom, looking for socks. Her children's
voices crying for night feedings. Older now and from the living room,
wanting to know if dinner is on the table. Grand­children, asking for
presents. Other voices from other rooms. They're glad to meet the mother
of, wife of. They all forget her name.

Lately, she forgets the names of neighbors. Forgets she cleaned the oven
yesterday. At fifty-nine, she reads articles on senility and refuses
senior citizen discounts. Lately, she's taken to reading the
classifieds.

Her husband chides, ``Lucky for you. You never had to work.''

Miriam pulls last night's newspaper from the trash can under the sink.
She cuts out an ad she first noticed two weeks ago when she wrapped
chicken bones. Just an ordinary ad. ``We find jobs for people,'' it
said. In small letters, the name of an employment agency and an address.
Tomorrow they can find a job for her.

Next morning. She stands with her back against the metal bus stop sign.
When the bus opens its doors, she rummages in her purse for change. She
is on the second step when the driver takes off. She grabs the pole and
hands him a dollar. He points to a sign above his head: Correct change
only. They make another stop. People rush at her from behind. She folds
the dollar bill and stuffs it into the fare box.

Holding onto the seats, she stumbles to the back, finds a seat over the
motor. After a while, she bounces with the bus, rocked by a mother with
hard, yellow arms.

Miriam opens her purse and pulls out the ad. The clean black strokes
against the white. The same words, no matter how often she checks.

A man squeezes into the aisle seat next to her. She knows he's reading
the ad in her hand. Quickly, she stuffs it into her purse. His tweed
wool suit is worn and out of style. He is old.

His right hand moves in his pocket. Miriam cannot turn her eyes away.
She bites the side of her cheek. The man withdraws a small penknife. She
gasps.

Waving the knife in Miriam's direction, he says, ``You okay, lady? You
look like you're going to be sick.'' Reaching across her, the knife
still in his hand, he raises the window.

He watches Miriam as he takes a small block of wood from his pocket.
Both hands are visible now. They're larger than she expected. She
follows their movements.

When the bus jerks to a stop, the man says without looking up, ``This is
where you're sup­posed to get off. Just walk back one block. It's the
grey building across the street.''

Miriam brushes past his legs, never doubting his words.

She spends an hour filling out forms, using words where she can and ink
lines where she can't. The girl behind the desk takes her papers, tells
her to come back tomorrow for the interview. Miriam pushes open the
heavy glass door. The ball jangles her departure.

The rest of the afternoon, Miriam shops but buys nothing. She knows she
must leave herself enough time to get dinner ready before her husband
gets home. Every day they eat at 5:30. Even on vacations they stop what
they're doing and eat.

On the way home. She sinks into the seat on the bus, leans against the
window and closes her eyes. She feels someone slide into the seat next
to her and is depressed like some nights when her husband lies down
beside her.

It is the man of the morning working with his penknife, scraping away
slivers of wood. Although his hands never stop, his back stiffens.
Miriam knows he recognizes her.

``Thought you were sleeping,'' he says without looking up. Says it in
the same tone Miriam's husband uses when he comes to bed after reading
late. She never answers then and she doesn't answer now.

``Get your job?'' he asks. ``Don't know why you want a job. Doesn't look
like you're starving.''

Miriam pulls in her stomach and looks away. Looks under the window where
school children have scratched their names and dates into green paint.
Scraggly lines recording every bounce.

He nudges her arm. Says something that sounds like ``sorry.'' It may be
just the sound of his shoes against the metal floor.

Avoiding his eyes, she looks at his hair, full and thick. Her husband is
almost bald now. Even her hair is thinning. It no longer holds a curl.

``For your grandchildren,'' he says as he drops the carving in her lap.
``A bear. I used to hunt them when I was younger. Up in Canada.'' And he
pulls the bell and is gone.

In bed that night. Miriam wishes she had given back the bear, wonders if
she should tell her sleeping husband about this afternoon.

It's the same driver the next morning. Before the doors open she has the
correct change in her hand. She's practicing for the interview, making
up questions she has a tough time answering. She moves toward a seat.

As the bus turns right, she grips the seat in front of her, feels the
stickiness of other hands. Tired, she shuts her eyes against the Indian
Summer glare. She smells familiar wool, knows the man of yesterday is
sitting beside her.

Miriam opens her eyes, watches his penknife, an extra finger stroking
the wood. ``Don't you ever make mistakes?''

``Nope. Can't make a mistake 'cause I carve whatever turns out. Easier
that way.''

``Another bear?'' she asks.

Maybe. Maybe not. Once I made a whole wooden garden. Used to farm once.
Down in Virginia.

Miriam sees his hands pulling out weeds, spading the soil, much as her
hands move in her small garden. She drifts, content to have him beside
her.

She wakes startled as he touches her elbow.

``Your stop,'' he says.

Miriam makes sure her beret is in place as she enters the employment
office. The girl behind the desk motions her forward, asks her to sign
her name in a book filled with pages of other names. Miriam takes a
seat. No one else waits with her.

When her name's called, Miriam walks past the closed doors, searches for
room 115. She finds it and knocks.

The woman behind the desk is younger than Miriam's daughter, has a tan
that hasn't started to fade yet. The young woman smiles, nods her head.
She writes down what Miriam says. In the end, she shakes Miriam's hand.

On the way out, she passes a girl in a waitress uniform. She moves to
the side, lets the girl pass.

Miriam walks through Hudson's, stopping in the book department to read
the first paragraphs of the best sellers and in cosmetics to squirt her
wrists with a new fragrance. She waits until it's time to go home again.

The man gets on at the stop after Miriam's. Out the window she can see
St. Matthews, built 1896. A work crew is sandblasting the bricks,
replacing the crumbling mortar.

``There's hope for us yet,'' the old man chuckles, pointing to the
church. ``Get your job?''

``No,'' Miriam says, picking lint from her sweater.

``Do you mind?''

She doesn't answer, glances away to the man in red swim trunks laughing
from an ad over Miriam's head. He is applying a suntan lotion. Soon
he'll be replaced by a man with a muffler wrapped twice around his neck,
offering to protect her from chapped lips.

The old man offers Miriam a chunk of wood. ``You try it,'' he says,
laying the penknife in her hand.

``I can't,'' she says, wondering if the man has been drinking, wondering
how he spends his afternoons.

``Just start shaving the parts that don't seem right. It'll come,'' he
says, folding his arms against his chest.

Miriam starts skinning the wood like a potato until the next layer shows
through. Then she begins, timidly at first. She remembers the circus she
carved long ago from Ivory soap---a monkey, ringmaster, even a small
cage. One day she took the whole circus into the bath. Watched it float
and melt.

The old man is gone when she looks up. She knows she'll have to take the
bus again, return the knife.

That night her husband is watching the news. Miriam cleans the handle of
the knife until it shines and puts it back in her purse. Not feeling
like doing needlepoint, she files and polishes her nails. Her mother
used to tell her it was a pity she had her father's hands. If they had
been longer, more slender, her mother said, she might have played the
piano better. Miriam always hated the piano, hated the way the music
teacher said, ``Stretch those fingers. Reach.''

The next morning. She watches the old man walk down the aisle toward
her. She remembers other old men she watched from car windows. When he
sits next to her, she hands him the knife. He examines it, puts it in
his pocket.

He rubs his fingers over the torn armrest. ``Nice hat,'' he says, not
looking at her.

She touches the beret, angles it more over her forehead. ``You like
it?''

``I used to own lots of clothes, a real dandy,'' and he chuckles at some
memory. He runs his fingers over his worn lapel. ``But this here
suitcoat's been with me a long time.''

Miriam inhales deeply: the mustiness, the wood, the wool, even the fumes
from the bus. But she can't detect the sourness of the first day.

Watching the city move past her, Miriam tries to count the
going-out-of-business sales. As the bus stops in front of Drop Inn,
rocks hit metal below her window. She sees two children running as the
bus driver shouts, shakes his fist.

The man beside her throws back his head and laughs, a raucous sound that
fills the bus and turns heads. He touches Miriam's arm. ``Kids,'' he
says. ``Always have to remind us they're around.''

He hands Miriam the knife and wood. Silently, she sets to work,
returning the knife only when she gets off at her stop.

The bus rides become routine. Her husband doesn't notice the dust piling
up around the house. Her children ask where she's been lately when they
try to call.

A cold morning in December. Miriam shivers as she waits for the bus.
When it comes, she's glad of its headlights in the morning darkness.

The old man is already on the bus. He has saved the window seat and she
hurries to claim it.

When seated, she opens her purse and hands him the carving she finished
last night after her husband was asleep.

The man turns the carving in his hands, rubs his thumb over the surface.
``I knew you had good hands for this,'' he says, dropping the carving
into his inside coat pocket.

From time to time Miriam looks at the small pocket bulge.

``I got on at the first stop today,'' the man says. ``The driver was in
getting coffee and left the bus idling.''

Miriam thinks about sitting on the bus alone.

``I could've just driven it away, the man says, carving his rhythmic
laugh into the wood.

``That would be nice. But you'd get in trouble.'' After some thought,
she says, ``Wonder how far you'd get before they stopped you.''

``They wouldn't bother with an old man like me.''

``They'd want the bus back.''

``They'd have to stop me first.''

Miriam pictures it. Police cars, speed, people left waiting under metal
signs, shifting gears, the old man's laugh.

``Your stop,'' the man says.

She needs to sit down someplace and think. She has seen the Greyhound
Terminal but never had a reason to go there before.

Now she walks five blocks over. She sits on one of the long benches in
the terminal eating an egg salad sandwich she bought from the machines.
People come and go, alone and with others, carry­ing suitcases or
magazines, studying the departure and arrival signs, moving through
turnstiles, going places.

Back on the city bus, headed home. Miriam watches the old man swaying,
sees the bulge still in his pocket. He looks younger.

When he's settled and carving again, she says, ``I've never been to
Canada to hunt bears.''

``You could do it,'' he says. ``Even now,'' and his voice trails off as
he concentrates, moving his right hand faster, more sure.

He folds the penknife, puts it in his pocket, and hands the carving to
Miriam. ``For you,'' he says.

After pulling the bell cord, the man stands up, leans over to Miriam and
whispers so quietly and quickly she almost misses the words, ``Tomorrow.
Be ready. We'll go hunt bears in Canada.''

Miriam is at the bus stop early the next morning. She wears heavy boots
and mittens. She is ready. She watches the bus plow through traffic like
an ice cutter. It comes to a full stop before the doors swing outward.
The everyday driver motions her on.

Although Miriam leans forward at each stop and wipes the steam from the
window, the old man does not appear. She hopes he isn't ill, hopes he
hasn't gone without her.

For the next two weeks she rides the bus alone, saves his aisle seat.
She remembers asking him once if he really ever hunted bears in Canada.
He said maybe he had and maybe he hadn't. He told her about flying a
plane, shipping out on a freighter, working on the assembly line. Said
whatever he hadn't done for real, he dreamed of doing and that was real
enough.

Sometime during the third week alone, Miriam gives up on seeing the old
man. She settles easily enough into cleaning house during the day and
doing her needlepoint in the evenings.

Mid-afternoon in May. All day Miriam is raking up the winter leaves,
clipping the dead branches. She wants to get the roses in early.

But she grows tired. Wants to take a shower, get into some other
clothes.

She is changing into her favorite dress. There, back on her closet
shelf, she sees the blue beret. She puts it on. It's been a while, she
tells herself. But just maybe she'll run into the old man, she thinks as
she locks up the house.

The bus is on schedule. She boards and takes her usual window seat. At
the next stop, a heavy-set lady pushes in beside her, brown paper bags
erupting in her arms and flowing into Miriam's lap.

``Lordy, it's been a lost day,'' the lady says as Miriam helps her wedge
a shopping bag between the seats. ``A gardener, I see. I can tell a lot
about people from their hands.''

Miriam looks at her hands, at the soil ground under the nails, into the
chapped corners. ``Just some rose bushes back by the garage,'' she says.

``Some people have green thumbs,'' the lady says as she takes off her
shoes and massages her feet. ``These feet of mine sure aren't good for
much anymore.''

``Used to own a nursery once,'' Miriam says. ``Up in Canada. Sturdiest
flowers you ever saw.''

``I can't grow anything, anything that lives for long,'' the lady says.
Too tired to fuss with plants after waiting on people all day long.''

The lady rummages in the bag at her feet and pulls out a bag of
chocolate-covered raisins. She offers some to Miriam. ``Too tired to
even cook for myself when I get home,'' she says.

Miriam lets the chocolate melt on her tongue, rubs her hands together.
Tomorrow she will bring the lady cuttings from her grape ivy. Ivy grows
well in dark places, even grows in Canada's cold climate. The lady's
hands look right for growing ivy.



\cleardoublepage
\chapter{Rearranging Furniture}

Mr. Bertilano leans across the desk toward me and balances the pencil
until the pen falls. Sets it up again. He's the boy with the erector
set. Tighten the bolts. Call in the girl child to ``ooh'' and ``ah.'' She
can use the wrench only after he shows her how.

I know why he calls me into his office today. I've been expecting it,
but I'm not prepared.

Two years ago, I sat before him just as I'm sitting now: Picking
imaginary lint off my skirt, biting the inside of cheek, listening to
his words, keeping my eyes on his first shirt button, not daring to look
into his face. Two years ago, he leaned back in his oversized leather
chair and laughed from side to side. He laughs like that again today.

Two years ago, he said there were no openings, said he'd call me if
anything turned up. I knew he wouldn't, knew it from the way he looked
above my head as he said it.

Today, he looks at my face, probes it for signs he expects. A wrinkle
perhaps at the corner of the mouth or a slight twitch in the right eye.
He knows I know.

``You've done a lot since you first came to us,'' Mr. Bertilano says,
then waits.

``Yes,'' I agree, wanting him to make it real, put it into words.

The buzzer rings and Mr. Bertilano pushes a button. ``I told you to hold
the calls, Shirley.'' To me, he says, ``Five years now she still can't
follow simple directions. If I could find a good secretary, I'd fire
Shirley as fast as that.'' He snaps his fingers.

I believe him. He'd fire all the Shirleys in the world if he could. All
the Charlies, too. But most of all, it's the Shirleys who irk him.

Two years ago, during the interview, he called me Shirley. I corrected
him. He pleated the corner of my resume and doodled circles at the top,
by my name.

I pictured his black wingtips behind the desk, one shoe kicking at the
thick green carpet. Just like the team captain's white sneakers in
fourth grade, drawing circles in the sand until Miss Smyth divided the
girls, half on each team. All batting last.

But there had been no Miss Smythe to nudge the wingtips. Only me.

I made my own opening. I just walked into the first department with the
first empty desk and sat down. The five other people looked at me and I
introduced myself as the new copy writer. (I sus­pected correctly that
Mr. Bertilano rarely moved from his own office.)

The bald man walked over and dropped copy on my desk, saying, ``You
better be good.''

At lunch, one of the women told me that Charlie was okay once you got to
know him. It's just that Charlie usually did the hiring and firing in
the copy department. ``Doesn't look good for Charlie,'' the woman had
said. She'd been right.

Today, Mr. Bertilano capsulizes my progress, ticks off each step on his
fingers, pudgy fingers that don't move as he strikes them one by one.
``You're a real dynamo. On your way up and moving fast. You can handle
the rush jobs. The advertisers ask for you. The theme issue you
suggested on mothers and daughters was terrific. The working-woman
series drew letters from all over the country. You know what sells.''

He's the King of Id and I'm the favored wizard, at least for the moment.
He's on the balcony, making pronouncements as I stand below, waiting. I
can laugh at the King in the comic strip.

``Three days after you started, Charlie came griping to me about you. He
said you were too high and mighty and should be fired. I told Charlie he
must be drinking too much again. I asked him how I could fire someone I
hadn't even hired. Charlie went roaring out of here like a mad bull.''
Mr. Bertilano taps the pen against the black signet ring on his little
finger. ``When you were still here after the first week, I hired you.''
He pauses and shakes his head. ``My god, most men don't have the guts to
just walk in and take over like you did.''

``I'm good at my work,'' I tell him. I needed a job. That's all it was,
nothing more. How can I tell him the anger I felt two years ago when he
looked above my head.

``You certainly proved yourself. You're tough and good. Later even
Charlie had to admit it. You were the only one Charlie couldn't bully,''
Mr. Bertilano says as he drops the pen into the desk drawer and shuts
it.

In my mind, the sun pounds on Mr. Bertilano's bald head, melting his
bulk into a puddle of golden butter.

``It's time to move Charlie out. He's getting stale. His job's yours if
you want it,'' he says cross­ing his creased cotton-shirted arms over
his chest.

When I ask what will happen to Charlie, Mr. Bertilano tells me that
Charlie is his problem, not mine.

I remember the time in fifth grade when I beat Richie Neuman in a
running race. I won the Siamese fighting fish and everyone gathered
around to watch the electric-blue fins sweep the sides of the glass.
Richie, who already had a fish tank and heater, offered to buy the fish.
But I won it fair and square. On the way home, I dropped the fish down
the sewer and smashed the bowl against the curb.

It's like that again today. And even though I've rehearsed the words,
they won't leave the dressing room.

Mr. Bertilano waits for the curtain to go up.

I lower my head and accept the promotion.

``That's my girl,'' he says, and I want to punch his smug face, make his
eyes rattle like jelly beans in a plastic egg.

That night, I call my mother to tell her the news. She says, ``That's
nice, dear,'' and tells me Sarah Memet's daughter just had a baby boy
last week in Grace Hospital.

I ask her if she wants me to have a son.

She ignores the question, tells me Sarah's daughter plans to nurse the
child. ``Sarah's so happy,'' she says.

``Sarah didn't have the baby, her daughter did.'' My voice is whiny, the
same tone Mother gets when Father builds a table in his basement
workshop. I hate it when I sound like her. Sometimes I wish I were like
other daughters, married and having babies. I think she'd like me
better. I'd probably like her better too.

Mother changes the subjects, asks about my latest boyfriend, asks if I
want to bring him to Sunday dinner.

I try to picture Ken fitting his long legs under the white lace
tablecloth, trying to talk about the archeological dig he's just left in
South Dakota to my father who grunts at anything he doesn't under­stand.
My mother will flutter about, pouring coffee, asking Ken if he can bring
her a pot for her begon­ias since she so loves old things.

The next day my mother will call and tell me what a nice boy Ken is and
warn me not to scare this one away with all my talk. I don't want my
mother to meet Ken.

As if she knows what I'm thinking, Mother says. ``I'll be good.
Sometimes I think you're ashamed of us.''

She asks me about coming this Sunday. I tell her I can't. She says, ``I
understand. It's just that your father misses seeing you. You should
call more often. Someday you'll know.'' This is typical of my mother's
vague threats that make me dream of falling bedroom floors and
strawberry Jell-O that won't harden in the mold.

I never know how to end phone conversations with her. I fumble with
good-bye words and she says, ``So I have to go finish dinner now,'' and
hangs up the phone.

A tiny, nagging Mother voice follows me into the kitchen. ``Eat better.
You wouldn't be so nervous if you'd eat better.'' I fill the bowl to the
top with potato chips, slip the grilled cheese sandwich from the skillet
and pour the bottle of Pepsi into a glass from McDonald's.

``Plenty of milk and fresh vegetables,'' the Mother voice persists.
``It's only a couple of miles. You can't spare the time to come over for
dinner? Such a busy person, my daughter.'' The clucking tongue.

I pull out the photograph album Mother gave me last Christmas. The
scenes are snapshots I haven't mounted yet. ``So you won't forget us
all,'' Mother said as I opened the Santa Clause package. Everyone
laughed, even my sister-in-law. Grandmother nodded solemnly, and said
with authority, ``Your family is all you have. Value it.''

Clearly, Mother had selected the photographs. I wipe the chip crumbs
from Mother as an infant in Grandmother's arms. Mother in a weeping
willow tree, her dress tangled in the branches. Mother in a bathing suit
at the beach, her knees turned inward, her breasts full. Mother at the
kitchen sink washing dishes. Mother giving me a permanent, rollers
between her lips.

I finish my grilled cheese sandwich and wipe butter from the plastic
pages. Wedding pictures, formal expressions as Father poses beside
Mother. Father has no pictures of himself as a child or young man. In
the photographs, he's only a husband and father. Father in the workshop
building my dollhouse. Father holding me on my red bike. Father and
Mother dressed for his retirement dinner.

And the me that exists in the photographs, in my mother's eyes. My First
Communion dress, white lace itching my neck. My first birthday party. My
first date with Billy next door whom I always detest for telling
everyone I wouldn't kiss him goodnight. No graduation pictures. Someone
forgot to buy film. My parents argued over whose fault it was, clearer
than any snapshot.

I worry about the promotion as I wash the few dishes, dust, vacuum,
rearrange the furniture. It's my mother's habit, remembered from the
times I talked back or the neighbor's cake rose evenly and hers fell.
The female cure: eat right, clean house, say nothing, rearrange the
flowers in the vase.

I know my mother would be shocked if she knew I called Ken or that he
slept here some­times. When she calls and he's here, I tell her I'm
alone, reading or baking. Usually, I can hear her smile over the phone.
Then she tells me if I had a nice husband, I wouldn't have to be lonely.
I tell her I'm not lonely. She tells me I'm a really pretty girl and
bright. I feel the way I used to when I was sixteen and getting ready
for a date. She'd make me twirl slowly. Then she'd rub a spot of makeup
on my forehead and say, ``There. Now you look fine.''

I dial Ken's number and cross my fingers behind my back for luck, a
gesture left over from childhood. When Ken answers, I tell him about my
promotion.

Ken is quiet. His voice finally comes from far away, as if he's not
holding the speaker to his lips. ``Congratulations, boss lady.''

It's not Ken's voice, Ken's words. Those two ugly words: boss lady.
Before, I'd only hard them said about other women. Heard them said by
women I'd worked with at Ford's, heard them whis­pered over promotional
brochures we were working on. And the men said the same---boss lady,
bitch, whore---all the ugly words.

Ken is laughing now. ``I can't believe it. A boss lady who forgets where
she puts her car keys.'' He tells me his theory that in the future,
archeologists will know how many female and male bosses existed in each
corporation. All they'll have to do is count doors. Those doors without
locks belonged to female bosses; the ones with locks belonged to male
bosses.

I tell him he's not funny and he says I can't take a joke. I tell him
I'm really worried about tak­ing the promotion. ``What if I'm not good
at it? What if I get to like the job and someone else comes in and does
it better? What if I get fired like Charlie?''

Ken says I'll do fine, says I worry too much.

I want to pull him through the wires, lean my head against his chest.
Then I could ask him what ``fine'' means. Instead, I ask him to Sunday
dinner at my parents' house.

He says I must be getting serious if I want him to meet my parents. He
warns me he's leaving for South Dakota in the summer. But, he says,
he'll even wear a suit Sunday.

When I tell him to fall down a flight of stairs, he hangs up the phone.

I call him back and say, ``Can't you take a joke?''

The speaker must be closer to his lips; his laugh comes across loud,
deep. I know he's no longer angry.

``Why don't you come over?'' I don't want to tell him the apartment
seems empty or that I'm afraid to be alone tonight.

Ken says he has to finish the report he's writing, has to present it to
the funding committee tomorrow. I try to persuade, use my smiling voice
when I tell him he can write it here. He says I'm too distracting.

``But tomorrow night I'll take you dancing to celebrate your
promotion.'' Then he adds, ``That is, if the boss lady's not too tired
for dancing.''

I hear my mother's voice in his, warning, ``Don't step on cracks.''

He says seriously that he's looking forward to meeting my parents
Sunday. I warn him that he won't like them. He says I'm getting too
hard. Just teasing, he adds as he hangs up the phone.

In my dreams, Ken is one of the advertisements on my desk. His body
slides from the slick paper and he tap dances across my open calendar.
My mother hands him a chocolate cake. He kisses her hand and pulls her
into the waltz position; the cake falls to the mirrored dance floor. I
try to lean forward but I've become fat, fatter than Mr. Bertilano who's
standing behind my swivel chair, clapping his hands. My enormous breasts
sag toward my lap, keep me from reaching the dancing figures. ``Bravo!''
Mr. Bertilano shouts.

I wake up at the sound of my own voice. My pillow is damp. The room is
dark. When I get up to go to the bathroom, I bump into the chest of
drawers I'd moved earlier tonight. I wash my face and leave the bathroom
light on.

The next morning I tell Mr. Bertilano that I can't take the promotion,
that, in fact, I'm giving my notice. He drops the silver pen on the
floor and has trouble finding it.

``You can't do that,'' Mr. Bertilano says from under his desk. ``I've
already told Charlie.''

I pick up the pen at my feet and hand it to him.

``But why? I deserve a reason. You can't just give up your future here.
Why don't you take a few more days to think about it?'' He rubs his
hands together.

Acceptable reasons (much better than real reasons) are tough to come by.
But I sort through possibilities. Finally I say, ``My fiancé was told
yesterday that he's being transferred back to South Dakota. We're
getting married.''

``Well, I can't blame you for choosing marriage.'' Mr. Bertilano pats my
shoulder. ``But the winters are cold in South Dakota. If you ever need
your old job back, just walk in and pick a desk.'' He chuckles.

He puts his arm around my shoulders and walks me to the door. ``Don't
forget, you're always welcome here. You'll never find a better boss.'' I
shake his hand.

It reminds me of the day I moved from my mother's house.



\cleardoublepage
\chapter{Rosy's Mother}

Three quick knocks, a pause, one louder knock. It's Vera's knuckles
against the wood, Vera on the other side.

Vera tries the door and walks in, calling, ``Rosy, are you home?''

Her shoes click against the tile, stop by the table in the hallway.
Perhaps she wipes off the dust. For sure, she reads the cards, sees: the
cross in moonlight, a spray of lilies, stained glass windows,
signatures.

``Rosy,'' she calls again.

Since we were ten, I told you to call me Rosalie, but you never
remember. ``In here,'' I call from the living room.

``You ought to keep that door locked,'' she says and hugs me, half
lifting me from the couch. ``You've sure kept your figure.'' She pulls
her striped blouse over her hips and sucks in her stomach. ``That's what
five kids do to you. Put on twenty-five pounds with the last one.''

Sitting next to me on the couch, she says, ``You know I looked in on
Roxie all the time.''

I nod my head. Of course, I know. Didn't Mother tell me in every letter
about you bringing the children, even your husband, for Sunday dinner?
Didn't I hear how your little Robbie got his first tooth? Didn't I hear
how she hated to bother you to take her shopping? Didn't she always want
to know when I was coming home or getting married or doing something
with my life?

Vera starts to cry and pulls a wadded Kleenex from her sleeve, tucks it
back again after wiping her eyes. ``Sorry, I remember how you hate
tears.''

``It's okay,'' I tell her. After all, you loved my mother better than I
did. On the days when your mother worked, you couldn't wait to get home
from school and show my mother your papers. I went to my room and drew
pictures of robins flying west for the winter. Their red breasts were
larger than the sun.

``I'm going to miss her, Rosy. Remember how all the kids always came to
your house?'' She pauses and tears at a small hole on the arm of the
couch.

How can I forget? They came even after I left. They come still. Children
and their mothers come, bringing cards drawn on manilla paper and
casseroles. The cards pile up on the hall table. Macaroni and cheese,
tuna surprise, Hungarian goulash, and even Mexican Delight are stacked
in the refrigerator.

``How I used to wish my mother, God rest her soul, was like yours,''
Vera says.

You used to want it all, Vera.

``I can still hear my mother screaming, `Don't track up that floor or
you'll get a whipping.'\,'' Vera tucks her foot under her.

``At least your place never looked like this.'' I wave my hand at the
newspapers and magazines stacked in the corner, the dirty clothes piled
on the chair.

When Buddy Lawson came to get me for a date, Mother just scooped the
clothes on the floor and kicked them under the chair. Buddy sat down and
before I knew it, Mother turned on the radio and Buddy danced her all
over the living room, showing her the latest steps. By that time, we
were late for the show and Mother's hair was out of the French twist,
hanging loosely on her shoulders.

``She had trouble climbing the stairs toward the end,'' Vera says.

``Sure,'' I say. Didn't Mother always have trouble climbing the stairs
or doing much of any­thing? Didn't Father tell her often enough that she
drank too much?

``Do you want something to drink?'' I ask Vera.

She follows me into the kitchen and looks over my shoulder, into the
refrigerator.

``So you've met the neighbors,'' Vera laughs and reaches for the nearest
casserole dish. She peels off the Saran wrap and sniffs. ``It's spoiled,
Rosy.'' She dumps it down the garbage disposal. One by one, she empties
all the dishes.

I rinse a glass and pour her a beer.

``The kids in the neighborhood are going to miss her,'' Vera says,
walking out of the kitchen and into the living room.

``The kid next door made her this.'' She takes a clay object off the
mantle and turns it in her hands.

``What is it?''

``I don't know.'' Vera sets it back in its place. She drinks the rest of
her beer and puts the empty glass next to the clay. ``Let's get
started,'' she says.

``I'll do it later.'' I don't want you poking around in my mother's
stuff, finding things I never knew were there.

``It'll go faster with the two of us and don't you have to get back to
work soon?''

``I quit,'' I tell her. Quit because I never stay in one place too long.
Quit because my legs are giving out and waitress work isn't what it used
to be.

``Oh. A man? Roxie always wanted grandchildren.''

I don't answer.

``It's about time. You should have written and told me.'' Vera hugs me
again.

``I'm not getting married.'' I pick up the clothes from the chair and
put them in a plastic garbage bag.

``Save those for the church,'' Vera says.

I can't picture my mother's clothes in a rummage sale. Older, heavy-set
women lifting dresses to their bodies, measuring the fit across the
hips.

``After I get the beer stains out,'' I say.

``Your mother did like her beer,'' Vera laughs and picks up a dress from
the floor. A pink-flowered dress with chartreuse leaves split down the
front to show cleavage.

``Wore this one to her `Welcome Summer' party last month,'' Vera says,
running the dress through her hands. ``Great party.''

I open the window and turn on the fan, adjusting it so the hot air
upstairs is sucked out the window.

``Lots of parties. Kids running in and out all day,'' Vera says. ``Just
like the old days.''

Mother sat up all night writing Halloween party invitations to everyone
in my class. You were the one who told her when I tore them up and
dropped them down the sewer, piece by piece. I watched the names she
wrote disappear into the hole. You passed out all the invitations after
that.

``I'm glad you're staying. It'll be fun to have you back home again.
Roxie would be happy.'' Vera drops the dress to the floor. ``I want you
to meet Brad. He's divorced now and looking.''

``I'm done looking.'' My voice cracks in my ears.

Vera isn't listening. She moves toward the stairs. ``Let's do the
bedroom first. That'll be the hardest.''

My body is heavy, filled with the house dust and summer pollen. Vera is
already up the stairs; her feet move above me in my mother's bedroom. I
follow.

``Look at our old school papers,'' Vera holds up a dancing clown; its
accordion legs bounce as she waves it at me. ``Mine from first grade,''
she says. ``Didn't you make one?''

``Probably not.'' Sweat drips into my eyes. It's hotter upstairs even
though I've kept the drapes closed.

Vera dumps the clown into a cardboard box she's using for trash. ``I'd
keep it if I didn't have enough of my own kids' stuff to store.'' She
sorts through more papers. ``Here's all your stuff tied with red yarn.''
She tosses the packet to me.

Mother must have retrieved them from the trash where I threw them as
soon as I got home from school.

I rip up the papers and throw them into the box.

``Are you okay?'' Vera asks, putting the back of her hand against my
forehead. ``Just warm.''

I sit down beside her on the floor and lift papers from the bottom
drawer, spelling tests and stick drawings signed by names I don't
recognize.

``Neighbor kids,'' Vera says.

Your papers, their papers next to mine. All pushed down to make room for
the next. My mother's voice the night I left, begging, ``At least tell
me what I did.'' I wrote a year later and she answered.

In the top bureau drawer, behind costume jewelry, Vera finds more
papers, official papers. She reads, tucking loose hair behind her ears.
Th sudden intake of air.

``Roxie never told me you were adopted.'' Vera holds out a document and
when I don't take it, drops it in the middle of the bed.

``I'm not adopted.'' I shake my head. ``Everyone says I have my mother's
ears.'' I pull back my hair.

In the mirror, I watch Vera's face, see the way it takes more space than
mine. She leans toward me as I whisper, ``You'd like it to be true,
wouldn't you?'' Her face moves toward the middle, away from me. ``You
always wanted her for yourself.''

A gurgle starts low in my throat and rises to fill my mouth like cool
spring water. There's a slight mineral taste to the laughter. It's my
house now, Vera. And my mother.''

``I should have cleaned all this before you got here.'' Vera straightens
the bedspread, pulling out the wrinkles where she sat. She leaves the
room and closes the door behind her. A toilet flushes in another room.

I read the paper, twice, before I lay it down.

``Get out of here,'' I tell Vera when she comes back and sits beside me.
The white chenille ridges are rough against my fingertips.

But she just sits there, a lump of clay beginning to harden in the dray
air.

I ball up the stiff piece of paper, roll it between my palms. Saliva
gathers on my tongue. ``Get out,'' I scream and pull back my arm.
``Out.'' The paper ball hits her on the forehead.

``I'm sorry,'' she says and moves to put her arm around me.

``Get out, now.'' This time the paper bounces off her cheek, onto the
floor.

The bed creaks as Vera moves. Her footsteps go past the stairs, into the
bathroom. Water runs behind the closed door.

The old skeleton key is still in the bathroom door; it clicks as I turn
it.

Vera rattles the knob. ``Let me out,'' she says.

I don't answer. Instead, I go into the bedroom and pick up the paper
ball from the floor. After putting it into an ashtray, I dump drawers,
looking for a pack of matches. There, between the silver bobby pins and
cold cream, a box of safety matches. The smell of sulfur and smoke. A
burst of flame.

``What are you doing?'' Vera screams from the bathroom. Her shoes are
soft thuds against the door.

At first, I worry the wood will split but it's old, probably petrified
by now.

``What's burning, Rosy?'' She uses her mother voice, softer now. ``It's
going to be okay. Come to the door so we can talk.''

After the flames, only charcoal swirls. I carry the ashtray to the
bathroom door, turn the key. Rushing past me, Vera almost knocks it from
my hand. Her feet loud against the wood steps.

Kneeling, I hold the ashtray above the toilet and watch the ashes sift
down, some clinging to the white porcelain sides. Depress the silver
handle and all are caught in the whirlpool. For a while, I stare into
the clean water that bubbles back up and settles.

I go downstairs where it is cooler and pull back a corner of the living
room drape. The sun is hot against the window. Children next door are
playing in a plastic wading pool; they fill buckets with water and dump
them over each other's heads. Squeals and shouts fill the afternoon.

Opening the refrigerator, I take out cans of beer and line them up on
the white Formica table. Sweat runs down their aluminum sides and
puddles at the bottom.

I pull out a chair and sit in Mother's place. I drink from the cans,
wiping the foam from the corner of my lips with my finger. After the
fifth beer, I draw hearts on paper and pierce each with an arrow. I am
particularly careful with the feathers on the shaft; they must be evenly
spaced and slanted at the right angle. I print: COME TO MY PARTY
TOMORROW AT NOON.

Tonight I'll bake gingerbread men with chocolate chip eyes and cinnamon
grins. We can play Red Rover, Red Rover and I'll hold tight to my
partner's hand so you can't break through.

My eyes are tired and the stains under my arms reach clear to my waist.
Stripping off my clothes, I stand in front of the fan until I am cool.
Mother's dress is still on the floor where Vera dropped it. I stoop and
gather it to me, a silk ball between my breasts. It slides easily over
my head, down my hips. The armholes gape like dead eyes. I fold the
extra material toward the middle and belt it tightly, knowing that with
all the beer and cookies, I'll soon fill out, maybe even burst the
seams.


\cleardoublepage
\chapter{Under the Stars}

Even though I don't like steak, especially when it's pink in the middle,
I cut it into neat pieces and chew, remembering to keep my mouth closed.
I decide not to ruin dinner, not to tell them now. Instead I tell Mom
about the dance routine I'm doing for the variety show and even get up
to do a few steps.

With a forkful of mashed potatoes, Mom waves me back to my seat. ``Don't
dance on a full stomach.''

I do two quick sidesteps, take the fork from between her fingers, and
lay it on her plate. She giggles as I pull her toward me, bow, place my
right hand in the small of her back, and waltz her around the kitchen.
We bump into the stove. Lids rattle against the pots.

``Crazy. The both of you crazy,'' Dad mutters.

I twirl Mom back to her chair. She picks up her fork and pauses, the
mashed potatoes halfway to her mouth. Her eyes widen, a showcase for
those gold flecks she always gets when she remembers. She reminds Dad
about their first summer when they danced every night at the Walled Lake
Casino.

``Just the stars overhead,'' she says to me. ``Nothing like that around
here anymore.''

Dad clears his throat twice before he reminds her that her father never
knew, that she was too young to go to a place that served liquor.

``Father never trusted you. Said you were too wild; told you that to
your face once.'' Mom slices the mashed potato walls; gravy flows in all
directions.

I wonder if my parents are making this up; I know they're not. I'm
surprised that they ever danced under the stars. Maybe they'll be able
to understand what I want to tell them after all.

``A bad influence on me,'' Mom chuckles, remembering things she isn't
saying.

``Should have married Henry Miller like your father wanted you to,'' Dad
says, quickly drop­ping his eyes to his plate and sopping up the rest of
the gravy with his bread.

Mom puts her hand over his, squeezing gently. ``You can't still be
jealous,'' she says.

My dad jealous? It's tough to imagine. I've always thought jealousy
belonged to my age and movie stars. Maybe tonight's not a good night to
tell them. I better check it out with Mary Jane.

I pull the phone into my bedroom and shut the door. As soon as Mary Jane
answers, I tell her about my parents, almost bragging. In the
background, I hear Mary Jane's mother screaming at one of her brothers.

Mary Jane whispers, ``You better never try it. They'd kill you.''

``I know,'' I tell her. ``But to think they actually did it. I mean, you
know how straight my mom is.''

Mary Jane says she thinks her parents were born adults, little
pinched-faced worriers who started screaming and never stopped. Says she
can't wait to get away to school and have some fun. Asks if I've had my
parents fill out the scholarship forms yet. Says her parents don't want
her to go away to school. All the usual. Strangers, drugs, wild kids, no
supervision. The whole bit. Says she told them she's going into nursing.

I guess if I had Mary Jane's parents I'd lie too. But I don't lie to my
parents and I know the whole discussion will make Mom cry, later tonight
when she thinks I can't hear. I hate to make her cry.

But Mary Jane says to go ahead and get it over with. All mothers have to
cry sooner or later and it's always best to get a thing over with before
you think too much about it and never get around to doing it.

Maybe she's right. She's got a good mind. Even the kids who don't like
Mary Jane have to admit she's got a good mind.

A shrill ringing comes over the line. Mary Jane says her mother sets a
timer when she's on the phone. A five minute limit. Then, the jangling
bell. Says her mother is taking some sort of class in night school on
how to organize time. Says she ought to take one on how to organize her
mind.

``Do it now,'' Mary Jane hisses and hangs up the phone.

My parents are still at the table, sipping their coffees and talking.
They don't see me standing in the doorway, so I move in closer, knowing
that in a few minutes the smiles will disappear. I feel bad about the
whole thing.

``Whatcha got?'' Dad says, seeing the papers dangling from my hand.
``Some more physics problems you need help with?''

He's proud of his science and math abilities, says all the males in his
family have that ability. Every time he works with me, he reminds me
that girls don't inherit that trait, then ruffles my hair and says,
``But you wouldn't want to be too smart anyway, would you?''

I sit down in my chair and push the papers across the table to him. When
Mom gets up to do the dishes, I touch her arm and say I need her to stay
too.

Dad looks at the papers. Says he'll need a couple of days to fill them
out. Asks me why I'm applying to a college in Florida.

I take a deep breath and tell him it's the only place where I can learn
to be a clown.

He throws back his head and laughs, gets up, and pretends to slip on a
banana peel. From the floor, he asks if I've heard the one about why the
clown crossed the street.

I shake my head, thinking how stupid he looks sprawled on the floor.

``Because he wanted to get to the other side.'' His face reddens as his
laugh deepens and I turn my head away. Mom just stares at both of us.

I tell him I mean it. That's what I'm going to study. And Mom tries her
own tactics. Tells me no man would ever marry a woman clown. Imagine the
children going to school and saying their mother is a clown. Mom wipes
the corner of her mouth with the dinner napkin.

Dad's still making corny jokes and no one's listening. He stands up and
pats Mom's hand. ``It's nothing to worry about. She'll change her mind.
When she was four, it was a policeman because he had a gun. Then a
teacher because she wrote on the blackboard. A cashier at a grocery
store so she could ring the cash register. A candy counter lady so she
could have all the coconut bon-bons she wanted. Don't worry, she'll
change her mind.''

``A teacher or nurse I can understand. But a clown? Probably one of Mary
Jane's ideas.''

I run up the stairs. It's been a long time since they've seen me cry; I
don't want them to see me now. I hate them both and I don't believe they
ever danced under the stars.

Mary Jane's voice is reassuring until the timer goes off and her weird
mother starts screaming. Maybe it's just that all parents forget how to
have fun.

It takes four pages in my diary to get it all down. I'm already into
September and it's only March. But it's important that I keep track of
everything because when I become a famous clown, I'll write my
autobiography.

Sometimes, Mom's voice gets louder, comes through the floor like carbon
monoxide fumes. (Dad thinks I can't remember anything about science, but
he's wrong.) She keeps saying it's not a suit­able job for a woman. But
there's a lot my parents don't know about me, a lot they wouldn't want
to know. Sometimes I know how much I'm letting them down. But it's my
life and I want something more exciting than sitting at the table after
dinner drinking coffee.

Two days later, Dad hands me the papers all filled out. Says I have to
make up my own mind. He thinks he's being clever but I know he and Mom
are firm believers in child psychology. They have one whole shelf
devoted to such books. Before, it was always how Dr. Spock says it's
only a stage. Now, it's Dr. Ginnott with his \emph{Between Parent and
Teenager.} I haven't the heart to tell them it's really between me and
me.

It's the next week in school that we have to get up and give our career
speeches. Mary Jane raises her hand to volunteer. She tells everyone how
she wants to be a clown, shows chalk drawings of the kind of face she's
going to use. When she starts the side shuffle and goes into a routine,
James Becker who sits behind me, begins throwing spitballs at her. Even
though he winds up for each throw, she doesn't notice him. She's telling
jokes when a spitball lands in her hair and moves back and forth each
time she wrinkles her forehead. The whole class laughs and she gives me
the high sign as she bows and sits back into her seat next to me.

I can't let my eyes meet hers. Instead, I focus on the next speaker who
mocks Miss Greenstead when he says he wants to be a teacher and fiddles
with the chalk the way she always does. Miss Green­stead makes him sit
down and warns the class to take the world of work seriously.

With my pencil, I doodle Bob's initials on the inside cover of my
physics book. I sometimes think he notices me but he's never asked me
out. No one but Mary Jane knows I have a crush on him and I can count on
her to keep her mouth shut. I've studied the other girls he's taken out
and they all seem kind of flighty and dumb. That's why I groan when I
get my physics test back. I would die if he ever found out I get A's on
them.

When Miss Greenstead calls my name, I move to the front of the room and
direct my talk toward Bob. I can't let myself look at Mary Jane or I'll
start giggling. I tell Bob I'm going to Beauty School. He sits
straighter in his chair and winks at me.

When the bell rings, Mary Jane walks out of the room, holding her books
to her chest and not looking at me. She'll get over it.

Bob strolls over, puts his hands on Mary Jane's desk, leans over, and
asks me to the dance Saturday night. I'm so nervous I hear myself
popping my gum. I swallow it and try to casually lean against my desk.
He moves closer, picks up the physics paper I've been shredding between
my fingernails.

``Hey, I'm not asking you to elope, only to go to a dance,'' he says.

``Sure. Sure, I'd like to go.'' I wish I hadn't swallowed my gum. My
mouth is too dry now.

When I see Mary Jane at the locker after school, she still won't talk to
me. I have to rush to keep up with her and explain that being a doctor
or a research scientist is more prestigious than being a clown. We can
make more money too. As a team, we might find a cure for something and
get our pictures on magazine covers. I can see she likes the idea and
when, four hours later, she tells me that we might even practice
someplace like France or Brazil, I know everything is all right again
between us.

Besides, I can just picture my parents when I tell them. My dad will
lean back in his chair and say, ``Everybody in our family was always
good in the sciences.'' Mom will raise her coffee cup, set it down
without drinking, and finally say in a flimsy voice, ``It's a good skill
to have with small children around. They get hurt a lot, you know.''

I'll ask Mom and Dad not to mention it to Bob. And I'll ask Bob to
button his shirt so my parents won't think he's a bad influence or
anything like that.



\cleardoublepage
\chapter{Until She Gets Home}

Five years ago. At Amelia's wedding, Mother Ernestine takes John aside
and whispers, her wet lips moving against his ear. John says something
Amelia can't hear. Then he links his mother's crepe---covered arm with
his and moves toward her, his new wife.

Still Amelia hears Mother Ernestine's words. ``Take care of him. Fathers
give away daughters and mothers give sons.'' Yet, she and John are old
enough to give themselves, if they want to. And at 35, they both want
to. At least they think so then.

A year ago. At Father John's funeral, Mother Ernestine weeps, dabs at
her eyes and weeps some more. She clings to John's arm, stopping now and
then to rest her body against his as they walk through the cemetery.
Amelia walks behind John's two sisters, careful to avoid tripping on
headstones.

The cemetery has regulations. No towering monuments. Only small,
unpretentious markers, extending no more than two inches above the
ground. Name and dates only. No plastic wreathes. Grounds are
maintained. Cost included in original purchase agreement.

Mother Ernestine becomes angry. She wants a large headstone, carved
marble. Something to mark his life. They show her the contract her
husband signed, all in proper order. For a year, Mother Ernestine tells
everyone he must not have known what he was signing. Father John is
wrong even in death.

John exhausts himself trying to split his time between his mother's
house and his. At both, the dripping faucets, the peeling paint, the
blocked-up sinks claim his time. With each season, comes new
demands---snow, floods, green grass, brittle leaves. He grows tired,
older looking. He loses weight.

Mother Ernestine sends covered dishes home with him, always his
childhood favorites. Big bowls of spaghetti, apple dumplings, homemade
breads, and stews. Amelia's freezer becomes over­crowded and she takes
to dumping food in the garbage disposal.

John buys his mother dresses of many colors. She returns them for black.
He introduces her to widowers. She declines invitations, politely. She
only accepts invitations to the house that come directly from Amelia.
For as she tells Amelia at dinner one night, ``I don't want to intrude
on my son's life.'' Then Mother Ernestine dabs at her eyes with a
lace-edged handkerchief and Amelia excuses her­self to go to the
bathroom.

One night while Amelia and John drink their usual before-bed glass of
wine, Amelia tells John about her promotion at work.

``It was that new account. They liked my ad campaign. Head of the
department and a raise. That's the offer.'' She isn't sure John is
listening, but she continues, ``Of course I told them I'd like to talk
it over with my husband first.''

``How many other women bosses do they have?'' John asks, holding the
wine up to the light and squinting one eye.

``Five.''

``About 150 boss slots. That's only three percent for the women. Doesn't
look good for the company.'' John's mathematical mind reduces everything
to numbers and graphs.

Amelia often thinks John could, on a moment's notice with the use of his
calculator, recite her vital statistics---her birth date, weight,
height, measurements, and life expectancy. At times his eyes remind her
of grocers' scales, moving back and forth trying to find the right
balance.

To break the silence of her thoughts, Amelia says, ``John, stop
measuring. I'm good enough at my work to handle it.''

He laughs, swirls the wine in its glass, and says, ``Take the job. They
need you.''

They do need Amelia. Her latest account is a soil remover that's rubbed
into a garment, dries to a white powder, and removes grease, dirt, and
stains. The woman in the ad rubs it into everything. All spots removed.
A clever tune---a brief funeral dirge followed by the wedding march.
Slow motion. The miracle of it all. The woman smiles. Fade out.

``Do you need me?'' she asks John, setting her wine glass on the small
table, leaving a wet ring where wine has dribbled over the rim, onto the
stem, over the base.

``For what?'' he asks, smiling.

``For nothing,'' she answers.

``I'm teasing,'' he says. ``Of course I need you. Let's go to bed.''

Amelia needs to sleep.

Mother Ernestine comes to dinner at Amelia's invitation. She doesn't
look well. Her skin is drying out and her hair is losing its curl. She
talks about getting a new permanent, getting her hair colored, getting
an apartment, getting old.

John tries to console her. ``You're only 63, Mother.'' He gives her a
hug. If you'd get out more, you'd feel better.''

``I can't go out without your father. That wouldn't look right. Besides
I think I've got cancer,'' Mother Ernestine says, helping herself to
another spoonful of mashed potatoes.

``It doesn't matter at your age,'' John says.

Amelia picks at her food, keeping her head down as the two talk,
wondering if it's the going out or the cancer that doesn't matter.

``Cancer,'' Mother Ernestine repeats between the mashed potatoes and
peas.

``We all die from something, Mother.'' John knows how to handle his
mother's complaints, her dramatic illnesses.

``First your father, now me. Then you and Amelia---dead. Your sisters
are already dead to me. They don't care. That a mother should bear such
children is a waste.''

Amelia thinks Mother Ernestine might make a good advertising agent,
writing the copy herself after she snares the client.

Over dessert, John asks Mother Ernestine to live with them.

Mother Ernestine cries, louder than her funeral sobs. She hugs John,
says she can't think of such an arrangement, says she doesn't want pity,
doesn't need anyone. She can look after herself, she says.

After she finishes her dessert, John drives her home. He is gone a long
time.

Amelia does the dishes, wipes the table, showers, and reads a book until
John returns. She wishes she could go to bed and sleep, have the double
bed to herself for a night.

When John finally returns, Amelia pours the wine.

``If you asked her, she'd come,'' John says.

Amelia already knows this and says nothing.

``With your promotion, you could use some help around the house. Mother
likes to cook and we'd be some company for her. She's lonely, Amelia.''

``I know,'' Amelia says. ``I'm lonely too.''

``That's ridiculous. You've got your job, me, all our friends. How can
you be lonely?

Amelia knows he can't understand. To him, anything that doesn't make
sense just can't be. She doesn't try to explain.

``What do you think, Amelia? It's a good deal for all of us. We have two
spare bedrooms. Plenty of room. She'd be happy with us.''

``You're right, of course.'' How can she argue with his logic. She can't
think of anything that would tip the scales in her favor.

John leans over and kisses her on the forehead, saying softly, ``I'm
glad you don't mind. After all, she is my mother. I owe her a lot.''

``I owe her a lot too,'' Amelia says.

Amelia calls Mother Ernestine from work the next day and invites her to
live with them.

Mother Ernestine is hesitant, finally saying, ``Well, dear, if you
really want me to. I'll be glad to help out.''

John takes care of all the details. The house sells quickly and for a
good price. Mother Ernes­tine's belongings are sorted and moved in with
theirs. Her favorite chair, a pink-flowered chintz, is set in front of
the television, across from Amelia's blue velvet couch. The living room
begins to look like a resale shop.

Amelia works later at the office, becomes more involved with the
clients, is preoccupied with intricate advertising campaigns. She tells
Mother Ernestine and John to eat without her. But they still wait to eat
until she gets home, no matter how late.

At dinner, Amelia is quiet, tired. They try to draw her out with
questions about her work. Like a sullen child, she doesn't always hear
the questions, admonishments.

John gets angry and says, ``You're working too hard.''

Mother Ernestine disagrees with him. ``It's her age, John. She's going
through the change.''

``Can't be. She only just turned forty.''

They talk as if Amelia is not there.

``Women who never bear children go through it earlier,'' Mother
Ernestine says, clearing away the plates.

Amelia has never heard this before. She wonders if it's true.

John is no longer angry.

Amelia still says nothing. She knows nothing to say in her defense.

Amelia goes to bed without her wine, letting John and Mother Ernestine
carry out the ritual, drink the blood-red Lambrusco without her. She
hears them talking quietly.

At work, Amelia is restless, forgetful. She feels senile and begins
writing notes to herself. Tasks that must be done. Words that must be
written. Squiggly lines of letters. She may need glasses to stop the
movement, the blurry words.

Amelia calls the oldest sister. She asks about her nieces and nephews,
the new house, the heavy-set brother-in-law. She waits while the sister
answers, waits to pose the real question.

``Your mother,'' Amelia says, ``is getting on my nerves. She's taking my
place. Could you\ldots''

``The bitch,'' the sister screams, ``always chose John. He can keep
her.''

Holding the empty receiver, Amelia waits for the dial tone. She punches
buttons and listens to the time. ``At the tone, the time will be five
forty-five and ten seconds.'' The shrill beep. ``At the tone, the time
will be five forty-five and twenty seconds.'' The shrill beep. ``At the
tone\ldots''

The shrill beep intervenes. It marks passing time, destroys continuity;
the smooth, calming flow of darkness and light.

Amelia is on time tonight for John's favorite dinner---thick, simmered
spaghetti sauce with fresh herbs ladled over rich noodles, crusty
homemade bread with softened butter. Mother Ernestine spends much time
in preparation. Amelia isn't hungry.

Together, they watch the news. Mother Ernestine in her pink, chintz
chair; Amelia on the blue velvet couch; John in the black, overstuffed
leather recliner that used to be his father's.

It's the usual news every night. An old man is mugged, a fire erupts
after a three-car crash slows traffic on the expressway, the police have
a lead, the mayor is at a conference. Amelia keeps hearing shrill beeps.

There is live coverage of a 45-year-old man who is barricaded in his
aged parents' home. He chased them out with a rifle which he is now
using against the police. His old mother is in her bath­robe, sobbing
into the cameras, telling the world that she doesn't want her son hurt,
that he means no harm. The father, also in a bathrobe, is standing
stoop-shouldered, watching his house.

The son is till shooting, but everyone is out of range. The son isn't
talking to anyone. He won't even answer his mother when she pleads
through the bullhorn.

``Those poor people,'' Mother Ernestine says. ``To have their son go
crazy like that and turn on them.''

``The police will get him,'' John says.

Amelia silently applauds the son, her hands fluttering against her
knees.

Amelia will buy John a rifle for his birthday next month. Maybe she will
buy herself a rifle for Christmas in memory of the son she'll never
have.


\cleardoublepage
\chapter{Waiting}

I hear Marilyn before I see her. Her shoes tapping against the oak
floor, coming fast around the corner.

``For God's sake, Sarah, how long have you been waiting?'' Marilyn says,
pushing her hair into place. ``Are all these others waiting too?''

I nod. ``But we're next. I put our names in as soon as I got here.'' I
motion to the owner, tell him we're ready.

Marilyn slides into the chair the owner pulls out for her, leans her
briefcase against the bottom rung, reaches for the menu he hands her.

``What's all this hocus-pocus stuff? Psychometry, smoke readings, tarot
cards?'' Marilyn asks, pointing to the symbols on the menu. ``Lunch, you
said. That's all, just lunch.''

``This is nonsense, Sarah. I want to eat. Tea room, you said. Just
around the corner from my office.'' She looks past me to the poster on
the far wall: a hand spread wide, palm up, writing too small to read.
``What have you gotten yourself into now?''

``It's Mama, Marilyn, I can't do it.''

``Can't. Can't. Can't. Can't. That's all you ever say.'' Marilyn reaches
over and drinks what's left in my tea cup. She puts the soggy Lipton tea
bag on a napkin, slices it with her knife, spreads the wet leaves.
``Read these. Go ahead. Tell me what they say.''

I see them again. Mama standing behind the screen door, shaking her
finger at me. Her other arm encircles Marilyn's thin shoulders and pulls
her close against her hip. Mama and Marilyn. The flour spilled on the
floor behind them. The sweet smell of sugar cookies.

``It's decided,'' Marilyn says. ``There's only you and me. Someone has
to do it. Even the tea leaves say so.'' Marilyn laughs as she stirs the
brown leaves, shapes them into an ``s.'' ``Come on, Sarah. It won't be
so bad.'' She glances at her watch.

I trace the lifeline in my palm, cup my hand to make the line appear
deeper, stronger. As I reach for Marilyn's hand, she draws back, drops
her hands beneath the table, away from me, the older sister.

Marilyn is the younger, the tougher, the more responsible. Mama says so
every chance she gets. Mama reminds me when she can't sleep and calls me
late at night. ``Your sister is going places. Probably to be a partner
soon in that CPA firm,'' her voice groggy from the sleeping pills that
don't take hold fast enough for her.

And I know Mama is right. My sister is the one who had my house assessed
before my divorce, who made sure I got my fair share out of Charles
before he left, who got me the job typing for a law firm two blocks over
from her office, who worries I'll get fired if I stay too long at lunch.

Women come and go around us. They climb the stairs to sit with psychics.

Marilyn refuses to turn her head, to see these women, to scan their
faces when they come back down. I see them all from my chair facing the
stairway.

``It's too late to do anything else,'' Marilyn says. ``I've already told
Mama you're moving in next month. She's counting on you, Sarah.''

``I can't give up my job. I just can't,'' I tell her, knowing she thinks
I'm just trying to make more trouble for her.

``Christ,'' Marilyn mutters, turning her eyes upward. ``You call that a
job. Sitting all day: typing up legal briefs and answering phones. Just
other people's words, Sarah. That's all it is. Dead end.''

``I like it,'' I say.

It's my first real job in over 25 years. My first since I worked at the
A \& P when I was in high school. Unless you count all that time married
to Charles.

``Yes, other people's words,'' I say. ``That's what I'm good at.''

``This is crazy,'' Marilyn shouts. People look at her. ``Absolutely
crazy.''

She picks up her briefcase, stands, holding the case against her tweed
skirt. I stand too, holding Marilyn's eyes with mine, yanking the
briefcase from her curled fingers. The clasps spring open, papers
cascade onto the table, over the floor.

``Don't touch those,'' I tell the owner who stoops to pick them up. I
put my foot on top of the nearest papers, leaving dirty smudges on the
parallel rows of numbers.

Marilyn sits down, ignoring the spilled papers, whispering through her
teeth, ``Sit down, Sarah. What mumbo-jumbo have they been feeding you
here?''

``You promised,'' I say.

``Be sensible. I promised to meet you for lunch.''

``It's our turn.'' I stand and move behind her, try to lift her from her
seat. She stands, knocking over the chair. I hand her the briefcase.
``You'll need the papers too,'' I say, shoving them toward her with my
foot.

She picks up the papers, puts them into a stack, slips them in a folder
and puts a rubber band around them. I hear the metallic click as she
latches the briefcase.

``Let's go, Sarah,'' Marilyn says. ``Don't make a scene.'' Her face
flushes, voice grows raspy.

``The stairs,'' I say. ``Just up the stairs.''

``I'm late for an appointment, Sarah. Can't keep clients waiting for
long,'' she says, trying to pretend nothing is happening.

``They'll wait. I'm waiting now.''

``If I do it, then will you stop this foolishness?''

We're halfway up the stairs when she pauses, leans toward me. ``Have you
been giving them money?''

I laugh. ``Money? I already paid for both our readings.''

``I mean donations. Lots of money. Have you?''

I don't answer. I know that drives her crazy. She's used to talk, lots
of her own words.

``Well, have you?'' she says again, grabbing my arm.

``Have I what?'' I say, shaking her loose.

``Money. We're talking about money. Money you've given to these fakes.
Money to burn candles or cast spells or do whatever it is they're
supposed to do. You never did understand money.''

``But you do. Isn't that good enough?'' I nudge her ahead of me into the
small room.

Lillian sits in front of us, her flesh overflowing the chair. She
smiles, reaches for the clear quartz crystal, the size of a human heart.

``Ah, Ladies,'' Lillian says, ``please sit down, sign the guest book.''

I sign the book and hand the pen to Marilyn. She writes down Mama's
address, hiding behind Mama, wanting Mama to open any mail that might
come, knowing Mama will drop it in the garbage pail under the sink.

``I do tarot cards, astrology or psychometry,'' Lillian says. ``What's
your pleasure?''

``Nothing,'' Marilyn says.

``And you, little one?'' Lillian asks. ``Have you gone ahead and done it
yet?''

``Soon,'' I say, hoisting Marilyn's briefcase to the table, slamming it
down hard between Marilyn and Lillian. ``What does this tell you?''

``No,'' Marilyn says, reaching for the case. ``You can't do that. It's
mine.''

``No reason to be afraid,'' Lillian says, closing her eyes, rubbing her
hands over the smooth leather. ``Just sit back. Open up to the
experience.''

Marilyn starts to stand up. I grab her wrist, pull her down, watch her
rub the red welt I raised.

Lillian draws back, stares straight into Marilyn's eyes.

``Well,'' Marilyn says, ``your gifts fail you? Can't you see anything?''

Marilyn's eyes survey the room. I try to see it as she does: the
tobacco-stained, flowered curtains tied back with the heavy drapery
cord; the palm-reading poster same as downstairs; the dish for tips.

``Are you not tired?'' Lillian asks. ``Too long carrying others on your
shoulders.'' Lillian shrugs. ``You can shake them off if you will. But,
no, you pile them there. Your ballast. Like numbers for you---solid,
hard.''

Lillian caresses the briefcase, her eyes steady on Marilyn. ``Beware,''
she warns. ``I see you with paper dolls\ldots bending their legs,
dressing them in fashions out of date. You hide them where they can't be
found.''

``It was you.'' The paper dolls Daddy gave me for my birthday, long ago.
Forgotten until now.

``Enough,'' Marilyn says, walking from the room.

I take the briefcase Lillian hands me and run after Marilyn, down those
steep steps, out onto the concrete, into the October afternoon.

``You set me up,'' Marilyn says as I catch up to her, stand beside her
as she stares at the five-tiered wedding cake in the bakery window.

``Want something?'' I ask, feeling guilty she hadn't eaten. But then she
probably skips lunch most days anyway.

``Yes. I want you to keep your promise,'' she says to my reflection in
the window. ``Stay with Mama. Won't be for long, you know.''

``Too long,'' I tell her. ``I don't want Charlie's words, or your words,
or Mama's words. Not anymore. I won't listen.''

``There's nothing else to do.'' She pauses. ``You planned it all didn't
you? With that fat woman. You told her about me, about Mama. You gave
her money to do it.''

Why can't she just say she's scared---scared and tired? Even when we
were little and I got her on the high diving board, she wouldn't say it.
She jumped while I had to push past all the kids waiting behind me,
crying while I climbed down the ladder. But not this time. I won't do
it. Simple as that.

``Only six months, Sarah. That's all they give her. That's not long.
After that I'll find you another job. I promise.''

``That's only what Mama says. You know Mama.'' I take Marilyn's hand,
turn it palm upward. It's all there for the reading. I should have
guessed sooner. ``There's nothing wrong with Mama is there?'' I shake
her shoulders, ``Answer me.''

Marilyn stumbles over her briefcase, falls. I kneel beside her.

``Why?'' I kick her briefcase against the wall of the bakery. ``Answer
me. How could you do it?''

Marilyn fingers the hole in her nylons, just above the right knee. ``I
can't go back to the office like this.''

``Now. Tell me right now. Go on.''

``You wouldn't understand. You had Charlie. Mama had me. Just me.''

We both hear the cars passing in the street. No one stops. No one
speaks.

``It's not too late for you,'' I say, knowing it may be.

``I'm in for a big bonus this year,'' she says, lifting her head. ``May
even be offered the partner­ship. If not this year, next year for
sure.''

``Marilyn, I'm leaving.'' I say it without bitterness. ``I need to go
other places, hear other voices.''

``They did this to you, didn't they? Those people at the tea room. That
fat woman. It's just a con game, Sarah. How dumb can you be?''

``Dumb enough.''

``But you don't have any money. You never were good at thinking things
through.''

I hug Marilyn, help her to her feet, hand her the briefcase.

``You'll be back,'' Marilyn says, turning toward her office. ``I'll tell
Mama you can't stay, at least not yet.''

``You do that,'' I say, waving and smiling at Marilyn as she walks
away.''

{
  \large
\begin{center}
  $\star \ \ \star \ \ \star$
\end{center}
}

The post cards I write to Marilyn and Mama from other cities, other
states, are brief. There's not much to say. But sometimes I do think
about going back for a visit, some Halloween when the kids are all
dressed in costume and the moon is full. And I do stop at palm readers
in different towns whenever the mood hits me. But I still haven't run
across anyone who could read me like Lillian did.



\cleardoublepage
\chapter{When the Birds Sing}

I pay off the Checker cab driver, tipping him a dollar more than I
usually tip, because I'm feel­ing a bit frightened and on edge. Sort of
excited, too, I must admit, but that's probably nothing but heart
palpitations. As we had passed by houses in Highland Park with condemned
signs on them, by the junkyard with the tires all piled up and that
terrible acrid smell of burning rubber so bad I had to hold my
handkerchief over my nose, by that guy with the cardboard sign with
words misspelled telling people he ``wood work fer fud,'' by the old
Methodist Church with graffiti spray-painted on it that no church-going
person ought to see, the cab driver kept reassuring me that this was
just something we had to pass through to get where I was going.

As I snap my purse shut, I'm sorry I didn't tip even more. He probably
could have used it if those photos taped to the dashboard were his kids.
But it's too late now. Now all I can do is move ahead with what I came
for although I don't know why I let Eartha bully me into being places I
don't want to be and doing things I don't want to do, don't even want to
think about.

I can hear Eartha's voice echo in my ear, the way it did on the phone
yesterday after dinner, ``Safe. You worry about safe when I need your
help? There's such a thing as being too safe, you know.''

Nothing wrong with taking precautions, if you ask me. Fewer people would
end up as news stories if they were more careful. Just as the policeman
told us to do when he spoke to our Golden Oldies Senior Citizens Social
Club last week, I clutch my black leather purse with its thick straps
and gold metal clasp tight against my stomach and stand straighter when
two teenage boys in dirty tee shirts and jeans bump against me as they
tromp up the steps to Ginopolis' Pizzeria. I feel their heat as they
pass by, their sweat against my bare arm.

Some women carry white purses in summer while I find black suitable for
all times of year and good leather lasts forever, I had told myself when
I paid more than I should have for it at Lord \& Taylor's over twenty
years ago. I know what's in every compartment, where I keep my social
security card, my teacher's retiree discount card, and photos of my
nieces and nephews, all grown now with children of their own. I'm not
about to let any young kid yank it from me like you read about in the
papers all the time. Not that I'm accusing those two teenagers of being
thieves. But you can never tell. I wait until they're inside before I
try to get my bearings.

At first I walk past the Boston Tea Room and have to retrace my steps
until I find it sand­wiched between the pizzeria and a real estate
office with pictures of prime property taped to the win­dows. It's not
at all like the elegant Russian Tea Room in New York where I went once
with a group of teachers before going to a Broadway play. Instead,
Eartha saw it advertised in the \emph{Psychic Phenomena News}, she had
explained last night when I asked her where she heard of it. She thought
it sounded more reputable than Madame Tina's, the psychic healer, or
House of Spiritual Renewal. Besides, I don't need healing or renewal
although heaven knows Eartha sure could, what with all those donuts and
chocolate bars she eats.

The wooden door squeaks as I pull at its sticky brass handle and squeeze
through the opening. The candle-lit interior is heavy with incense that
starts me sneezing. My eyes water. I hear Eartha's voice out of the
darkness, ``Roger's here. I just know it. Come on,'' she urges. ``Just
four steps straight ahead. Just follow my directions. I won't let you
trip.''

``Not until I can see where I'm going,'' I snap, wiping my eyes and
waiting for them to adjust. Directly on my right is a poster of a hand,
palm upturned, lines intersecting. It spooks me: a hand dis­embodied, no
arm connected to a shoulder, nothing human about it.

``Ah, come on. Roger will show you the way,'' Eartha gives me that
straight-up hand salute, palm turned out, that expects the world to stop
and notice. An affectation if you ask me. Even told her that once. Told
her it was the kind of thing that works its way right up my nose like
pollen.

``Darn it,'' I mutter as the clasp on my purse catches a loose edge of
the poster and rips it right across the wrist, right where the lifeline
starts. Wonder if that means someone just got zapped off the streets of
Detroit. Gives me the creeps to think Eartha really believes this stuff.
As if lines on your hand can speak your destiny, lock it in. I only
believe what I see and then I have to see it more than once to make sure
it's not a speck of dust or a gnat that's gotten under my eyelid.

``I never should have come,'' I tell Eartha after seating myself and
ordering comfort herbal tea that's advertised as a mixture of leaves and
roots that soothes frazzled nerves. ``This is just plain non­sense.
Roger is dead, Eartha. Been dead almost fifteen years this past May.
Heart attack, just like that. Only you persist in acting like he's still
hanging around after all that time,'' I say and wipe my eyes again.
Sometimes all her hocus-pocus stuff really irritates me. Mostly, I just
ignore it. ``Besides, even when he was alive, he wasn't the kind of man
who stood around waiting on things. I don't know why I ever listen to
you.''

``Sure you do,'' Eartha says. ``You know you have the power even though
you won't admit it.'' She absently twirls a strand of her brightly dyed
red hair around her index finger as if she can straighten out the kinks
from her last perm. ``Besides that, you're a helping kind of person.''

``So, what kind of help do you need? You made it sound so urgent last
night.'' I watch Eartha tear shreds from the outside edges of her napkin
and let them flutter down onto the black Formica table top where they
lay like pigeon feathers. ``Out with it, Eartha,'' I say in my
schoolteacher voice and lean across the table, closer to her. ``Did you
look at the man behind the counter flipping over those cards with funny
pictures on them? It's not safe for two women our age to be here.''

``Tarot cards. That's all they are. They can tell your past, present and
future.'' Eartha looks over her shoulder at the man laying out the
oversize cards, crossing some over others.

``My future? Not much of that left,'' I say. ``And I know my past. It's
gone.'' At a table across the room a young woman in a yellow sundress
rubs her bare arms, stands and walks behind the man with his Tarot cards
and enters a curtained space behind him.

``Oh, the way you go on,'' Eartha says. ``You're healthy as a horse. And
seventy-three isn't old anymore.'' She presses her chest right over her
heart and her large breasts rise and fall with each breath as if she
wants me to ask about her health, which I refuse to do. ``What is it you
need my help with?'' I ask, separating each word with wide spaces,
hoping it's something I can get over fast so I can get back home and
weed my vegetable garden before it's completely overgrown.

``It's this skirt,'' Eartha says. ``Look at how these little people have
changed since I last wore it. Doesn't this man look like Roger? I swear
it's Roger,'' she says, pulling up the long chartreuse skirt until the
embroidered hem rests flat above her knees.

``Fiddlesticks,'' I say. ``You just want some excuse for showing off
your legs. Disgusting, if you ask me. All those varicose veins and
you're not even wearing hose.''

``No one says fiddlesticks anymore.'' Eartha taps her long red-lacquered
fingernails against the table---clickety-clack, clickety-clack until I
can't stand her sound anymore---and then I see a silver star flash at
the tip of one nail and grab her fingers.

``What's this? Some cult thing like I see on television? I wouldn't put
it past you.'' Although nothing could make me tell Eartha, I envy the
way she stays up with things and never cares what any­one else thinks.
That woman does just what she wants and her crazy ways do stir things up
in my life.

``Afraid to look, aren't you?'' Eartha folds her hands over her huge
belly and gives out with that rumbling, snorting laugh like she has a
fishbone caught in her throat. ``Come on, just take a peek. Maybe that
woman who sold it to me years ago when I was in Ecuador with Roger was
just making stuff up when she promised it was a magical story skirt with
bird spirits sewed into every stitch.''

``Hogwash.'' My hand shakes as I set my teacup back into its saucer with
a slight clink. ``Pure hogwash, as my father used to say. And, just for
your information, I'm only seventy-one.'' I can't stand it when Eartha
starts talking like this, as if anything could happen like a horse being
born with a rabbit's head or a grown man living in a thimble---all that
crazy stuff she reads in that \emph{Enquiring Minds} tabloid she buys
every week at the grocery store.

``She did too tell me that.'' Eartha fingers the hem of her skirt, runs
it through her fingers, back and forth. ``And she said the birds would
sing their secrets when the time was right.''

``Can we just get on with whatever we came to do?'' I feel under my
chair with my foot to make sure my purse is still there. ``Now that I
drank my tea can we go?''

``You always were impatient,'' she chides. ``Just do, do, do. Then you
don't have to think about anything.'' She waves the Tarot-card man away
when he sets another cup of tea in front of her. ``I told you I need
your help. And I do.''

``So what do you want me to do?'' Even though I block my ears with my
hands, I can hear Eartha chanting, ``Do, do, do, do, dupey do.'' Over
and over again. She even swings her skirt from side to side as if she
were dancing a cha-cha.

I swear Eartha gets more juvenile every day, worse even than those
second graders I used to teach the same things to, year after year,
until the school board did away with phonics and refused to let the
children sound out words anymore.

Finally Eartha seems to get tired of hearing herself and stops. When I
drop my hands to the table and fold them, she says, ``Don't do, I don't
want you to do anything.'' She spaces the words far apart and enunciates
each one. ``What I need is for you to just offer yourself up. Use your
power. That's all. Hear the birds. I know you can do it if you want, if
you open your mind to silence, you will hear them sing their songs.''

Eartha stares at me, as if she's said something profound, as if she's
giving directions for a test, as if she expects me to take her
seriously. After all these years, I've certainly learned it's better to
humor Eartha, let her have her way. Even Roger knew that and he wasn't
what anyone would call a sensitive man.

Eartha points out one of the figures riding a llama. ``Don't you think
that's Roger? Now you can see why I couldn't explain it last night. I
had to show you.''

As I look where her finger points, the man on the skirt seems to move,
take shape and reach out toward me. Must be this tea, I think, and push
my cup and saucer to the middle of the table.

``Scared?'' Eartha asks the question like a schoolyard dare. ``Raises
goose bumps on my arms when my Roger comes back to me. Last week his
voice came to me. Woke me out of a sound sleep. Never could get much
sleep when Roger was alive, what with his snoring and all.''

``Nonsense. You know this stuff is a bunch of hooey,'' I tell her,
reaching down and opening my purse for a tissue to dab my forehead. I
can feel the pens and pencil, my wallet, my breath mints in all their
normal places and the tissues that come out one at a time. I hunch over
her skirt for a closer look at the man who seems to have climbed down
from the llama and is drinking now from a tin cup. ``You moved the
hem,'' I accuse her. She shakes her head at me and I just mutter, ``Yes
you did. You moved it.''

Eartha stares at the astrology chart on the wall, gets up to find
Roger's sign, traces the large circle holding her finger against a line.
``Born on the cusp. See. That's what I blame for Roger's drink­ing
problems. Sort of caught between signs, never knowing which side to come
down on. Still, he was a good man.''

``Hogwash,'' I say, ``Roger did what he wanted, when he wanted and
dragged you along. That man never appreciated what he had in you.
Complain, complain, complain. That's all he ever did.''

Tracing one of the multi-colored birds on her skirt, Eartha says,
``Well, I wasn't a saint, either. You never really knew him. We got
along in our way. We had adventures. Always something new with Roger.''
She tucks a loose strand of blue thread under the green head feathers of
the bird and smiles. ``I miss Roger. All these years and I miss him. I
just know that's why he comes back to me from time to time.''

The bird seems to spread its wings and lift its head. Its throat ripples
and I hear musical notes. Must be this air conditioning after all the
heat we've had this past week that makes me feel faint or maybe it's
time to get my eyes checked again. Over Eartha's left shoulder I see the
young woman pull aside the curtain and leave the room and walk past us
into the afternoon heat.

``Eartha,'' a voice calls from behind the curtain. ``Eartha Bridgeway,
it's time now.''

Eartha stands as I grab at her hand and miss. ``Sit down,'' I tell her,
but she doesn't seem to hear me. ``Sit down,'' I say louder. But Eartha
is already moving toward the curtain and I watch it draw behind her.

I order renewal tea from the man with the Tarot cards and he brings me a
small ceramic tea­pot with matching cup and saucer. Looks translucent
like real china. The pink cherry blossoms drip from the painted branches
and robins rim the edge of the saucer, feathers puffed as if ready for
flight. I pour the tea, even fill up Eartha's cup. I drink slowly from
my cup and listen to the silence of this room shut off from the outside
sounds. I try to recall the noises I heard before entering this room:
tires against pavement, pigeons screeching, teenagers and derelicts
brushing against each other, sirens. All I can hear is the sound of
myself swallowing.

Then, the chirp of a bird I do not recognize, have not heard before,
lingers on the rim of my ear before sliding into my ear canal. I reach
down and lift my purse to my lap, stroke the worn, famil­iar leather
before opening it and removing a breath mint. I unwrap it and place it
on top of my tongue. The mint melts there, dripping back into my throat
and I swallow the sticky sweetness.

I conjure up an image of Roger sitting in Eartha's seat. He drinks a
beer straight from the can and smiles at me. ``Go away,'' I whisper,
``before Eartha gets back.'' But he sits there just smiling as I clutch
my purse and snap and unsnap the clasp, enjoying the hard, metallic
sound it makes as it closes. Roger cups his ear and waves his arms as if
he's conducting an orchestra and I hear birds screech like untuned
violins.

Roger reaches out his hand to me. I push it away just like I did that
night years ago when he came next door to my house drunk while Eartha
was at one of her night school classes. Can't even remember what she was
taking that year. Could have been Egyptian history, algebra, or maybe
even pottery-making. She was always studying something. Birds weren't
singing that night when he turned on the radio and tried to get me to
dance. And me with my hair in curlers, wearing only my old white
chenille bathrobe and having to get up early for classes the next
morning, I needed my sleep. To get him to go, I had to dance one dance,
a waltz, I think, and he kissed me on the cheek and whispered the only
thing he said all the time he was there. I still remember his words,
``You are Eartha's other self.'' Then he waggled his finger teasingly
and said, ``You shouldn't let her have all the adventures,'' before
tripping on the step and falling sideways on the grass were he sat and
laughed while I locked my door and went to bed. All these years and I've
never told Eartha about that night.

I squint my eyes and watch Roger fade slowly, his smile the last to go,
first the teeth and then the lips as the birds now sing in soft harmony.
The Tarot-card man just keeps playing his cards and does not look up.
Maybe I'm going crazy but the song seems to be a waltz, a smooth gliding
sound in triple time. So delicate and graceful. Captivating. And, yes,
romantic. I never much cared for waltzes before but this one seems
different somehow. If only I could place the tune or name its title, I
would feel better.

``Excuse me,'' I say, walking over to the man laying out the cards. I
wait for him to look up, before continuing in a voice loud enough to be
heard over the music. ``Do you know the name of that waltz?'' The whole
time I keep watch on my purse at the table even though there's no one
around to take it.

``Ah, the magician in a foundation position,'' he says as he turns over
a card. ``Means you're strong and can have anything you want.''

``The waltz?'' he says, flipping over the next card in his hand. ``Look
at that. I just knew it would be the death card.''

``Please, then, would you mind turning off the radio or the tape or
whatever you have going? It's the same waltz playing over and over again
and it's getting on my nerves.''

``Nothing I can turn off,'' he says, kneading the back of his neck. ``No
radio, no tape, no cassette, no nothing we're playing. Maybe you could
turn it off.'' As he stares at me, I note that one of his eyes looks
greener than the other, maybe even a slightly different color, more of a
hazel in his right eye.

``Must be the hum of the air conditioner, then,'' I say, shivering and
wishing I had remembered to bring my pink sweater. The curtain is still
down behind him and I wonder what Eartha is doing back there and who
she's doing it with. None of my business I tell myself and turn toward
the table where I can still see my purse nestled against the table leg.

``Wait.'' He puts his hand on my arm and I draw away instinctively.
``Aren't you curious about the death card? Most people are. They don't
even want to see it.''

If he thinks some card with a black-hooded character carrying a scythe
is enough to scare me, he has another think coming. ``Not really. It's
only a card,'' is what I say aloud.

``More than a card,'' he says as he scoops the card into a single stack.
In that position, the death card is positive. Means birth and renewal,
moving ahead in new directions. A very good card, in fact.'' He looks
behind him and sees that the curtain is still drawn. ``Want me to read
you before your friend comes out?''

``No, thanks. I can read myself.''

``Of course,'' he says. ``I know you have the power. You should use it
more often.''

When I get back to the table, Roger is sitting in Eartha's seat again
with his arms positioned as if he were dancing. I wave my hand at him
and he goes. None too soon, either, because I hear Eartha's feet against
the wooden floor before I see her lift the curtain and walk my way.

``Can we go now?'' I ask before she sits down. I lift my purse from the
floor and stand across from her. I watch as she rubs her left arm and
sinks into her chair. ``Are you okay?'' I ask standing above her and
then sit down again when she doesn't answer. She just looks at me with
that determined stare she gets when she wants me to do something I don't
want to.

``I won't do it, whatever it is,'' I tell her before she can say
anything. ``Let's just get out of here.''

She reaches over absently and lifts my hand from my purse and holds it
in hers, slowly rubbing up and over each knuckle. With her other hand
she fingers the embroidery at the hem of her skirt. ``You are my dearest
friend, my only family since Roger passed away. You are me and I am you
in so very many ways.'' She stops to catch her breath and lets go of my
hand.

``Oh, fiddlesticks,'' I say, embarrassed since I'm not used to Eartha
going on this way. And the waltz continues to play the unnamed tune.
``You're just tired. Did something get you upset? What happened in that
room anyway?''

``Oh, no,'' Eartha says. ``Nothing happened that I didn't expect. The
woman read my palm while the birds sang their story. And you heard, too.
I know you did. She said you summoned them. Said I should tell you.
Said\ldots''

``Tell me what? Why can't you just get to the point, Eartha? Why do you
prefer to go all the way around the block and to the intersection before
telling me? Why can't you just spit it out. Could save us both lots of
time. And just for your information, I don't hear things that aren't
there.''

``Still the impatient one, aren't you? You always did like to get right
to the bones, no padding no soft-peddling.'' She takes a sip of the tea,
cold by now, and then drains the cup, the same way she drank the white,
chalky barium before her x-ray at St. John's hospital last month. That's
why Roger keeps showing up. He's as impatient as you are, always was,''
she says with a quiet chuckle. ``But that's okay, I'm ready.''

``No, Eartha. You're just imagining things. Can't be as bad as you
think. Is it your heart? Is that it?'' I twist my fingers in the leather
strap of my purse and hang on. ``Is it something else? Modern medicine
can work miracles, you know.''

``Doesn't really matter,'' she says. ``I don't want miracles. I want one
last adventure. One last trip to Ecuador to hear the entire story that
the birds sing. I want you to go with me.''

``You know I don't like to fly. Just last night on the news I heard
about a big plane crash. Metal everywhere. And people's belongings just
scattered across the ground.'' I watch Roger take shape behind Eartha's
chair and put his two hands on her shoulders while he gives me that look
he gave me when he came over to my house drunk that night so long ago.

``At the end of the month, we leave.'' With a flourish, Eartha pulls two
plane tickets from her jacket pocket and drops them in the middle of the
table.

``If this is one of your tricks to get me to do something I don't want
to do, so help me Eartha, I'll get back at you.'' I read our names on
the tickets to Ecuador and the dates. ``We can't spend a whole month
there. What about my vegetable garden?'' Although I must confess, the
whole thing sounds exotic and madcap. A real adventure as Eartha would
say. And why not? What has 71 years of being safe and predictable gotten
me?

``What about it? What's a vegetable garden next to hearing the birds
sing? Maybe we can find you a skirt just like this one that Roger got
me.''

``I already hear them singing and they're driving me crazy,'' I say,
sticking my fingers in my ears, trying to block the sound.

``Ah, ha. Caught you.'' She laughs her deep belly laugh so loud it
drowns out the waltz for a moment. I catch the rhythm of her laugh and
laugh until my whole body shakes and my purse falls to the floor.
``See,'' Eartha says, ``you finally admit it. You have the power, just
like I always knew.''

``Okay,'' I agree. ``If you go, I'll go. After all, I can always buy my
vegetables at the grocery store.'' Sometimes I have to agree with her
just to get her to calm down. If I change my mind, maybe she can get her
money back for the tickets. And I can go to the doctor with her next
time. Hear for myself the way things are.

The card-turning man is shuffling cards as we leave and doesn't even
look up as we move to the door, shut it behind us and blink in the
bright afternoon sun, too hot and bright after that air conditioning.
Eartha is so busy chattering on about all the things we're going to do
in Ecuador that she doesn't even seem to notice the beads of sweat
running in ripples down her cheeks, smearing her makeup.

``We forgot to call a cab before we left,'' I say to Eartha who stops to
rest in a shaded doorway of a shop.

``So, we'll take the bus,'' she says. ``I always take the bus. Gives me
company for the ride.'' Eartha wipes her forehead with the hem of her
skirt as she peers into the shop window.

``Eartha, that's just plain indecent,'' I say as I move in front of her
to block the view of passers-by. Brushing the skirt from her hand, I see
a tiny figure enlarge for a moment and then grow small again. ``Here,
let me brush this off your skirt. I think you've got something on it.''
I raise the hem slightly and peer at the figure that looks like me. I'm
standing by a tree with another woman who looks like Eartha. We're
holding baskets on our head. They look heavy, as if they're filled with
water, maybe even grain to bake into bread. As I bend closer, I see
there's only one woman. Maybe there always was only the one woman but I
would have sworn there were two.

``Look,'' Eartha says, pointing to the multi-colored woven purse with a
shoulder strap in the window. ``I saw those when Roger and I were in
Ecuador. We must get one. It'll make us feel like real world
travelers.''

I follow her inside and turn to look above the door at the small bells
that sway as the door opens to chime our arrival. ``Oh, no,'' I say,
``Oh, no. My purse is gone. I left it at that other place. We've got to
go get it now.'' I feel naked without my purse, as if somehow my whole
life has been misplaced.

``Too far,'' Eartha says, ``and too hot. We can call them when we get
home and ask them to send it to you.''

``I've got everything in there. How could I ever replace everything?'' I
link my arm with Eartha's. ``It's life or death. I've got to get it
right now.''

``Nothing's life or death,'' Eartha laughs as she swings me around as if
we were partners at a square dance. Eartha looks at her watch.
``Besides, they close at 3:00. Door would be locked if we went back
now.'' She's still holding the purse she was looking at and slips it
over my shoulder. ``Looks good,'' she stands back and nods her head.
``That one's yours. Just look at all the purple, blue, violet. All
shades. And the little people and birds woven around the band in the
middle. Made for you.''

``What do the people and birds stand for?'' Eartha asks the young girl
behind the counter who looks odd with a tattooed butterfly on her hand
that seems to move its wings when she spreads her hand.

``Who knows,'' the girl says, popping her gum. ``Do you wanna buy it?''
I can't take my eyes off the movement of the butterfly as the girl's
hand seems in constant movement: brushing lint off her black t-shirt,
scratching a spot on her left arm, pushing the hair off her forehead,
rubbing the glass counter where she leaves streaks.

``Those are from Ecuador, aren't they?'' Eartha asks and the girl nods
without even looking. ``We're going there you know.'' The girl doesn't
say anything, just rings up the two purses Eartha hands her and puts
them in a large bag. I got the rust-colored earth tones for me,'' Eartha
says, looking behind her where I wait at the door.

All I can think about is getting my purse back from that card-playing
man, wondering if he will look through all my photos, check all the
zippered compartments, or take the money, not that I care about the
money. I just can't stand the thought of someone looking through my
private things.

As we ride the bus back, Eartha keeps showing me the tickets and the
purses and chatters on about what she and Roger did when they went to
Ecuador and what she and I will do this time around. She keeps talking
about finding the woman who made her skirt. I don't even need to nod or
say anything. I stare out the bus window thinking about my lost purse
and watching the people and build­ings rush by even though I know we're
the ones moving as we're cradled in the belly of this green and white
bus.

***

It's been a hot summer and it's still sizzling as I sit in this plane
and say nothing when the busi­nessman's arm bumps my elbow from the arm
rest. I check again to make sure that my tickets and passport are in
that blue and purple purse Eartha bought for me. I hold the purse in my
lap against the skirt that Eartha gave me, the chartreuse skirt with the
people and birds. I stroke the birds and tuck that same blue thread
under the green head of one of the birds, just as Eartha had done a
couple of months ago. Was that June? Seems like ages ago.

Even when she got really sick, Eartha wanted me to move up the flight
dates so she could go with me to Ecuador. Not that I had made up my mind
to go anyway but then I never had the heart to tell her that. Instead, I
postponed our flight and took all the books the library had on Ecuador
and read them to Eartha. We looked at the pictures together to decide
where we wanted to go and where we didn't. Neither of us really wanted
to ride a llama but Eartha said that was something Roger would have
absolutely wanted to do. We both laughed about Roger riding on a
woolly-haired, cud-chewing beast since he wouldn't ride anything but in
his car at home, not even buses, which he had said made him nauseous.

When I had talked to Eartha's doctors about putting her in the hospital
again, they all agreed there was nothing else they could do for her. Not
that they didn't try all kinds of medication and even surgery at first
until Eartha made them unhook her from all the machines and just stopped
taking her medication so she could go home.

One day while we were looking through the books, Eartha had made me
promise that I would go to Ecuador myself if she wasn't well enough to
go. I had promised. It was then that she insisted I take the chartreuse
skirt. Even when I tried to refuse, she got up out of bed, pulled it
from the hanger and handed it to me. ``If it goes,'' she said, ``then I
go with it.''

Whenever I finger the figures on the skirt, they move and grow with the
heat from my hand. I find the man who looks like Roger and the woman who
looks like Eartha. They are standing together on a tiny road with houses
on both sides. The blue, green, and yellow birds are singing the same
waltz again and the title comes to me just like that: ``Blue Danube,'' a
crystal blue river you can float on through Germany, Hungary, Romania,
and finally into the Black Sea.

The plane engines rev up, loud in my ears but not loud enough to drown
out the song of the birds. This whole week, I had been dreading this
moment when the plane readied itself and lifted its wheels from the
ground. I probably would have backed out at the last minute if I wasn't
the kind of person who keeps her promises. Once I say I'm going to do
something, I do it.

``First trip to South America?'' the man beside me asks politely,
removing his elbow from the armrest.

``My first flight anywhere,'' I confess, remembering the library books
still stacked on Eartha's night stand, her empty bed stripped of sheets
and my black purse that I told the card-playing man to just hold until
next time I am there.

\cleardoublepage
\thispagestyle{empty}
\part{Flash Fiction}



\cleardoublepage
\chapter{Archeological Dig}

In my father's garage, I uncover artifacts of his past lives---my
third-grade math tests boxed with his wife's valentines, Merrill Lynch
statements filed back to the 1950s, his mother's note when he escaped
the Ohio farm and hitchhiked to Detroit, an empty box of Miracle Grow.

What I did not find spoke loudest.



\cleardoublepage
\chapter{By the Book}

Muriel knew all the ways to get a man. After her fifth divorce at
sixty-five, she knew how to get rid of them, too. She planned to write a
book about it some day.

``Ed, over here. You're in my grief support group,'' Muriel called and
pulled out a chair for him.

Ed stood stunned for a moment. ``Edith wore her hair like yours. Same
color,'' Ed said. ``And the locket. Just like the one I gave her years
ago.''

Muriel leaned over, pulling the locket from between her rounded breasts
and opening it to show a photo of a handsome man. ``My former husband, a
good man. Gone a year now.''

``Edith died a month ago,'' Ed sniffled, leaning toward Muriel, holding
the locket, viewing the photo inside.

Wanting to pull him closer, Muriel satisfied herself with his moist
puffs of breath against her chest. She closed the locket gently,
skimming his cold hand with her warm one.

Muriel blushed, smoothed her skirt. ``Maybe I remind you of your wife.''

``No, that's not it.'' Ed paused and stared at Muriel. A steady stare,
eyes narrowing slightly. Then he snapped his fingers. ``That's it. The
hospital.''

``That's okay,'' Muriel soothed. It's natural to associate new
experiences with your wife. Every­one does it after a loss.''

Muriel picked up her pen, started to sign her name as group leader on
the attendance sheet. The pen didn't write. She flicked it with her
fingers as if she were releasing air bubbles in a hypodermic needle.

Ed paled and began to shake as the overhead lights dimmed and then went
out.

``There, there,'' Muriel said. ``It's all right. Don't be afraid.''

``You are the one,'' Ed shouted, recognizing the soft voice of darkness.
He scraped back his chair and grabbed where he thought her arm would be.
``It's her. I'm sure it's her.''

The lights snapped on, showing Ed with a death grip on Muriel's arm.
``Let her go,'' others in the group shouted before they saw the two
police officers rushing forward to save Muriel from the new guy in the
group, a madman for sure.

The taller police officer wrestled Muriel's arms behind her, snapped on
the cuffs and gave the standard warning. ``Good job, Ed,'' the officer
said.

``Oh, Ed. You would have made a wonderful sixth husband,'' Muriel said.
``Edith didn't have long anyway and you were both suffering.''

``The volunteer who was always putzing about Edith's room,'' Ed said.
``That's who you are. You stole her locket. You killed her.''

``I only borrowed the locket. You would have had it back when you
married me.'' Muriel's voice was harsher than the soft one he had
recognized in the dark. ``Besides, it worked last time. Even this time
it might have worked. You know, give us something in common.''

Muriel hated hearing Ed's sobs. She knew in time he would have grown to
love her. The good side, Muriel thought to herself, was that she would
now have plenty of time to write her book.



\cleardoublepage
\chapter{Charitable Visit}

Martha ran into the living room, a mouse trap dangling from her ring
finger.

``There, there,'' Reverend James soothed his wife as he freed her. ``How
ever do you get in these fixes?''

``Caught like a rat,'' I answered, scratching my leg cast.

``We'll pray for you,'' Martha said, flashing the money from my lingerie
drawer.



\cleardoublepage
\chapter{A Comforting Woman}

When I poured that Speedway gasoline, all five gallons, into the yellow
jacket nest down by the back gate, I never figured Erlene would get as
fired up about those buzzing devils with stingers.

``You're an old pig-headed fool, Timothy Morton,'' Erlene said, bringing
me to, by dumping a bucket of cold water on my head. ``You never do
listen. Told you to just let them be.''

``Thousands came at me, chased me around the cherry tree and back
through your marigolds. They flew right up under my best flannel
shirt.'' I moaned, hoping she would make some allowances for good
intentions.

``Indecent,'' I mumbled through puffed lips as Erlene just yanked off my
clothes every which way. ``No!'' I shrieked as she went for my boxer
shorts.

``Don't go thinking I'm after your old scrawny body,'' she said, ripping
cotton in her bare hands, tossing bits and pieces on the grass.

Above me, I could see the yellow jackets skedaddle, probably tuckered
out from all that sting­ing and the sound of Erlene's screeches. I
rubbed my hands over the welts on my face, my arms, my legs, even places
I wouldn't have thought those dive bombers could find.

Erlene slapped my hands away. ``Don't be scratching those bites. Spreads
the poison.'' She eyed me up and down.

I felt again the long-ago desire when I married her quick-like in the
Church of God, just like she wanted back in 1941.

Almost lost my arm flapping it out the window when I left on that bus
for San Antonio and stayed until I was shipped out to places I had never
been, rifle in hand.

My thoughts hopscotched back and forth. A whole swarm of kamikaze yellow
jackets were after me. I couldn't hide, couldn't outrun them. I felt
their bites like bits of shrapnel pierce my skin. I was dying.

``Don't you pass out on me,'' Erlene scolded. She yanked me up by an
arm, got me stark naked into the truck. I rested my head against the
window, heard the helicopters overhead.

Leave it to Erlene to push our old Chevy pickup truck to its limits,
speeding, hitting every pothole, slamming on the brakes for that fool
squirrel that sprinted across the road.

She could yank up old complaints by the bucketful from the well I dug
over thirty years ago. ``Never did have a lick of sense,'' she started.
She had plenty of proof. ``You bought this old pickup for more than it
was worth new, I swear. Did you listen to me when I warned you the new
preacher's wife was a gossip? Of course not. You had to go and tell her
my sciatica was acting up and all those church women swarmed over my
house cleaning and fixing what I didn't want fixed. And that shed you
insisted on building right up near the house with tar for the floor.
Can't even get my hoe out in sum­mer without my feet sticking. Could fry
an egg on it, gets so hot. That is if we had any eggs. Darn fool hens
won't even lay half of what they did after you ordered that new cooling
contraption from the cata­log. Makes so much noise, scares them to
death. Believe me, I know how they feel about death. Thought you were
dead for sure when I heard that old shotgun go off during deer season
and found you hanging upside-down with your ankle caught in the crook of
the apple tree. You were supposed to be getting apples for canning, not
hunting poor defenseless things. So what, they eat our tomatoes? You
won't eat venison anyway `cause it messes up your bowels. And how could
I ever forget\ldots''

Erlene's complaints were so familiar, I fell asleep, not even caring I
was naked as the day I was born, not caring that melting ice dripped
onto places that should never be frozen, not caring I was dying.

When she drove up in front of the county hospital, I had to admire the
way she took charge. Told that guard exactly what to do. I knew how he
must be feeling. Something real comforting about a woman who knew her
own mind and never forgot anything.



\cleardoublepage
\chapter{Disappearing Act}

Betty had no choice. She was losing parts of herself, bit by bit,
sometimes a whole chunk. At 57, her bones were thinning. Her gall
bladder, tonsils, appendix, uterus, a lump in the right breast, two
teeth, strands of hair---all gone. Her disappearing act, she joked with
Charles last night before he told her he could not leave his wife of
forty years. He had no choice, he said; his wife counted on him.

Betty counted on no one except herself. All 300 pounds of her flesh
anchored her to her lawn chair on her rotted front porch. Boards were
broken, slanting down, missing. A chipped ceramic bowl rimmed in red and
full of pea pods from her garden rested in her lap. She snapped the first
peeped like it was Charlie's finger, the one that used to stroke the
surgical scar on her breast. Then, she emptied the other pea pods,
setting up a rhythm---snap, snap, snap.

She adjusted her glasses, watched the demolition of the boarded-up house
across the street as the wrecking ball swung at the roof, the walls, the
very foundation. Once she dated a man who destroyed houses. She had been
with all kinds of men---teachers, house painters, clerks, bikers---and
they all fell through.

Crack, a board splintered under her. Then another until she dropped
straight down, still sitting upright in her chair. Peas rolled like
pinballs in her hair. She looked straight up through the porch to the
right of the rotted gutters and all she saw was blue sky. Then she heard
a bell, a pinball scoring the jackpot, as the bald head of a demolition
worker peeked over the jagged boards and reached down to help her
up.


\cleardoublepage
\chapter{Driving Home}

At 65, Bernie lost his license. One too many times he got caught driving
recklessly.

Still, Bernie needed to get places. He rode buses, drove stolen cars,
caused accidents, and walked away.

When the stroke paralyzed Bernie's legs, he begged, doctors prescribed,
and Medicare approved his own electric cart---all chrome and zoom.

Bernie revved the motor, zipped off the top step into noise and
darkness. No license required.



\cleardoublepage
\chapter{For the Birds}

Bird poop slid down our picture windows, plopped on our heads as we ran
to our cars, polka-dotted our sidewalks, filled our nostrils with its
stench. Everyone blamed me. All because I helped Ms. Mannerly hang one
bird feeder from her cherry tree.

How was I to know Ms. Mannerly wouldn't stop with just one. She kept
adding until I counted seventy-seven feeders. ``One for each year of my
life,'' she giggled.

Holding an umbrella and covered in a garbage bag with eye holes, Bill
Stringer came barging into my house covered with bird droppings. I
dumped my microwaved beef dinner into the garbage and tried to explain
that the situation wasn't my fault.

Bill jabbed his garbage-bag finger at my chest. ``You better make it
stop or else.'' Next thing I heard was the door slam.

My answering machine was full of nasty messages each day. Every one of
my neighbors except Miss Mannerly played pranks, They filled paper bags
full of bird crap, set them on my front porch one night and set them on
fire. Must have been at least 30 bags. Ruined my best shoes stomping out
those fire bags.

I had already called the Humane Society, the Rescue Mission, police,
fire department, city council, fifteen lawyers, social services. Even
called the minister of the Baptist church at the corner of our street.
He offered to pray for me.

I was sick of birds. Ravens dive-bombed me every time I went outside
while geese hissed at me and pigeons strutted up and down the sidewalks
like they meant business. The seagulls swooped down to grab a chunk of
bread and one even snatched my lunch bag from my hand yesterday.

When Miss Mannerly came over this morning, her white hair was dark from
bird stuff and one raven feather stuck straight up at the back of her
head. ``For you,'' she said, swooping a red-checked vinyl tablecloth off
a domed wire cage. Inside was a parakeet, fluffing its powder blue
feathers.

``The color of the June sky today,'' Miss Mannerly said, setting the
cage on my dining room table. ``Birds, such lovely creatures. Yours will
sing and talk, if you let it.''

``Good to see you taking an interest again in something, Miss
Mannerly,'' I said. I worried about her since she came home from the
hospital this winter after her second heart attack. I took her chicken
soup which she poured down the drain, tried to talk with her and she was
silent, brought her fresh fruit which spoiled. I sat with her every day
saying nothing.

``The birds bring me messages,'' she said, leaning closer to me. ``They
tell me the sea is shrink­ing and our houses stand where birds once
lived. I build them new houses and feed them until their feathers shine.
They have no one but us,'' she confided.

``Our family is too large,'' I tell her. ``Like good parents, we must
let them go.''

``Go? I must protect my children,'' Miss Mannerly sobbed and the
parakeet screeched until I draped the tablecloth over the cage.

``We shall,'' I promised, knowing then what to do.

After she left, I ordered the biggest rental truck available. Then I
bought artificial grass, potted trees, netting, and pounds of bird seed.

When I had everything ready, I found Miss Mannerly under her cherry tree
in her back yard. Birds sat in her lap, perched on her shoulder, huddled
at her feet. ``They love you,'' I said, ``and we have a treat for them,
a special place. Remember the nature preserve where I took you last year
for a picnic?''

I saw that the bird she was caressing was dead. ``Bill Stringer's cat
got it,'' Miss Mannerly explained. ``Awful thing.''

``We will make the birds safe.''

I showed her the truck, the map, the birdseed. She smiled and the birds
followed her up into the truck. She chose to ride in back with her
family. It took us twenty trips over the next couple of weeks to set
them free. Now, we're both busy cleaning up the neighborhood.



\cleardoublepage
\chapter{Luck Rode My Shoulder}

When the ATM ate my card at midnight, I kicked that heap of metal twice
and drew back for a third kick when those twenties spewed out of the
slot like advertisements for how-to-get-rich-quick schemes.

``About time I had some luck,'' I muttered into the moonless night while
cramming my pockets with Andrew Jackson faces.

Until now, it had been a typical day. My boss at McDonald's fired me for
sassing a customer, my boyfriend took the last \$20 from my purse for
booze, my fat-butt landlord threatened to padlock my door and my kid's
high school math teacher called her slow just because she didn't like
the algebra alphabet of unknowns instead of real numbers.

Now, luck was riding my shoulder and I wasn't about to hand out any
transfers. My kid and I hooked up at Bob's Big Boy. Over double burgers,
vanilla shakes, and onion rings, my kid came up with a plan that would
take us places we had never been.

Over the next week, my kid and I got makeovers, the whole
shebang---face, hair, nails, clothes. We looked like any suburban mom
and daughter treating ourselves to whatever struck our fancy. Our
disguises fooled even us.

When we spent our last Andy, we sauntered into Comerica as natural as
you please, waited our turn in line, wiggled our fists in our jacket
pockets like we had guns and told that smiling cashier to fill our Sak's
bag with green---and with shaking hands, she did.

On television news that night, we admired our own blurry photos. We
hardly recognized our­selves. Wonder what my kid's math teacher would
say if he knew she planned the whole caper.

The next five years, we jumped up, down, and across every state,
stopping long enough to visit a few banks. We pasted our photos and
stories into a scrapbook we carried with us. Luck continued to ride my
shoulder like a Great Dane---sleek, loyal, pedigreed, and smart enough
to sniff out money.



\cleardoublepage
\chapter{A Man's Best Friend}

Mary read the personals to Spirit, her gray terrier. ``Men want too
much,'' Mary sighed. ``They want you---long walks, devotion, loyalty,
unconditional love.''

With Spirit licking her hand, Mary answered Macho Man's ad.

Macho Man responded, asking Spirit for a coffee date, demanding she
leave the unruly Mary at home where dogs belong.



\cleardoublepage
\chapter{The Numbers Game}

Jim Palonko never asked for a spirit guide, never wanted those little
balls of words pinging against his eardrum like some jazz improvisation.
He never knew what was coming next, which note would rumble and shake
the very chair where he sat. Like that spirit guide was boss. Like Jim
would do what he was told even though the swooshing in his ears hurt
deep down where he could not itch. He could not add the columns of
figures in front of him. Yet, he knew his real boss was waiting for the
number, the total of all. That final number.

Don't think Jim Palonko did not know numbers. He did. They had been
drummed into him at Walsh College where he went when his father paid for
him to be an accountant, which he did not want to be. Unlike his father,
Jim Palonko did not believe in the power of money or of CPAs. He did not
know what he believed in but it wasn't numbers.

Still, he became a numbers man stuck in a cubicle adding figures for
twenty years. Today, he knew what he wanted. He would fix his spirit
guide who set up house in his ear last year. He had a plan. He powered
up his cassette player, turned it up full volume to Muddy Rivers. So
loud the other voice was silenced. Then he lowered the volume, decibel
by decibel, until there was almost silence when he removed his
earphones, set them on his chair and walked away.

When he looked back, there was another man sitting in his chair. This
man had a shape, hummed a few bars of those pinging words that used to
roll around in his ear like seven zeros.



\cleardoublepage
\chapter{Performance Art}

Women loved Oliver. Oliver loved women but not himself. Oliver edited
his image, hired a PR agent. Olivia got her start pounding drums in an
all-girl band every Saturday downtown.



\cleardoublepage
\chapter{Playing for Love}

I can't believe I just heard that. Not again. But Ben lobs last night's
phrase at me once more during our tennis match: ``Irreconcilable
differences.''

What does he know about differences, about how he seems to crawl inside
the television to watch tennis matches, baseball, football, golf, even
bowling. All those balls, games won and lost. Never hears me, never pays
me one bit of attention. I could be anywhere, do anything. At least,
that's what I had counted on.

My return is out-of-bounds. ``Irreconcilable differences?'' All those
rolling ``r's'' send the yellow ball wobbling over the net, curving,
landing just outside the chalk with a soft plop.

Ben strides to the line, flips the ball up with his racket, raises his
arm, and serves. ``That's what I said, `Irreconcilable differences.'\,''

That ball slams into my gut and I'm down flat on the concrete, wind
knocked right out of me. Maybe it's the summer heat, over 100 degrees
today in Phoenix. I rise from the fire, questions flaming and burning my
mouth.

``Who says?''

``I say.'' He sweeps his racket from right to left, pointing out the
empty courts and taut nets. ``See anyone else?''

``No one else dumb enough to play in hell.''

``Anything can be reconciled. Give and take makes the ball bounce,'' I
plead as Ben turns his back, pulls at the crotch of his white shorts.

Silence. That red globe of sun beats down on both of us at high noon,
Ben's favorite time to play. A good workout, he used to say when we were
first married seven years ago. Since then, we've sweated in our bed, the
air conditioner going full blast, stained sheets twisting beneath our
bodies.

``Dick, Harry, Tom. Then, Sam,'' Ben chants, tapping his racket against
concrete in beat with each name. ``Working through the alphabet?'' He
admires the ball before he swings.

The ball hits my chest like a massive heart attack. I scream, fall, and
curl up, a small ball of myself.

``Sam. My tennis partner for more years than I've known you,'' Ben
shouts, walking around the net toward me. ``Every year we win the
doubles trophy.''

With each footstep, Ben curses, then stops. From now on, it's singles
for me.'' He sighs twice. ``This is a new game, sweetheart. Forty-love.
Next point, I win.''

I smell Ben's sweat. I hear the whoosh of his tennis racket through dry,
desert air. Up, then down. Metal edge against my skin. I hear him speak
or it may be me spitting out that phrase, ``Irreconcilable
differences.'' Each syllable pulses hot, moist as a forbidden kiss. I
can't believe I heard that. Again.



\cleardoublepage
\chapter{Rub Out}

``Call me despot, will he? That puny accountant,'' Gina's boss shrieks,
``Bring me all the erasers.'' A few swipes. His name disappears from the
floor plan.

``Don't,'' Gina pleads. She loves that puny guy who is no more.

``Now all men. Only holes in the plan.'' She caresses her blood-red
nails.

Gina draws the last eraser hidden in her bra.

Rub-a-dub, the boss's name is gone.



\cleardoublepage
\chapter{Sharing Space}

Sam told Rachel he had to work late tonight and wouldn't be able to see
her. She chuckled.

Sam didn't know that when Rachel meditated, she went outside herself,
not inside. Tonight she winged over rooftops, bounced off stars, zipped
through storm clouds, her third eye focused on her lover, Sam, a family
man.

On the back of her retina, Sam was upside-down, his arm around his wife
watching a comedy show, not laughing. Rachel's eye flipped the married
couple right-side-up in her mind and back again. She turned them like
glass fragments in a kaleidoscope, shifting their jagged edges in the
show she directed.

``Nothing's wrong,'' Sam said, removing his arm from his wife's shoulder
and wiping his fore­head with a tissue.

``I didn't ask,'' his wife retorted, narrowing her eyes as the man on
the sitcom broke an heir­loom vase practicing new tango steps. The actor
stuffed pieces of the broken glass vase in his jeans pocket.

``Men think they can get away with anything.'' When the glass poked
through the denim and cut flesh, his wife laughed. ``He better watch out
what he slices.''

``What?'' Sam snapped, reaching for the remote control.

``Oh, no you don't.'' She held the control in her fist. ``Just because
you don't want to see it, doesn't mean I can't.''

When Sam sniffed, he smelled Rachel's White Shoulder perfume as if she
were sitting on his lap, caressing his chest, reaching into his pockets.
Sam held his breath, not wanting to speak.

``I don't smell anything,'' his wife said, as if Sam had spoken aloud.

``Hear the wind chimes?'' Sam recalled Rachel's wind chimes hanging all
over her house, knew they bumped together in the slightest breeze making
a tinkling sound that always startled him.

``We don't have wind chimes,'' his wife said as the actor on TV sat at
the Formica kitchen table. He was trying to glue the vase back together
so his wife wouldn't know he broke it. ``Won't work,'' his wife
predicted.

``Might,'' Sam said, even though he was no good at repair work.

``No way,'' Rachel whispered in Sam's right ear.

Sam jumped up from the couch, his eyes darting everywhere. Blood dripped
down his leg as if he had opened old wounds and let them flow freely. He
limped toward the bathroom. Slivers of glass dropped from his pocket,
left a trail from where he started to where he ended up.

``You're making a mess,'' his wife scolded. She chortled at the guy in
the sitcom who had glued his fingers to the broken glass.

Sam's wife took one deep breath after another trying to name the perfume
filling her living room, scenting her space with another woman. ``Sam,''
she screeched.

Rachel winked her third eye. This production was a wrap. She turned the
kaleidoscope once more, wondered where the glass chips would fall.



\cleardoublepage
\chapter{Still a Novice:\\
The Mystery Food Reviewer at Work}

I have my scruples, you know. Even for a hundred dollars, I won't
whisper my name in your ear or write it on your hand. No way will I
confess where I've been or where I'm going. At least, not to you I
won't. Not to any of you.

But who I am is not really an issue since none of you even ask. You pass
within an inch of me. Yet, you dare say nothing. You avert your eyes and
pull your wool coats around your necks on this cold night when the chill
wind skitters across the river.

Tonight I hunger for warmth, for comfort food, for Northern Italian
cuisine, either Chicken Piccata or Fettuccine Gabriella, something
different from the authentic Nepal dishes I indulged in last night. I
can still savor the manakamana sekuwa cooked in the Tandoor oven, even
though the lamb chops were a bit leathery and the Himalayan spices
scorched my tongue.

Actually, it is quite cozy by the foot of the dumpster where I dine. I
insist on smoothing out my white linen tablecloth on the concrete behind
the fancy schmantzy restaurant, the trendy place where those in the know
order dishes such as Provimi Veal chops or Veal Saltimbocca.

I set my place with my grandmother's china, crystal glasses and
silverware. I slip on plastic gloves and an apron to sort through the
food placed into the dumpster. I uncover small servings of chargrilled
lamb chops, broiled salmon, medallions of veal, lasagna verde alla
Bolognese---and finally, a bit of chicken piccata nestled against a
crumpled brown bag. My saliva glands work overtime as I scoop it onto my
plate and position it just right.

Each bite ignites my taste buds. The chicken breast is moist and flaky.
The capers are pungent, strong enough to make their presence known. The
baby artichokes are too grown up, too full of them­selves, too
overwhelming.

You can expect attentive service with a starlight atmosphere if you
accept alternative seating. It's worth the price. Only eleven pages
remain in the spiral notebook which I pull out of my coat pocket along
with a stubby pencil with no eraser. Getting down to work, I fill the
lines with words, the more words the better. Words like \emph{capers,
mushrooms, potatoes, tomatoes, carrots} sprout across the lines,
straight rows of words fresh as just-pulled scallions.

Read my words carefully. After all, I get paid for each one. Don't ask
my name or look for me where you think I might be found munching on a
melon. I must dine invisibly until I earn enough to eat inside.



\cleardoublepage
\chapter{Takin' Care of Business}

``You,'' I shout to my buddy who's goin' up the fire escape two steps at
a time. ``A big rat got you runnin' scared, Sinker?''

``No rodent gonna take a bite of me. Got me, Skintight?'' He throws his
laugh down those iron steps so hard it bounce off the rust and shatter
on my shaved head stickin' out the third floor window.

I let Sinker have some time alone on the roof with his weeds. He don't
like being bothered when he's tendin' to business.

I wait `til the sun drops some more in the sky before tip-tappin' up
after him. Don't know one sneak up on Sinker without payin' so I start
whistlin' and snappin' my fingers to make myself known.

Sinker's one of them big dudes. Everyone know Sinker and no one mess
with him. If he just sits on `em, they count `emselves gone. Ya know
what I'm sayin'?

``Who you be talkin' to? Ya goin' crazy on me, Skintight?'' He punch me
hard in the head. ``Here, go hang these weeds.''

I done like he said cause we was friends. ``Wanna go get us some of them
catfish from the River tomorrow? Have us a down home fish fry?''

``Ain't had me none of them since last month,'' Sinker said, wipin'
sweat from his head. ``Cooler there by the Deee-troit River. Let's go.''

``Tomorrow,'' I says as the sun drop further and the moon start
climbin'. ``Don't have our rods,'' I remind him.

``Makes no never mind. We borrow some rods,'' and he laughs like mama's
boyfriend.

When we gets down there, Sinker grab us two rods right off. No one says
nothin'. They knows what's good for `em. We catch us some big suckers
before everything falls apart. You knows how that happens. Everything
sweet like sugar. Then, bam---and it all falls apart. That's how this
went.

Sinker snatch that hook out of the fish mouth. Then, he get all wrapped
up in the line. Then come the fall. A really big splash cause Sinker a
big sucker himself. I never swim before but I jump in after my friend.
My hand grab the wood stickin' up. I don't mean to hold on but no way
can I get my fingers unstuck.

``Over here, Sinker.'' Then I spot the big rat just sittin' on the wood
above me. He watchin' like he want to jump in after Sinker. But that rat
just sit there, raise his paw almos' like he tellin' time.

Sinker kick best he can but he go down and come up with a mouth full of
dirty water. That rat just lick his pay like nothin' happennin'. I reach
out my hand to Sinker but he go down and don't come up again.

Ain't no one else to watch Sinker's weeds but me. Sinker would be proud.
I takes good care of `em.



\cleardoublepage
\chapter{Temptation}

There is no Agatha Christie story. Not even a novel by that woman who
sleuths by alphabet. No Sam Spades required. Just a simple mystery.

Who ate that last chocolate-covered cherry?

Not ants who cannot carry boxes. Only you, my love. While I slept, you
saved me from myself, from craven, calorie-laden lust.



\cleardoublepage
\chapter{That's Who I Am}

You may not know my name but you've seen me around. I'm the guy you know
from some­place but can't quite remember, the one you smile and wave at
because I may be a neighbor, the clerk at Home Depot, the man with six
children you see at church on Sundays, or even someone famous.

On my way home from the Engine Plant, I yield the right-of-way, avoid
orange construction barrels on Telegraph, watch my speed, keep six car
lengths from the Explorer. In my rearview mirror, I glimpse a Harley
weaving in and out of traffic, spraying rocks as it swerves onto the
shoulder and back onto the rutted road. That hog closes the distance
between us. I see the helmeted driver hunched forward as the front tire
nudges my bumper. Crowds me, just like my supervisor did all day. That
supervisor's mouth is full of rotten teeth and brown words.

When the stoplight ahead turns yellow, I speed up, then slam on my
brakes. That hog smashes me, falls on its side. The rider raises his
face shield. Blood streaks his lips and I confess: ``If I could ride a
Harley, top speed on a straightway, I wouldn't wear a helmet.''

He asks my name.



\cleardoublepage
\chapter{White Sunfire}

``You might call it entrapment,'' Danielle said as we stumbled over
curbs, beer cans, broken pavement, marking our way to her gallery
exhibit in a renovated loft on Michigan Avenue. ``At least that's how
the \emph{Art News} critic labeled the pieces.''

I never knew what to expect from Danielle's work, I thought as she
snatched up a shard of green glass and peered through it.

``Leprechauns, drunks, tulip stems,'' Danielle mused. ``An emerald
abandoned on these Detroit streets for me.''

``How will you use it?'' I was curious. I wondered if she knew, if she
ever moved in one direc­tion like I did. I thought of myself as heading
east, across an ocean and toward temples of gold where the soul could
rest. Not a vacation but a past life where I knew who I was and what I
wanted.

``When it speaks to me of greenness, then I will know.'' She held the
dirty piece of glass to her ear before rubbing it clean against her
sleeve. ``Ah, such tales are melted in green.''

As we moved along, Danielle stuffed her pockets with found cash register
receipts, pebbles, bottle caps, pens, cigarette butts, and objects lying
in the gutter waiting for her to snatch them up, rescue them. She turned
these objects into art, into pulsating power that grabbed you around the
neck and wouldn't let go.

In the autumn dusk, I was confused by these city streets as Danielle cut
through alleys, turned corners in a neighborhood I did not recognize. I
shivered, hoping people we passed did not know I was lost and dependent
on my artist friend to get us where we were going.

When Danielle touched my arm, I jumped, startled at the warmth of her
hand through my jacket.

``Here, Danielle said, pointing to the gallery sign. Its name, Aloft,
was painted in gold script on a forest green background.

The freight elevator had brass grillwork, polished to a sheen that
blinded me momentarily. Danielle pushed the button and the grille
parted. Then, green elevator doors opened. Slowly, we rose in a jerky
motion that threw me off balance.

When the doors opened again all I saw was stark white. Nothing but white
walls and one white canvas hung on a pillar in front of us.

``Come on,'' Danielle urged, guiding me past the pillar and inside the
huge room. ``What do you see?''

``White,'' I stammered. ``Lots of white.''

``Over here. Look at this canvas,'' Danielle directed. ``This one was
done for you, my friend.''

As I stared at the whiteness, I began to see brush strokes. Then the
texture, the shapes covered with white. ``There,'' I said, ``A
matchbook, I think.''

``To light your way.'' Danielle laughed. ``The dark scares you.''

``Never did,'' I argued. I began to see more, at least I thought I did.
``A light bulb, a stick, a lighter, a flare, and the shape of the sun.''

``Yes.'' Danielle spoke softly. ``You see. Keep going.''

The painting pulled me closer and closer until my face pressed against
it. ``Color,'' I whispered. ``Yes, gold, full of yellow, orange, red
fire.'' My face flushed with heat. I felt drowsy, peaceful as if I were
in a safe place I had never been. ``Enchanted work,'' I could hear
myself tell her. ``Your very best ever.''

``For you, my friend,'' Danielle said. ``I listen. I paint your dream
over with white. You see with your own eyes how it is love I paint, not
entrapment.''

\clearpage
\pagestyle{empty}
\mbox{ }

\cleardoublepage
\mbox{ }

\cleardoublepage
\mbox{ }

\cleardoublepage
\mbox{ }

\clearpage
\mbox{ }
\end{document}
